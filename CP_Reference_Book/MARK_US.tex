% Change 'article' to 'extarticle' so the 8pt option actually works!
\documentclass[8pt,a4paper,landscape]{extarticle} 

% --- PREAMBLE ---
\usepackage[utf8]{inputenc}
\usepackage[T1]{fontenc}
\usepackage[english]{babel}
\usepackage{graphicx}
\usepackage{amsmath}
\usepackage{amssymb}
\usepackage{amsfonts}
\usepackage{extsizes} % Guarantees 8pt base font size works
\usepackage{microtype} % Improves text density and justification
\usepackage{multicol} 
\usepackage{xcolor}
\usepackage{fancyhdr}
\usepackage{listings} 
\usepackage{enumitem}
\usepackage{titlesec}
% Tight margins for maximum space
\usepackage[top=0.3in, bottom=0.2in, left=0.15in, right=0.15in, includehead]{geometry}

% --- LAYOUT & STYLING ---
\setlength{\columnsep}{5pt}
\setlength{\columnseprule}{0.2pt} 
\setlength{\parindent}{0pt}

% --- TITLE & SECTION SPACING ---
\titlespacing*{\section}{0pt}{5pt}{2pt}    
\titlespacing*{\subsection}{0pt}{5pt}{2pt}   

% --- CODE LISTING STYLE ---
\definecolor{codegreen}{rgb}{0,0.6,0}
\definecolor{codegray}{rgb}{0.5,0.5,0.5}
\definecolor{codepurple}{rgb}{0.58,0,0.82}
\definecolor{backcolour}{rgb}{0.98,0.98,0.98}

\lstset{
    language=C++,
    backgroundcolor=\color{backcolour},
    keywordstyle=\color{magenta}\bfseries,
    commentstyle=\color{codegreen},
    stringstyle=\color{codepurple},
    numberstyle=\tiny\color{codegray},
    basicstyle=\fontsize{7}{8}\selectfont\ttfamily, % Defines code font size
    tabsize=2,
    breaklines=true,
    breakatwhitespace=false,
    showstringspaces=false,
    numbers=none,
    frame=t,
    aboveskip=2pt,
    belowskip=2pt,
    xleftmargin=0pt,
    postbreak=\mbox{\textcolor{red}{$\hookrightarrow$}\space}
}

% --- HEADER CONFIGURATION ---
\pagestyle{fancy}
\fancyhf{}
\renewcommand{\headrulewidth}{0pt}
\lhead{\textcolor{blue}{\textbf{International Islamic University Chittagong}}}
\chead{\textcolor{blue}{\textbf{IIUC\_MARK\_US}}}
\rhead{\thepage}

% --- COMPRESSION & LAYOUT TWEAKS ---
\setlength{\parindent}{0pt}
\setlength{\parskip}{0pt} % Remove paragraph skip completely
\setlist[itemize]{nosep, leftmargin=*} % Ultra-compact item lists
\setlist[enumerate]{nosep, leftmargin=*} % Ultra-compact numbered lists

% --- TITLE FORMATTING (Visual Separator for Algorithms) ---
\definecolor{darkgray}{rgb}{0.3,0.3,0.3}
\titleformat{\subsection}
    {\color{darkgray}\normalfont\bfseries}
    {}
    {0em}
    {}
    [\color{darkgray}\titlerule] % Adds a horizontal rule under the title
\titlespacing*{\subsection}{0pt}{5pt}{3pt} % Minimal vertical spacing

% --- DOCUMENT ---
\begin{document}

% --- FRONT PAGE ---
\begin{titlepage}
    \centering
    \vspace*{\fill}
    
    \includegraphics[height=4cm]{iiuclogo.png} 
    \\[1cm]
    
    {\Huge \textbf{\textcolor{blue}{International Islamic University Chittagong}}} 
    \\[1.5cm]
    
    {\fontsize{40}{48}\selectfont \textbf{IIUC\_MARK\_US}} 
    \\[1cm]
    
    {\LARGE \textbf{Arman, Istiaque, Mizan}}
    
    \vspace*{\fill}
\end{titlepage}
\newpage

% Start 4-Column Layout
\begin{multicols*}{4}

% ================= GRAPH & TREE =================
\section*{Graph \& Tree}

\subsection*{DSU}
{ \footnotesize
\textbf{Description:} Disjoint-set data structure with path compression and union by size. Supports \texttt{unite} and \texttt{findpar}. \\
\textbf{Time:} $\mathcal{O}(\alpha(N))$, where $\alpha$ is the inverse Ackermann function ($\approx$ constant).
}
\begin{lstlisting}
{
public:
  vector<int> parent, size;
  int comp;
  DSU(int n) {
    parent.resize(n + 1, 0);
    size.resize(n + 1, 0);
    for (int i = 0; i <= n; i++) {
      parent[i] = i;
      size[i] = 1;
    }
    comp = n;
  }
  int findpar(int node) {
    if (node == parent[node])
      return node;

    return parent[node] = findpar(parent[node]);
  }
  void unite(int u, int v) {
    int ulpar_u = findpar(u);
    int ulpar_v = findpar(v);
    if (ulpar_u == ulpar_v)
      return;
    if (size[ulpar_u] < size[ulpar_v])
      swap(ulpar_u, ulpar_v);
    parent[ulpar_v] = ulpar_u;
    size[ulpar_u] += size[ulpar_v];
    comp--;
  }
};
\end{lstlisting}

\subsection*{SPFA (Shortest Path Faster Algo)}
{ \footnotesize
\textbf{Description:} Queue-optimized Bellman-Ford. Computes single-source shortest paths and detects negative cycles. \\
\textbf{Time:} Average $\mathcal{O}(E)$, Worst $\mathcal{O}(VE)$.
}
\begin{lstlisting}
vector<int> dis(node + 1, inf);
queue<int> q;
vector<int> count(node + 1, 0);
vector<bool> inqueue(node + 1, false);
dis[1] = 0;
q.push(1);
while (!q.empty()) {
  int node = q.front();
  q.pop();
  inqueue[node] = false;
  for (auto it : graph[node]) {
    int newnode = it[0];
    int wt = it[1];
    if (dis[newnode] > dis[node] + wt) {
      dis[newnode] = dis[node] + wt;
      if (!inqueue[newnode]) {
        q.push(newnode);
        inqueue[newnode] = true;
        count[newnode]++;
        if (count[newnode] > node) {
          cout << "Negative Cycle Found" << endl;
          return;
        }
      }
    }
  }
}
\end{lstlisting}

\subsection*{Floyd Warshall}
{ \footnotesize
\textbf{Description:} All-pairs shortest path algorithm. Works with negative edges (no negative cycles). \\
\textbf{Time:} $\mathcal{O}(V^3)$.
}
\begin{lstlisting}
for (int k = 1; k <= nodes; k++) {
    for (int i = 1; i <= nodes; i++) {
        for (int j = 1; j <= nodes; j++) {
            graph[i][j] = min(graph[i][j], graph[i][k] + graph[k][j]);
        }
    }
}
\end{lstlisting}

\subsection*{Dijkstra}
{ \footnotesize
\textbf{Description:} Single-source shortest path for non-negative edge weights using a priority queue. \\
\textbf{Time:} $\mathcal{O}(E \log V)$.
}
\begin{lstlisting}
priority_queue<array<long long, 2>, vector<array<long long, 2>>, greater<array<long long, 2>>> pq;
int n = destination + 1;
vector<long long> dist(n, LONG_LONG_MAX);
vector<int> parent(n);
iota(parent.begin(), parent.end(), 0);

pq.push({0, source});
dist[source] = 0;

while (!pq.empty()) {
    int node = pq.top()[1];
    long long wt = pq.top()[0];
    pq.pop();
    if (wt > dist[node]) continue;
    for (auto it : graph[node]) {
        int newnode = it[0];
        long long newwt = it[1];
        if (dist[node] + newwt < dist[newnode]) {
            dist[newnode] = dist[node] + newwt;
            pq.push({dist[newnode], newnode});
            parent[newnode] = node;
        }
    }
}
\end{lstlisting}

\subsection*{SCC (Kosaraju)}
{ \footnotesize
\textbf{Description:} Finds strongly connected components using two DFS passes. Requires \texttt{rev[]} (transpose graph). \\
\textbf{Time:} $\mathcal{O}(V + E)$.
}
\begin{lstlisting}
// Assume graph[] and rev[] (reverse graph) are built
{
  int u, v;
  cin >> u >> v;
  graph[u].pb(v);
  rev[v].pb(u);
}
vector<int> vis(n + 1, 0);
vector<int> order;
auto get = [&](auto &&self, int node) -> void {
  vis[node] = 1;
  for (auto it : graph[node]) {
    if (vis[it]) continue;
    self(self, it);
  }

  order.pb(node);
};

for (int i = 1; i <= n; i++) {
  if (vis[i])
    continue;
  get(get, i);
}
vis.assign(n + 1, 0);
reverse(all(order));
vector<int> cur;
vector<int> comp_id(n + 1, 1);
vector<vector<int>> component;
auto rec = [&](auto &&self, int node, int root, int cid) -> void {
  cur.pb(node);
  comp_id[node] = cid;
  vis[node] = 1;
  for (auto it : rev[node]) {
    if (vis[it]) continue;
    self(self, it, root, cid);
  }
};
component.pb({0});
for (auto it : order) {
  if (vis[it]) continue;
  int c = component.size();
  rec(rec, it, it, c);
  component.pb(cur);
  cur.clear();
}

int sz = component.size() - 1;
vector<vector<int>> scc(sz + 5);
for (int u = 1; u <= n; u++) {
  int compu = comp_id[u];
  for (auto v : graph[u]) {
    int compv = comp_id[v];
    if (compu != compv) {
      scc[compu].pb(compv);
    }
  }
}
for (int i = 1; i <= sz; i++) {
  sort(scc[i].begin(), scc[i].end());
  scc[i].erase(unique(scc[i].begin(), scc[i].end()), scc[i].end());
}
\end{lstlisting}

\subsection*{LCA (Binary Lifting)}
{ \footnotesize
\textbf{Description:} Lowest Common Ancestor using binary lifting. \texttt{kth} returns the $k$-th ancestor. \\
\textbf{Time:} Build $\mathcal{O}(N \log N)$, Query $\mathcal{O}(\log N)$.
}
\begin{lstlisting}
int LOG = 1;
while ((1 << LOG) <= n) ++LOG;
vector<vector<int>> up(n + 1, vector<int>(LOG + 1, 0));
vector<vector<int>> mx(n + 1, vector<int>(LOG + 1, 0));
vector<vector<int>> mn(n + 1, vector<int>(LOG + 1, 1e9));

auto rec = [&](auto &&self, int node, int par, int cur) -> void {
  parent[node] = par;
  if (par != 0) depth[node] = depth[par] + 1;

  up[node][0] = par;
  mx[node][0] = cur;
  mn[node][0] = cur;
  for (int i = 1; i < LOG; i++) {
    int prev = up[node][i - 1];
    up[node][i] = up[prev][i - 1];
    mx[node][i] = max(mx[node][i - 1], mx[prev][i - 1]);
    mn[node][i] = min(mn[node][i - 1], mn[prev][i - 1]);
  }

  for (auto it : graph[node]) {
    if (it.ff == par)
      continue;
    self(self, it.ff, node, it.ss);
  }
};
rec(rec, 1, 0, 0);
auto kth = [&](int node, int k) -> array<int, 3> {
  int mxx = 0, mnn = 1e9;
  for (int i = 0; i < LOG; i++) {
    if ((1 << i) & k) {
      mxx = max(mxx, mx[node][i]);
      mnn = min(mnn, mn[node][i]);
      node = up[node][i];
    }
  }
  return {node, mnn, mxx};
};
auto lca = [&](int u, int v) -> pair<int, int> {
  int mxx = 0, mnn = 1e9;
  if (depth[u] > depth[v]) {
    auto it = kth(u, depth[u] - depth[v]);
    u = it[0];
    mnn = it[1];
    mxx = it[2];
  }
  else if (depth[v] > depth[u]) {
    auto it = kth(v, depth[v] - depth[u]);
    v = it[0];
    mnn = it[1];
    mxx = it[2];
  }

  if (u == v)
    return {mnn, mxx};

  for (int i = LOG - 1; i >= 0; i--) {
    if (up[u][i] != up[v][i]) {
      mxx = max({mxx, mx[u][i], mx[v][i]});
      mnn = min({mnn, mn[u][i], mn[v][i]});
      u = up[u][i];
      v = up[v][i];
    }
  }
  mxx = max({mxx, mx[u][0], mx[v][0]});
  mnn = min({mnn, mn[u][0], mn[v][0]});
  return {mnn, mxx};
};
\end{lstlisting}

\subsection*{Centroid Decomposition}
{ \footnotesize
\textbf{Description:} Decomposes a tree into a tree of centroids (depth $\mathcal{O}(\log N)$). \texttt{update}/\texttt{qry} example solves min distance to marked nodes. \\
\textbf{Time:} Construction $\mathcal{O}(N \log N)$, Queries $\mathcal{O}(\log N)$ or $\mathcal{O}(\log^2 N)$.
}
\begin{lstlisting}
vector<int> used(n + 1), size(n + 1), parent(n + 1);
vector<int> ans(n + 1, 2e5);
function<int(int, int)> get_size = [&](int node, int par) {
  size[node] = 1;
  for (auto it : graph[node]) {
    if (it == par or used[it])
      continue;
    size[node] += get_size(it, node);
  }
  return size[node];
};
function<int(int, int, int)> get_cen = [&](int node, int par, int sz) {
  for (auto it : graph[node]) {
    if (it == par or used[it])
      continue;
    if (size[it] > sz / 2)
      return get_cen(it, node, sz);
  }
  return node;
};
function<void(int, int)> decompose = [&](int node, int par) {
  int sz = get_size(node, 0);
  int cen = get_cen(node, 0, sz);
  used[cen] = 1;
  if (par == 0)
    par = cen;
  parent[cen] = par;
  for (auto it : graph[cen]) {
    if (used[it])
      continue;
    decompose(it, cen);
  }
};
auto process=[&](int cent)->void {
    vector<int> nodes;
    cnt[0]=1;
    for(auto it : graph[cent]) {
        if(used[it]) continue;
        vector<int> sub_dis;
        auto get_dis=[&](auto &&self,int node,int par,int dis)->void {
            if(dis>k) return;
            sub_dis.pb(dis);
            for(auto it : graph[node]) {
                if(it==par or used[it]) continue;
                self(self,it,node,dis+1);
            }
        };
        get_dis(get_dis,it,cent,1);
        for(auto d : sub_dis) {
            ans+=cnt[k-d];
        }
        for(auto d : sub_dis) {
            if(d<=k) {
                cnt[d]++;
                nodes.pb(d);
            }
        }
    }
    for(auto d : nodes) {
        cnt[d]--;
    }
};
    
function<void(int)> update = [&](int cur) {
  int x = cur;
  ans[cur] = 0;
  while (1) {
    ans[x] = min(ans[x], getdis(x, cur));
    if (parent[x] == x)
      break;
    x = parent[x];
  }
};
function<int(int)> qry = [&](int cur) {
  int x = cur;
  int go = ans[x];
  while (1) {
    go = min(go, getdis(x, cur) + ans[x]);
    if (parent[x] == x)
      break;
    x = parent[x];
  }
  return go;
};
decompose(1, 0);
update(1);
while (q--) {
  int type;
  cin >> type;
  if (type == 1) {
    int u;
    cin >> u;
    update(u);
  }
  else {
    int u;
    cin >> u;
    cout << qry(u) << el;
  }
}
\end{lstlisting}

\subsection*{Block cut tree}
\begin{lstlisting}
const int N = 200005; // Max nodes (adjust as needed)
vector<int> adj[N];          // Original Graph
vector<int> tree_adj[2 * N]; // Block-Cut Tree (Size 2*N because of block nodes)
int tin[N], low[N];
int timer;
vector<int> stk;
int block_cnt; // Counts the number of blocks found
int n;         // Number of original nodes

// DFS to find Biconnected Components and build the tree
void dfs_bct(int u, int p = -1) {
    tin[u] = low[u] = ++timer;
    stk.push_back(u);
    
    for (int v : adj[u]) {
        if (v == p) continue;
        
        if (tin[v]) {
            // Back-edge
            low[u] = min(low[u], tin[v]);
        } else {
            // Tree-edge
            dfs_bct(v, u);
            low[u] = min(low[u], low[v]);
            
            // Check for Articulation Point / Block condition
            if (low[v] >= tin[u]) {
                block_cnt++;
                int block_node = n + block_cnt; // New node index for this block
                
                // Add edge between Articulation Point u and the Block Node
                tree_adj[u].push_back(block_node);
                tree_adj[block_node].push_back(u);
                
                // Pop all nodes in this component from stack
                while (true) {
                    int node = stk.back();
                    stk.pop_back();
                    
                    // Link component node to block node
                    // (Avoid adding u again if it was already added above)
                    if (node != u) {
                        tree_adj[node].push_back(block_node);
                        tree_adj[block_node].push_back(node);
                    }
                    if (node == v) break;
                }
            }
        }
    }
}

void build_bct() {
    timer = 0;
    block_cnt = 0;
    stk.clear();
    
    // Clear previous tree adjacency if reusing
    for(int i = 1; i <= n * 2; i++) {
        tree_adj[i].clear();
        tin[i] = 0; // 0 means unvisited
    }

    for (int i = 1; i <= n; i++) {
        if (!tin[i]) {
            dfs_bct(i);
            // Handle isolated nodes or leftover stack if needed, 
            // but the loop covers standard components.
            if (!stk.empty()) stk.pop_back(); 
        }
    }
}
\end{lstlisting}

\subsection*{Bridge and Articulation point}
{ \footnotesize
\textbf{Description:} Finds Bridges and Articulation Points in an undirected graph using DFS entry times (\texttt{tin}) and low-link values (\texttt{low}). \\
\textbf{Time:} $\mathcal{O}(V + E)$.
}
\begin{lstlisting}
/*
  Finds articulation points and bridges in an undirected simple graph.
  - Nodes are 1..n
  - Input: adjacency list `adj` where adj[u] contains neighbors of u
  - Output:
      vector<int> is_cut(n+1)  : is_cut[u] == 1 if u is an articulation point
      vector<pair<int,int>> bridges : list of bridges (u,v) with u < v
  Usage:
    build adj (size n+1), then call find_articulation_and_bridges(n, adj, is_cut, bridges)
  tin[v] = discovery time of v in DFS.
  low[v] = smallest discovery time reachable from the subtree of v via at most one back edge (i.e., possibly going up to an ancestor).
  If for a child to of v, low[to] > tin[v], then there's no back edge from to's subtree that reaches v or an ancestor of v. So removing the edge (v,to) disconnects the graph $\rightarrow$ a bridge.
If for a child to of non-root v, low[to] >= tin[v], then removing v disconnects to's subtree from the rest $\to$ v is an articulation point. The root is special: it is an articulation point only if it has $\ge 2$ children in the DFS tree.
*/
void dfs_art_bridge(int v, int p,
const vector<vector<int>> &adj
,vector<int> &tin, vector<int> &low
,vector<int> &is_cut, vector<pair<int, int>> &bridges, int &timer) {
  tin[v] = low[v] = ++timer;
  int children = 0;
  for (int to : adj[v]) {
    if (to == p)
      continue; // skip the edge back to parent (simple graph)
    if (tin[to]) { // back edge
      low[v] = min(low[v], tin[to]);
    }
    else { // tree edge
      ++children;
      dfs_art_bridge(to, v, adj, tin, low, is_cut, bridges, timer);
      low[v] = min(low[v], low[to]);

      // bridge condition (strict)
      if (low[to] > tin[v]) {
        int a = v, b = to;
        if (a > b)
          swap(a, b);
        bridges.emplace_back(a, b);
      }
      // articulation point (non-root)
      if (p != -1 && low[to] >= tin[v]) {
        is_cut[v] = 1;
      }
    }
  }
  // root articulation check
  if (p == -1 && children > 1)
    is_cut[v] = 1;
}
void find_articulation_and_bridges
    (int n, const vector<vector<int>> &adj,
           vector<int> &is_cut,
            vector<pair<int, int>> &bridges)
{
  is_cut.assign(n + 1, 0);
  bridges.clear();
  vector<int> tin(n + 1, 0), low(n + 1, 0);
  int timer = 0;
  for (int i = 1; i <= n; ++i) {
    if (!tin[i])
      dfs_art_bridge(i, -1, adj, tin, low, is_cut, bridges, timer);
  }
  sort(bridges.begin(), bridges.end()); // optional: sorted list of bridges
}
int main() {
  vector<int> is_cut;
  vector<pair<int, int>> bridges;
  find_articulation_and_bridges(n, adj, is_cut, bridges);

  // print articulation points
  vector<int> cuts;
  for (int i = 1; i <= n; ++i)
    if (is_cut[i])
      cuts.push_back(i);
  cout << "Articulation points (" << cuts.size() << "):";
  for (int x : cuts)
    cout << ' ' << x;
  cout << '\n';

  // print bridges
  cout << "Bridges (" << bridges.size() << "):\n";
  for (auto &e : bridges)
    cout << e.first << ' ' << e.second << '\n';
}
\end{lstlisting}
% ================= STRING =================
\section*{String}

\subsection*{Hashing}
{ \footnotesize
\textbf{Description:} Double rolling hash using two sets of mods/bases to minimize collisions. Supports $\mathcal{O}(1)$ substring hash queries after $\mathcal{O}(N)$ precomputation. Uses 1-based indexing for queries. \\
\textbf{Time:} Construction $\mathcal{O}(N)$, Query $\mathcal{O}(1)$.
}
\begin{lstlisting}
constexpr int mod1 = 1000012253;
constexpr int mod2 = 1000000009;
constexpr int base1=163;
constexpr int base2=271;
template<typename T>
class MultiHashing {
public:
    int n;
    string s;
    string rev;
    vector<pair<T, T>> prefix_hash;
    vector<pair<T, T>> suffix_hash;
    vector<pair<T, T>> power;
    vector<pair<T, T>> inv;
    T mul(T a, T b, T mod) {
        return ((1LL * a % mod) * (b % mod)) % mod;
    }
    T add(T a, T b, T mod) {
        return (1LL * a + b) % mod;
    }
    T sub(T a, T b, T mod) {
        return ((a % mod) - (b % mod) + 2LL * mod) % mod;
    }
    T bigmod(T base, T power, T mod) {
        T res = 1;
        while (power > 0) {
            if (power & 1) {
                res = mul(res, base, mod);
            }
            base = mul(base, base, mod);
            power >>= 1;
        }
        return res;
    }
    MultiHashing(const string& str) : s(str) {
        n = s.size();
        rev = s;
        reverse(rev.begin(), rev.end());
        prefix_hash.resize(n + 1, {0, 0});
        suffix_hash.resize(n + 1, {0, 0});
        power.resize(n + 1, {0, 0});
        inv.resize(n + 1, {0, 0});
        precom();
    }
    void precom() {
        power[0] = {1, 1};
        for (int i = 1; i <= n; i++) {
            power[i].first = mul(power[i - 1].first, base1, mod1);
            power[i].second = mul(power[i - 1].second, base2, mod2);
        }
        T inv_base1 = bigmod(base1, mod1 - 2, mod1);
        T inv_base2 = bigmod(base2, mod2 - 2, mod2);
        inv[0] = {1, 1};
        for (int i = 1; i <= n; i++) {
            inv[i].first = mul(inv[i - 1].first, inv_base1, mod1);
            inv[i].second = mul(inv[i - 1].second, inv_base2, mod2);
        }
        for (int i = 1; i <= n; i++) {
            int ch = s[i - 1] - 'a' + 1;
            prefix_hash[i].first = add(prefix_hash[i - 1].first, mul(ch, power[i - 1].first, mod1), mod1);
            prefix_hash[i].second = add(prefix_hash[i - 1].second, mul(ch, power[i - 1].second, mod2), mod2);
            ch = rev[i - 1] - 'a' + 1;
            suffix_hash[i].first = add(suffix_hash[i - 1].first, mul(ch, power[i - 1].first, mod1), mod1);
            suffix_hash[i].second = add(suffix_hash[i - 1].second, mul(ch, power[i - 1].second, mod2), mod2);

        }
    }
    pair<T, T> get_hash(int l, int r) {
        T val1 = sub(prefix_hash[r].first, prefix_hash[l - 1].first, mod1);
        val1 = mul(val1, inv[l].first, mod1);

        T val2 = sub(prefix_hash[r].second, prefix_hash[l - 1].second, mod2);
        val2 = mul(val2, inv[l].second, mod2);

        return {val1, val2};
    }
    pair<T, T> get_hash_rev(int l, int r) {
        T val1 = sub(suffix_hash[r].first, suffix_hash[l - 1].first, mod1);
        val1 = mul(val1, inv[l].first, mod1);
        T val2= sub(suffix_hash[r].second, suffix_hash[l - 1].second, mod2);
        val2 = mul(val2, inv[l].second, mod2);
        return {val1, val2};
    }
    pair<T, T> combine_hash(pair<T, T> h1, pair<T, T> h2, int l1) {
        T val1 = add(h1.first, mul(h2.first, power[l1].first, mod1), mod1);
        T val2 = add(h1.second, mul(h2.second, power[l1].second, mod2), mod2);
        return {val1, val2};
    }
};
\end{lstlisting}

\subsection*{Trie}
{ \footnotesize
\textbf{Description:} Prefix tree. \texttt{insert} adds string, \texttt{count} checks existence, \texttt{erase} lazily removes. \texttt{pref} counts words passing through node, \texttt{end} counts words ending at node. \\
\textbf{Time:} $\mathcal{O}(|S| \cdot \Sigma)$ per operation.
}
\begin{lstlisting}
class Trie {
public:
    static const int N=26;
    struct Node {
        int next[N];
        int pref;
        int end;
        Node() {
            fill(next,next+N,-1);
            pref=0;
            end=0;
        }
    };
    vector<Node> tree;
    Trie(int sz=1) {
        tree.reserve(sz);
        tree.emplace_back();
    }
    void insert(string &s) {
        int cur=0;
        tree[cur].pref++;
        for(auto it : s) {
            int ch=it-'a';
            if(tree[cur].next[ch]==-1) {
                tree[cur].next[ch]=(int)tree.size();
                tree.emplace_back();
            }
            cur=tree[cur].next[ch];
            tree[cur].pref++;
        }
        tree[cur].end++;
    }
    int count(string &s) {
        int cur=0;
        for(auto it : s) {
            int ch=it-'a';
            if(tree[cur].next[ch]==-1) return 0;
            cur=tree[cur].next[ch];
        }
        return tree[cur].end;
    }
    int prefixnode(string &s) {
        int cur=0;
        for(auto it : s) {
            int ch=it-'a';
            if(tree[cur].next[ch]==-1) return -1;
            cur=tree[cur].next[ch];
        }
        return cur;
    }
    void erase(string &s) {
        if(count(s)==0) return;
        int cur=0;
        tree[cur].pref--;
        for(auto it : s) {
            int ch=it-'a';
            cur=tree[cur].next[ch];
            tree[cur].pref--;
        }
        tree[cur].end--;
    }
};
\end{lstlisting}
\subsection*{Z-Function}
{ \footnotesize
\textbf{Description:} \texttt{z[i]} is the length of the longest common prefix between string $s$ and the suffix starting at $i$. \\
\textbf{Time:} $\mathcal{O}(N)$.
}
\begin{lstlisting}
vector<int> z_function(const string& s) {
    int n = s.length();
    vector<int> z(n);
    for (int i = 1, l = 0, r = 0; i < n; ++i) {
        if (i <= r)
            z[i] = min(r - i + 1, z[i - l]);
        while (i + z[i] < n && s[z[i]] == s[i + z[i]])
            ++z[i];
        if (i + z[i] - 1 > r)
            l = i, r = i + z[i] - 1;
    }
    return z;
}
\end{lstlisting}

\subsection*{KMP}
{ \footnotesize
\textbf{Description:} \texttt{prefix\_function} computes $\pi[i]$, the length of the longest proper prefix of $s[0\dots i]$ that is also a suffix of $s[0\dots i]$. \texttt{kmp\_search} finds all occurrences of pattern. \\
\textbf{Time:} $\mathcal{O}(N)$.
}
\begin{lstlisting}
vector<int> compute_pi(const string &p) {
    int m = p.size();
    vector<int> pi(m);
    for (int i = 1, j = 0; i < m; i++) {
        while (j > 0 && p[i] != p[j])
            j = pi[j - 1];
        if (p[i] == p[j])
            j++;
        pi[i] = j;
    }
    return pi;
}
vector<int> kmp(const string &t, const string &p) {
    int n = t.size();
    int m = p.size();
    vector<int> matches;
    vector<int> pi = compute_pi(p);
    for (int i = 0, j = 0; i < n; i++) {
        while (j > 0 && t[i] != p[j])
            j = pi[j - 1];
        if (t[i] == p[j])
            j++;
        if (j == m) {
            matches.push_back(i - m + 1);
            j = pi[m - 1]; 
        }
    }
    return matches;
}
\end{lstlisting}

\subsection*{Suffix Array}
{ \footnotesize
\textbf{Description:} Constructs Suffix Array using Prefix Doubling \& Radix Sort. Uses Kasai's Algorithm for LCP.\\
\textbf{Variables:} 
\begin{itemize}
    \setlength\itemsep{0em}
    \item \texttt{sa[i]}: The starting index of the $i$-th lexicographically smallest suffix.
    \item \texttt{lcp[i]}: Longest Common Prefix between suffix \texttt{sa[i]} and \texttt{sa[i-1]}.
\end{itemize}
\textbf{Time:} Build $\mathcal{O}(N \log N)$, LCP $\mathcal{O}(N)$. \\
\textbf{Note:} Appends \texttt{char(0)} automatically. \texttt{sa[0]} is the sentinel. Valid indices $1 \dots N$.
}
\begin{lstlisting}
struct SuffixArray {
    string s;
    int n;
    vector<int> sa;
    vector<int> lcp;

    SuffixArray(string _s):s(_s),n(_s.length()){
        s+=char(0); 
        n++; 
        constructSA();
        constructLCP();
    }

    void constructSA() {
        const int ALPHABET = 256; 
        int m = max(n, ALPHABET);
        sa.resize(n+5);
        vector<int> rank(n+5), new_rank(n+5);
        vector<int> cnt(m+5,0);
        
        for (int i = 0; i < n; i++){
            rank[i] = (unsigned char)s[i];
            cnt[rank[i]]++;
        }
        for (int i = 1; i < m; i++) cnt[i] += cnt[i - 1];
        for (int i = n - 1; i >= 0; i--) sa[--cnt[rank[i]]] = i;

        vector<int> p(n+5);
        for (int k = 1; k < n; k <<= 1) {
            int cur = 0;
            
            for (int i = n - k; i < n; i++) p[cur++] = i;
            for (int i = 0; i < n; i++) {
                if (sa[i] >= k) p[cur++] = sa[i] - k;
            }
            fill(cnt.begin(), cnt.begin() + m, 0);
            for (int i = 0; i < n; i++) cnt[rank[p[i]]]++;
            for (int i = 1; i < m; i++) cnt[i] += cnt[i - 1];
            for (int i = n - 1; i >= 0; i--) sa[--cnt[rank[p[i]]]] = p[i];

            new_rank[sa[0]] = 0;
            int classes = 1;
            for (int i = 1; i < n; i++) {
                bool first_half_same = rank[sa[i]] == rank[sa[i - 1]];
                bool second_half_same = true;
                
                if (sa[i] + k < n && sa[i - 1] + k < n) {
                    second_half_same = (rank[sa[i] + k] == rank[sa[i - 1] + k]);
                } else {
                    second_half_same = (sa[i] + k >= n && sa[i - 1] + k >= n);
                }

                if (!first_half_same || !second_half_same) classes++;
                new_rank[sa[i]] = classes - 1;
            }
            
            rank = new_rank;
            m = classes;
            if (m == n) break;
        }
    }
    void constructLCP() {
        lcp.assign(n+5, 0);
        vector<int> rank(n+5);
        for (int i = 0; i < n; i++) rank[sa[i]]=i;
        int k = 0;
        for (int i = 0; i < n; i++) {
            if (rank[i] == 0) {
                k = 0;
                continue;
            }
            int j = sa[rank[i] - 1];
            while (i + k < n && j + k < n && (unsigned char)s[i + k] == (unsigned char)s[j + k]) k++;
            lcp[rank[i]] = k;
            if (k > 0) k--;
        }
    }
};
\end{lstlisting}

\subsection*{Manacher}
{ \footnotesize
\textbf{Description:} Computes maximal palindrome lengths. \texttt{d1[i]}: max \textbf{odd} palindrome centered at \texttt{i} has radius \texttt{d1[i]-1}. \texttt{d2[i]}: max \textbf{even} palindrome centered at \texttt{i} has radius \texttt{d2[i]-1}. \\
\textbf{Time:} $\mathcal{O}(N)$.
}
\begin{lstlisting}
vector<int> manacher(string s) {
    string t = "^#";
    for (char c : s) {
        t += c;
        t += "#";
    }
    t += "$";
    int n = t.size();
    vector<int> p(n,0);
    int l = 1, r = 1; 

    for (int i = 1; i < n - 1; i++) {
        int i_mirror = l + (r - i);
        if (r > i) {
            p[i] = min(r - i, p[i_mirror]);
        }
        while (t[i + 1 + p[i]] == t[i - 1 - p[i]]) {
            p[i]++;
        }
        if (i + p[i] > r) {
            l = i - p[i];
            r = i + p[i];
        }
    }
    return p;
}

\end{lstlisting}
% ================= DATA STRUCTURE =================
\section*{Data Structure}

\subsection*{Sparse Table}
{ \footnotesize
\textbf{Description:} Static Range Minimum Query (RMQ). \texttt{query} is idempotent ($\mathcal{O}(1)$), \texttt{query1} is cascading for non-idempotent functions ($\mathcal{O}(\log N)$). \\
\textbf{Time:} Build $\mathcal{O}(N \log N)$, Query $\mathcal{O}(1)$.
}
\begin{lstlisting}
template <typename T>
class SparseTable {
    public:
    vector<vector<T> > st;
    T op(T a,T b) {
        return max(a,b);
    }
    SparseTable(int n,vector<T> &vec) {
        st.resize(n+2,vector<T> (__lg(n)+2));
        for(int i=1;i<=n;i++) {
            st[i][0]=vec[i];
        }
        int k=__lg(n)+1;
        for(int j=1;j<=k;j++) {
            for(int i=1;i+(1<<j)<=n+1;i++) {
                st[i][j]=op(st[i][j-1],st[i+(1<<(j-1))][j-1]); 
            }
        }
    }
    T query(int l,int r) {
        int j=__lg(r-l+1);
        return op(st[l][j],st[r-(1<<j)+1][j]);
    }
    T query1(int l,int r) {
        int ans=0;
        for(int j=__lg(r-l+1);j>=0;j--) {
            if((1<<j)<=(r-l+1)) {
                ans=op(ans,st[l][j]);
                l+=(1<<j);
            }
        }
        return ans;
    }
};
\end{lstlisting}

\subsection*{BIT 1D}
{ \footnotesize
\textbf{Description:} Point update, Prefix sum. \texttt{lower\_bound} finds smallest index $i$ such that $\text{sum}(1\dots i) \ge \text{val}$ (requires non-negative values). \\
\textbf{Time:} $\mathcal{O}(\log N)$.
}
\begin{lstlisting}
struct BIT {
  int n;
  vector<long long> bit;
  BIT(int n=0){ init(n); }
  void init(int _n) {
      n = _n;
      bit.assign(n+1, 0);
  }
  // add value `delta` at index i (1-based)
  void add(int i, long long delta) {
      for (; i <= n; i += i & -i) bit[i] += delta;
  }
  // prefix sum [1..i] (1-based)
  long long sumPrefix(int i) {
      long long s = 0;
      for (; i > 0; i -= i & -i) s += bit[i];
      return s;
  }
  // range sum [l..r], 1-based
  long long sumRange(int l, int r) {
      if (r < l) return 0;
      return sumPrefix(r) - sumPrefix(l-1);
  }
  // find smallest index idx such that sumPrefix(idx) >= value (value >= 1)
  // returns n+1 if not found
  int lower_bound(long long value) {
      if (value <= 0) return 1;
      int idx = 0;
      int bitMask = 1;
      while (bitMask << 1 <= n) bitMask <<= 1;
      for (int k = bitMask; k; k >>= 1) {
          int next = idx + k;
          if (next <= n && bit[next] < value) {
              idx = next;
              value -= bit[next];
          }
      }
      return idx + 1;
  }
};
\end{lstlisting}

\subsection*{BIT 2D}
{ \footnotesize
\textbf{Description:} 2D Fenwick Tree for point updates and rectangle sums. 1-based indexing. \\
\textbf{Time:} $\mathcal{O}(\log N \log M)$.
}
\begin{lstlisting}
struct BIT2D {
  int n, m;
  vector<vector<long long>> bit;
  BIT2D(int _n=0, int _m=0){ init(_n,_m); }
  void init(int _n, int _m){
      n = _n; m = _m;
      bit.assign(n+1, vector<long long>(m+1, 0));
  }
  // point add at (x,y) (1-based)
  void add(int x, int y, long long delta){
      for (int i = x; i <= n; i += i & -i)
          for (int j = y; j <= m; j += j & -j)
              bit[i][j] += delta;
  }
  // prefix sum (1..x, 1..y)
  long long sumPrefix(int x, int y){
      long long res = 0;
      for (int i = x; i > 0; i -= i & -i)
          for (int j = y; j > 0; j -= j & -j)
              res += bit[i][j];
      return res;
  }
  // rectangle sum (x1,y1) .. (x2,y2), inclusive, 1-based
  long long range_sum(int x1, int y1, int x2, int y2){
      if (x2 < x1 || y2 < y1) return 0;
      return sumPrefix(x2, y2) - sumPrefix(x1-1, y2)
              - sumPrefix(x2, y1-1) + sumPrefix(x1-1, y1-1);
  }
};
\end{lstlisting}

\subsection*{MO's Algorithm (Hilbert)}
{ \footnotesize
\textbf{Description:} Offline range query processing using Hilbert Curve order to improve cache locality and reduce movement. Significantly faster than standard block sorting. \\
\textbf{Time:} $\mathcal{O}(N \sqrt{Q})$.
}
\begin{lstlisting}
class dat {
public:
  int l, r, id;
  dat() {};
  dat(int l, int r, int id) {
    this->l = l;
    this->r = r;
    this->id = id;
  }
};
void solve() {
  int n;
  cin >> n;
  vector<int> vec(n);
  for (int i = 0; i < n; i++) {
    cin >> vec[i];
  }
  int dis = 0;
  vector<int> freq(1e6 + 5, 0);
  int block_size = sqrt(n);
  int q;
  cin >> q;
  vector<dat> query(q);
  for (int i = 0; i < q; i++) {
    int l, r;
    cin >> l >> r;
    l--, r--;
    query[i] = dat(l, r, i);
  }
  auto hilbertorder = [&](int x, int y) -> long long {
    const int LOG = 21;
    long long d = 0;
    for (int s = 1 << (LOG - 1); s; s >>= 1) {
      bool rx = x & s, ry = y & s;
      d = (d << 2) | (rx * 3 ^ static_cast<int>(ry));
      if (!ry) {
        if (rx) {
          x = (1 << LOG) - 1 - x;
          y = (1 << LOG) - 1 - y;
        }
        swap(x, y);
      }
    }
    return d;
  };
  vector<pair<long long, int>> order(q);
  for (int i = 0; i < q; i++) {
    order[i] = {hilbertorder(query[i].l, query[i].r), i};
  }
  sort(order.begin(), order.end());
  vector<dat> sorted;
  sorted.reserve(q);
  for (auto [_, idx] : order)
    sorted.push_back(query[idx]);
  query.swap(sorted);
  vector<int> ans(q);
  auto add = [&](int ind) {
    freq[vec[ind]]++;
    if (freq[vec[ind]] == 1)
      dis++;
  };
  auto remove = [&](int ind) {
    freq[vec[ind]]--;
    if (freq[vec[ind]] == 0)
      dis--;
  };
  int L = 0, R = -1;
  for (int i = 0; i < q; i++) {
    int l = query[i].l;
    int r = query[i].r;
    int id = query[i].id;
    while (L > l) {
      add(--L);
    }
    while (R < r) {
      add(++R);
    }
    while (L < l) {
      remove(L++);
    }
    while (R > r) {
      remove(R--);
    }
    ans[id] = dis;
  }

  for (int i = 0; i < q; i++) {
    cout << ans[i] << el;
  }
}
\end{lstlisting}

\subsection*{Merge Sort Tree}
\begin{lstlisting}
class node {
public:
    vector<int> v;
    vector<ll> pref;
    node(){};
    node(int x) {
        v.pb(x);
        pref.resize(1,0);
        pref[0]=x;
    }
};
template <typename Node=node>
class SegmentTree {
public:
    vector<Node> st;
    Node op(Node &a,Node &b) {
        node cur;
        int sz=a.v.size()+b.v.size();
        cur.v.resize(sz,0);
        cur.pref.resize(sz,0);
        merge(all(a.v),all(b.v),cur.v.begin());
        cur.pref[0]=cur.v[0];
        for(int i=1;i<sz;i++) {
            cur.pref[i]=cur.v[i]+cur.pref[i-1];
        }
        return cur;
    }
    SegmentTree(vector<int> &vec, int n) {
        st.resize(4*n,Node());
        function<void(int, int, int)> build = [&](int id, int start, int end) {
            if (start == end) {
                st[id]=Node(vec[start]);
                return;
            }
            int mid = (start + end) / 2;
            build(2 * id, start, mid);
            build(2 * id + 1, mid + 1, end);
            st[id] = op(st[2*id],st[2*id+1]); 
        };
        build(1, 1, n);
    }
    ll query(int id, int start, int end,ll l,ll r,ll k) {
        if (start > r or end < l)
            return 0;
        if (start >= l and end <= r) {
            auto lo=upper_bound(all(st[id].v),k);
            int ind=lo-st[id].v.begin();
            if(ind==0) return 0;
            return st[id].pref[ind-1];
        };
        ll mid = start + (end - start) / 2;
        ll left = query(2 * id, start, mid, l, r,k);
        ll right = query(2 * id + 1, mid + 1, end, l, r,k);
        return (left+right); 
    }
};
\end{lstlisting}

\subsection*{XOR Trie}
{ \footnotesize
\textbf{Description:} Binary Trie for integers. Supports finding pair with maximum XOR. \\
\textbf{Time:} $\mathcal{O}(\log (\max A))$.
}
\begin{lstlisting}
class TrieNode {
public:
  TrieNode *left;
  TrieNode *right;
  int cnt = 0;
  TrieNode() {
    left = NULL;
    right = NULL;
    cnt = 0;
  }
};
class Trie {
  TrieNode *root;

public:
  Trie() {
    root = new TrieNode();
  }
  void insert(int n) {
    TrieNode *curr = root;
    for (int i = 31; i >= 0; i--) {
      int bit = (1 & (n >> i));
      if (bit == 0) {
        if (curr->left == NULL) {
          curr->left = new TrieNode();
        }
        curr = curr->left;
        curr->cnt++;
      }
      else {
        if (curr->right == NULL) {
          curr->right = new TrieNode();
        }
        curr = curr->right;
        curr->cnt++;
      }
    }
  }
  void remove(int n) {
    TrieNode *curr = root;
    for (int i = 31; i >= 0; i--) {
      if (curr == NULL)
        break;
      int bit = (n >> i) & 1;
      if (bit == 0) {
        curr = curr->left;
        curr->cnt--;
      }
      else {
        curr = curr->right;
        curr->cnt--;
      }
    }
  }
  int max_xor_pair(int n) {
    TrieNode *curr = root;
    int ans = 0;
    for (int i = 31; i >= 0; i--) {
      if (curr == NULL) {
        break;
      }
      int bit = (1 & (n >> i));
      if (bit == 0) {
        if (curr->right != NULL and curr->right->cnt > 0) {
          ans += (1 << i);
          curr = curr->right;
        }
        else
          curr = curr->left;
      }
      else {
        if (curr->left != NULL and curr->left->cnt > 0) {
          ans += (1 << i);
          curr = curr->left;
        }
        else
          curr = curr->right;
      }
    }
    return ans;
  }
};
\end{lstlisting}

\subsection*{LAZY SegTree}
{ \footnotesize
\textbf{Description:} Standard Lazy Propagation for range updates. \\
\textbf{Time:} $\mathcal{O}(\log N)$.
}
\begin{lstlisting}
struct ST{
    int n;            
    vector<int> t,lazy,arr; 
    void init(int n) {
        this->n=n;
        t.assign(3*n+5,0); 
        lazy.assign(3*n+5,0);
        arr.assign(n+5,0);
    }
    inline void push(int node,int l,int r){
        if(!lazy[node]) return;
        t[node]+=lazy[node]*(r-l+1); // check here
        if(l!=r){
            lazy[node*2]+=lazy[node];
            lazy[node*2+1]+=lazy[node];
        }
        lazy[node]=0;
    }
    inline void here(int node){
        t[node]=t[node*2]+t[node*2+1];  // check here
    }
    void build(int node,int l,int r){
        lazy[node]=0;
        if(l==r){
            t[node]=arr[l];
            return;
        }
        ll mid=(l+r)>>1;
        build(node*2,l,mid);
        build(node*2+1,mid+1,r);
        here(node);
    }
    void upd(int node,int l,int r,int i,int j,int value){
        push(node,l,r);
        if(l>j || r<i) return;
        if(i<=l && r<=j){
            lazy[node]+=value; // check here
            push(node,l,r);
            return;
        }
        ll mid=(l+r)>>1;
        upd(node*2,l,mid,i,j,value);
        upd(node*2+1,mid+1,r,i,j,value);
        here(node);
    }
    ll query(int node,int l,int r,int i,int j){
        push(node,l,r);
        if(l>j || r<i) return 0;           /// check here
        if(i<=l && r<=j) return t[node];
        ll mid=(l+r)>>1;
        return query(node*2,l,mid,i,j)+query(node*2+1,mid+1,r,i,j); // check here
    }
}t;
\end{lstlisting}

\subsection*{PST (Persistent SegTree)}
{ \footnotesize
\textbf{Description:} Persistent segment tree. \texttt{add\_copy} branches off a version. \\
\textbf{Time:} $\mathcal{O}(\log N)$ query/update. \textbf{Space:} $\mathcal{O}(Q \log N)$.
}
\begin{lstlisting}
class PST{
    private:
        struct node{
            ll sum=0;
            int lc=0,rc=0; // left child right child 
        };
    const int n;
    vector<node> tree;
    int timer=1;
    node join(int lc,int rc){
        return node{tree[lc].sum+tree[rc].sum,lc,rc}; // check here
    }
    int build_(int l,int r,const vector<int> &arr){
        int id=timer++;
        if(l==r){
            tree[id]={arr[l],0,0}; // check here
            return id;
        }
        int mid=(l+r)>>1;
        tree[id]=join(build_(l,mid,arr),build_(mid+1,r,arr));
        return id;
    }
    int upd_(int v,int l,int r,int pos,int val){
        int id=timer++;
        if(l==r){
            tree[id]={val,0,0}; // check here
            return id;
        }
        int mid=(l+r)>>1;
        if(pos<=mid) tree[id]=join(upd_(tree[v].lc,l,mid,pos,val),tree[v].rc);
        else tree[id]=join(tree[v].lc,upd_(tree[v].rc,mid+1,r,pos,val));
        return id;
    }
    ll query_(int v,int l,int r,int i,int j){
        if(l>j || r<i) return 0LL;           /// check here
        if(i<=l && r<=j) return tree[v].sum;
        int mid=(l+r)>>1;
        return query_(tree[v].lc,l,mid,i,j)+
               query_(tree[v].rc,mid+1,r,i,j); 
    }
    public:
    PST(int n,int mx_nodes) : n(n),tree(mx_nodes) {}
    int build(const vector<int> &arr) { return build_(1,n,arr); }
    int upd(int root,int pos,int val) { return upd_(root,1,n,pos,val); }
    ll query(int root,int l,int r) { return query_(root,1,n,l,r); }
    int add_copy(int root){
        tree[timer]=tree[root];
        return timer++;
    } 
};
int32_t main()
{
    const int mx_nodes=2*n+q*(2+__lg(n));
    PST t(n,mx_nodes);
    vector<int> roots = {t.build(a)};
    while(q--){
        int type,k; cin>>type>>k;
        k--;
        if(type==1){
            int pos,val; cin>>pos>>val;
            roots[k]=t.upd(roots[k],pos,val);
        }else if(type==2){
            int a,b; cin>>a>>b;
            cout<<t.query(roots[k],a,b)<<endl;
        }else{
            roots.PB(t.add_copy(roots[k]));
        }
    }
}
\end{lstlisting}

\subsection*{Dynamic SegTree}
{ \footnotesize
\textbf{Description:} Segment tree with sparse coordinates ($N \approx 10^9$). Nodes created on demand. \\
\textbf{Time:} $\mathcal{O}(\log (\text{Range}))$.
}
\begin{lstlisting}
class SparseSegTree {
private:
    struct node {
        ll freq=0;
        ll lazy=0;
        int left=0;
        int right=0;
        bool lazy_flag=false;
    };
    vector<node> tree;
    const ll n;
    int timer=1;
    // int comb(int a,int b) { return a+b; }
    void apply(int cur,ll l,ll r,ll val) { // check here
        tree[cur].lazy=val;
        tree[cur].lazy_flag=true;
        tree[cur].freq=(r-l+1)*val;
    }
    void push_down(int cur,ll l,ll r){
        if(!tree[cur].left){
            tree[cur].left= ++timer;
            tree.PB(node());
        }
        if(!tree[cur].right){
            tree[cur].right= ++timer;
            tree.PB(node());
        }
        if(!tree[cur].lazy_flag) return;
        ll mid=(l+r)>>1;
        apply(tree[cur].left,l,mid,tree[cur].lazy);
        apply(tree[cur].right,mid+1,r,tree[cur].lazy);
        tree[cur].lazy_flag=false;
        tree[cur].lazy=0;
    }
    void upd(int cur,ll l,ll r,ll ql,ll qr,ll val) {
        if(qr<l || ql>r) return;
        if(ql<=l && r<=qr) apply(cur,l,r,val);
        else {
            push_down(cur,l,r);
            ll mid=(l+r)>>1;
            upd(tree[cur].left,l,mid,ql,qr,val);
            upd(tree[cur].right,mid+1,r,ql,qr,val);
            tree[cur].freq=
                tree[tree[cur].left].freq + tree[tree[cur].right].freq; // check here
        }
    }
    ll query(int cur,ll l,ll r,ll ql,ll qr) {
        if(qr<l || ql>r || !cur) return 0;
        if(ql<=l && r<=qr) return tree[cur].freq;
        push_down(cur,l,r);
        ll mid=(l+r)>>1;
        return query(tree[cur].left,l,mid,ql,qr) + 
               query(tree[cur].right,mid+1,r,ql,qr); // check here
    }
public:
    SparseSegTree(ll n,int q=0) : n(n) {
        if(q>0) { tree.reserve(2*q*__lg(n)); }
        tree.PB(node()); tree.PB(node());
    }
    void upd(ll ql,ll qr,ll val) { upd(1,1,n,ql,qr,val); }
    int query(ll ql,ll qr) {return query(1,1,n,ql,qr); }
};
int32_t main(){
    const int range_size=1e9;
    SparseSegTree st(range_size+1,q); // pass n+q if there is n given
}
\end{lstlisting}

\subsection*{Wavelet Tree}
{ \footnotesize
\textbf{Description:} Partitions array based on values. \texttt{kth}: $k$-th smallest in range. \texttt{LTE}: count values $\le k$. \texttt{count}: range value freq. \\
\textbf{Time:} $\mathcal{O}(\log (\max A))$ per query.
}
\begin{lstlisting}
struct wavelet_tree
{
  int lo, hi;
  wavelet_tree *l, *r;
  vi b;
  wavelet_tree(int *from, int *to, int x, int y)
  {
    lo = x, hi = y;
    if (lo == hi or from >= to)
      return;
    int mid = (lo + hi) / 2;
    auto f = [mid](int x)
    {
      return x <= mid;
    };
    b.reserve(to - from + 1);
    b.pb(0);
    for (auto it = from; it != to; it++)
      b.pb(b.back() + f(*it));
    // see how lambda function is used here
    auto pivot = stable_partition(from, to, f);
    l = new wavelet_tree(from, pivot, lo, mid);
    r = new wavelet_tree(pivot, to, mid + 1, hi);
  }
  // kth smallest element in [l, r]
  int kth(int l, int r, int k)
  {
    if (l > r)
      return 0;
    if (lo == hi)
      return lo;
    int inLeft = b[r] - b[l - 1];
    int lb = b[l - 1]; // amt of nos in first (l-1) nos that go in left
    int rb = b[r];     // amt of nos in first (r) nos that go in left
    if (k <= inLeft)
      return this->l->kth(lb + 1, rb, k);
    return this->r->kth(l - lb, r - rb, k - inLeft);
  }
  // count of nos in [l, r] Less than or equal to k
  int LTE(int l, int r, int k)
  {
    if (l > r or k < lo)
      return 0;
    if (hi <= k)
      return r - l + 1;
    int lb = b[l - 1], rb = b[r];
    return this->l->LTE(lb + 1, rb, k) + this->r->LTE(l - lb, r - rb, k);
  }
  int count(int l, int r, int k)
  {
    if (l > r or k < lo or k > hi)
      return 0;
    if (lo == hi)
      return r - l + 1;
    int lb = b[l - 1], rb = b[r], mid = (lo + hi) / 2;
    if (k <= mid)
      return this->l->count(lb + 1, rb, k);
    return this->r->count(l - lb, r - rb, k);
  }
  ~wavelet_tree()
  {
    delete l;
    delete r;
  }
};
int main()
{
  wavelet_tree T(a + 1, a + n + 1, 1, MAX);
}
\end{lstlisting}

\subsection*{SEGTree Beats (main)}
{ \footnotesize
\textbf{Description:} "Jiry Match" Tree. Supports Range Chmin ($a_i = \min(a_i, x)$), Chmax, Add, Set, Mod, Divide, Negative. Handles history/break conditions. \\
\textbf{Time:} Amortized $\mathcal{O}((N+Q)\log N)$.
}
\begin{lstlisting}

const ll INF=1e18;
const ll NINF=-1e18;
struct STBeats {
private:
    struct node{
        ll max1;                // max value
        ll max2;                // second max value
        int max_cnt;            // cnt of the largest value
        ll min1;                // min value
        ll min2;                // second min value
        int min_cnt;            // cnt of the smallest value
        ll sum;                 // sum of the range 
        int len;                // length of the range 
        ll lazy_add;            // lazy teg
        ll lazy_set;
        bool lazy_neg;
        node() : max1(NINF),max2(NINF),max_cnt(0),
                min1(INF),min2(INF),min_cnt(0),sum(0),len(0),
                lazy_add(0),lazy_set(INF),lazy_neg(false) {}
    };

    int n;
    vector<node> tree;
    inline node merge(const node& left, const node& right) {       // O(1)
        node res;
        res.sum=left.sum+right.sum;
        res.len=left.len+right.len;
        res.lazy_add=0;
        res.lazy_set=INF;
        res.lazy_neg=false;
        if(left.max1>right.max1) { // merging max data for chmin
            res.max1=left.max1;
            res.max2=max(left.max2,right.max1);
            res.max_cnt=left.max_cnt;
        }else if(left.max1<right.max1) {
            res.max1=right.max1;
            res.max2=max(left.max1,right.max2);
            res.max_cnt=right.max_cnt;
        }else if(left.max1==right.max1) {
            res.max1=left.max1;
            res.max2=max(left.max2,right.max2);
            res.max_cnt=left.max_cnt+right.max_cnt;
        }

        if(left.min1<right.min1) {  // margin min data for chmax     
            res.min1=left.min1;
            res.min2=min(left.min2,right.min1);
            res.min_cnt=left.min_cnt;
        }else if(left.min1>right.min1) {
            res.min1=right.min1;
            res.min2=min(left.min1,right.min2);
            res.min_cnt=right.min_cnt;
        }else if(left.min1==right.min1) {
            res.min1=left.min1;
            res.min2=min(left.min2,right.min2);
            res.min_cnt=left.min_cnt+right.min_cnt;
        }
        return res;
    }
    inline void apply_negative(int v) {
        swap(tree[v].max1,tree[v].min1);
        swap(tree[v].max2,tree[v].min2);
        swap(tree[v].max_cnt,tree[v].min_cnt);
        tree[v].max1*=-1;
        if(tree[v].max2!=NINF) tree[v].max2*=-1;
        tree[v].min1*=-1;
        if(tree[v].min2!=INF) tree[v].min2*=-1;
        tree[v].sum*=-1;
        if(tree[v].lazy_set!=INF) tree[v].lazy_set*=-1;
        else tree[v].lazy_add*=-1;
        tree[v].lazy_neg^=1;
    }
    inline void apply_add(int v,ll x) {             // O(1)
        if(!x) return;
        tree[v].sum+=tree[v].len*x;
        tree[v].max1+=x;
        if(tree[v].max2!=NINF) tree[v].max2+=x;
        tree[v].min1+=x;
        if(tree[v].min2!=INF) tree[v].min2+=x;

        if(tree[v].lazy_set!=INF) tree[v].lazy_set+=x;
        else tree[v].lazy_add+=x;
    }

    inline void apply_set(int v,ll x) {
        tree[v].max1=x;
        tree[v].max2=NINF;
        tree[v].max_cnt=tree[v].len;
        tree[v].min1=x;
        tree[v].min2=INF;
        tree[v].min_cnt=tree[v].len;
        tree[v].sum=tree[v].len*x;
        tree[v].lazy_add=0;
        tree[v].lazy_set=x;
        tree[v].lazy_neg=false;
    }
    inline void apply_chmin(int v,ll x) {          // O(1)
        if(x>=tree[v].max1) return;
        tree[v].sum-=tree[v].max_cnt*(tree[v].max1-x);
        if(tree[v].min1==tree[v].max1) tree[v].min1=x;
        if(tree[v].min2==tree[v].max1) tree[v].min2=x;
        tree[v].max1=x;

        if(tree[v].lazy_set !=INF) 
            tree[v].lazy_set=min(tree[v].lazy_set,x);
    }

    inline void apply_chmax(int v,ll x) {          // O(1)
        if(x<=tree[v].min1) return;
        tree[v].sum+=tree[v].min_cnt*(x-tree[v].min1);
        if(tree[v].max1==tree[v].min1) tree[v].max1=x;
        if(tree[v].max2==tree[v].min1) tree[v].max2=x;
        tree[v].min1=x;

        if(tree[v].lazy_set !=INF) 
            tree[v].lazy_set=max(tree[v].lazy_set,x);
    }

    void push_lazy(int v,int tl,int tr) {             // O(1)
        if(tl==tr) return;
        if(tree[v].lazy_set!=INF) {
            apply_set(2*v,tree[v].lazy_set);
            apply_set(2*v+1,tree[v].lazy_set);
            tree[v].lazy_set=INF;
            return;
        }
        if(tree[v].lazy_neg) {
            apply_negative(2*v);
            apply_negative(2*v+1);
            tree[v].lazy_neg=false;
        }
        if(tree[v].lazy_add!=0) {       // for lazy add
            apply_add(2*v,tree[v].lazy_add);
            apply_add(2*v+1,tree[v].lazy_add);
            tree[v].lazy_add=0;
        }
    }
    void push_beats(int v,int tl,int tr) {
        if(tl==tr) return;
        apply_chmin(2*v,tree[v].max1);
        apply_chmin(2*v+1,tree[v].max1);
        apply_chmax(2*v,tree[v].min1);
        apply_chmax(2*v+1,tree[v].min1);
    }

    void build_(int v,int tl,int tr,const vector<ll>& a) {       // O(n)
        if(tl==tr){
            tree[v].len=1;
            tree[v].sum=a[tl];
            tree[v].max1=a[tl];
            tree[v].max_cnt=1;
            tree[v].max2=NINF;
            tree[v].min1=a[tl];
            tree[v].min_cnt=1;
            tree[v].min2=INF;
            tree[v].lazy_add=0;
            tree[v].lazy_set=INF;
            tree[v].lazy_neg=false;
        }else{
            int mid=(tl+tr)>>1;
            build_(2*v,tl,mid,a);
            build_(2*v+1,mid+1,tr,a);
            tree[v]=merge(tree[2*v],tree[2*v+1]);
        }
    }

    void upd_min_(int v,int tl,int tr,int ql,int qr,ll x) {   // O(log^2 n)
        push_lazy(v,tl,tr);
        if(tree[v].max1<=x || qr<tl || tr<ql) return;
        if(ql<=tl && tr<=qr && tree[v].max2<x) {
            apply_chmin(v,x);
            return;
        }
        push_beats(v,tl,tr);
        int mid=(tl+tr)>>1;
        upd_min_(2*v,tl,mid,ql,qr,x);
        upd_min_(2*v+1,mid+1,tr,ql,qr,x);
        tree[v]=merge(tree[2*v],tree[2*v+1]);
    }

    void upd_max_(int v,int tl,int tr,int ql,int qr,ll x) {   // O(log^2 n)
        push_lazy(v,tl,tr);
        if(tree[v].min1>=x || qr<tl || tr<ql) return;
        if(ql<=tl && tr<=qr && tree[v].min2>x) {
            apply_chmax(v,x);
            return;
        }
        push_beats(v,tl,tr);
        int mid=(tl+tr)>>1;
        upd_max_(2*v,tl,mid,ql,qr,x);
        upd_max_(2*v+1,mid+1,tr,ql,qr,x);
        tree[v]=merge(tree[2*v],tree[2*v+1]);
    }

    void upd_add_(int v,int tl,int tr,int ql,int qr,ll x) {   // O(log n)
        if(qr<tl || tr<ql) return;
        if(ql<=tl && tr<=qr) {
            apply_add(v,x);
            return;
        }
        push_lazy(v,tl,tr);
        push_beats(v,tl,tr);
        int mid=(tl+tr)>>1;
        upd_add_(2*v,tl,mid,ql,qr,x);
        upd_add_(2*v+1,mid+1,tr,ql,qr,x);
        tree[v]=merge(tree[2*v],tree[2*v+1]);
    }

    void upd_set_(int v,int tl,int tr,int ql,int qr,ll x) {   // O(log n) range set
        if(qr<tl || tr<ql) return;
        if(ql<=tl && tr<=qr) {
            apply_set(v,x);
            return;
        }
        push_lazy(v,tl,tr);
        push_beats(v,tl,tr);
        int mid=(tl+tr)>>1;
        upd_set_(2*v,tl,mid,ql,qr,x);
        upd_set_(2*v+1,mid+1,tr,ql,qr,x);
        tree[v]=merge(tree[2*v],tree[2*v+1]);
    }

    void upd_mod_(int v,int tl,int tr,int ql,int qr,ll x) {   // O(log^2 n)
        push_lazy(v,tl,tr);
        if(tree[v].max1<x || qr<tl || tr<ql) return;
        if(tl==tr) {
            apply_set(v,tree[v].sum%x);
            return;
        }
        push_beats(v,tl,tr);
        int mid=(tl+tr)>>1;
        upd_mod_(2*v,tl,mid,ql,qr,x);
        upd_mod_(2*v+1,mid+1,tr,ql,qr,x);
        tree[v]=merge(tree[2*v],tree[2*v+1]);
    }
    ll floor_div(ll a,ll b) {
        if(b<0) { a=-a,b=-b; }
        ll d=a/b;
        ll r=a%b;
        if(r<0) return d-1;
        return d;
    }
    void upd_negative_(int v,int tl,int tr,int ql,int qr) {
        if(qr<tl || tr<ql) return;
        if(ql<=tl && tr<=qr) {
            apply_negative(v);
            return;
        }
        push_lazy(v,tl,tr);
        push_beats(v,tl,tr);
        int mid=(tl+tr)>>1;
        upd_negative_(2*v,tl,mid,ql,qr);
        upd_negative_(2*v+1,mid+1,tr,ql,qr);
        tree[v]=merge(tree[2*v],tree[2*v+1]);
    }
    void upd_divide_(int v,int tl,int tr,int ql,int qr,ll x) {    // O(log^2 n)
        if(x==1) return;
        if(x==-1){
            upd_negative_(v,tl,tr,ql,qr);
            return;
        }
        if(qr<tl || tr<ql || !x) return;
        push_lazy(v,tl,tr);
        ll new_min=floor_div(tree[v].min1,x);
        ll new_max=floor_div(tree[v].max1,x);
        if(ql<=tl && tr<=qr && new_min==new_max){
            apply_set(v,new_min);
            return;
        }
        if(tl==tr) {
            ll val=floor_div(tree[v].sum,x);
            apply_set(v,val);
            return;
        }
        push_beats(v,tl,tr);
        int mid=(tl+tr)>>1;
        upd_divide_(2*v,tl,mid,ql,qr,x);
        upd_divide_(2*v+1,mid+1,tr,ql,qr,x);
        tree[v]=merge(tree[2*v],tree[2*v+1]);
    }

    ll query_sum_(int v,int tl,int tr,int ql,int qr) { // O(log n)
        if(qr<tl || tr<ql) return 0;
        if(ql<=tl && tr<=qr) return tree[v].sum;
        push_lazy(v,tl,tr);
        push_beats(v,tl,tr);
        int mid=(tl+tr)>>1;
        return query_sum_(2*v,tl,mid,ql,qr) + query_sum_(2*v+1,mid+1,tr,ql,qr);
    }
    ll query_max_(int v,int tl,int tr,int ql,int qr) { // O(log n)
        if(qr<tl || tr<ql) return NINF;
        if(ql<=tl && tr<=qr) return tree[v].max1;
        push_lazy(v,tl,tr);
        push_beats(v,tl,tr);
        int mid=(tl+tr)>>1;
        return max(query_max_(2*v,tl,mid,ql,qr) , query_max_(2*v+1,mid+1,tr,ql,qr));
    }
    ll query_min_(int v,int tl,int tr,int ql,int qr) { // O(log n)
        if(qr<tl || tr<ql) return INF;
        if(ql<=tl && tr<=qr) return tree[v].min1;
        push_lazy(v,tl,tr);
        push_beats(v,tl,tr);
        int mid=(tl+tr)>>1;
        return min(query_min_(2*v,tl,mid,ql,qr) , query_min_(2*v+1,mid+1,tr,ql,qr));
    }
    
public:
    STBeats(int n_val) : n(n_val) { tree.resize(4*n+4); }
    void build(const vector<ll>& a) { build_(1,1,n,a); }
    void upd_min(int ql,int qr,ll x) { upd_min_(1,1,n,ql,qr,x); }
    void upd_max(int ql,int qr,ll x) { upd_max_(1,1,n,ql,qr,x); }
    void upd_add(int ql,int qr,ll x) { upd_add_(1,1,n,ql,qr,x); }
    void upd_set(int ql,int qr,ll x) { upd_set_(1,1,n,ql,qr,x); }
    void upd_mod(int ql,int qr,ll x) { upd_mod_(1,1,n,ql,qr,x); }
    void upd_divide(int ql,int qr,ll x) { upd_divide_(1,1,n,ql,qr,x); }
    ll query_sum(int ql,int qr) { return query_sum_(1,1,n,ql,qr); }
    ll query_max(int ql,int qr) { return query_max_(1,1,n,ql,qr); }
    ll query_min(int ql,int qr) { return query_min_(1,1,n,ql,qr); }
};
int32_t main(){
    ios_base :: sync_with_stdio(0); cin.tie(0);
    int t=1; 
    // cin>>t;
    while(t--){
      int n; cin>>n;
      int q; cin>>q;
      STBeats t(n);
      vector<ll> v(n+1);
      for(int i=1;i<=n;i++) cin>>v[i];
      t.build(v);
      while(q--){
        int type,l,r; cin>>type>>l>>r;
        if(type==1) {
          ll val; cin>>val;
          t.upd_divide(l,r,val);
        }else if(type==2){
          ll val; cin>>val;
          t.upd_set(l,r,val);
        }else cout<<t.query_sum(l,r)<<endl;
      }
    }
}

/*
 @ the bellow code is dedicated for range and range divide it is more faster divide then main struct 
 cause it is dedicated only for make divide very faster 
*/

struct STBeats_Light {
private:
    struct node {
        ll sum;
        ll min1;
        ll max1;
        ll lazy_add;
        node() : sum(0), min1(INF), max1(NINF), lazy_add(0) {}
    };
    int n;
    vector<node> tree;
    void pull(int v) {
        tree[v].sum = tree[2 * v].sum + tree[2 * v + 1].sum;
        tree[v].min1 = min(tree[2 * v].min1, tree[2 * v + 1].min1);
        tree[v].max1 = max(tree[2 * v].max1, tree[2 * v + 1].max1);
    }
    void apply_add(int v, int tl, int tr, ll x) {
        tree[v].sum += (tr - tl + 1) * x;
        tree[v].min1 += x;
        tree[v].max1 += x;
        tree[v].lazy_add += x;
    }
    void push(int v, int tl, int tr) {
        if (tree[v].lazy_add == 0) return;
        int mid = (tl + tr) >> 1;
        apply_add(2 * v, tl, mid, tree[v].lazy_add);
        apply_add(2 * v + 1, mid + 1, tr, tree[v].lazy_add);
        tree[v].lazy_add = 0;
    }
    void build_(int v, int tl, int tr, const vector<ll>& a) {
        // here write the build function from main STBeats
    }
    void upd_add_(int v, int tl, int tr, int ql, int qr, ll x) {
        if (qr < tl || tr < ql) return;
        if (ql <= tl && tr <= qr) {
            apply_add(v, tl, tr, x);
            return;
        }
        push(v, tl, tr);
        int mid = (tl + tr) >> 1;
        upd_add_(2 * v, tl, mid, ql, qr, x);
        upd_add_(2 * v + 1, mid + 1, tr, ql, qr, x);
        pull(v);
    }
    ll floor_div(ll a, ll b) {
        // here write the floor_div function from main STBeats   
    }
    void upd_divide_(int v, int tl, int tr, int ql, int qr, ll x) {
        if (qr < tl || tr < ql) return;
        if (ql <= tl && tr <= qr) {
            ll new_min = floor_div(tree[v].min1, x);
            ll new_max = floor_div(tree[v].max1, x);
            ll delta_min = new_min - tree[v].min1;
            ll delta_max = new_max - tree[v].max1;
            if (delta_min == delta_max) {
                apply_add(v, tl, tr, delta_min);
                return;
            }
        }
        if (tl == tr) {
            ll new_val = floor_div(tree[v].min1, x);
            tree[v].sum = tree[v].min1 = tree[v].max1 = new_val;
            return;
        }
        push(v, tl, tr);
        int mid = (tl + tr) >> 1;
        upd_divide_(2 * v, tl, mid, ql, qr, x);
        upd_divide_(2 * v + 1, mid + 1, tr, ql, qr, x);
        pull(v);
    }
    ll query_sum_(int v, int tl, int tr, int ql, int qr) {
        // here write the query_sum_ function from main STBeats
    }
    ll query_min_(int v, int tl, int tr, int ql, int qr) {
        // here write the query_min_ function from main STBeats
    }

public:
    STBeats_Light(int n_val) : n(n_val) { tree.resize(4 * n + 4); }
    void build(const vector<ll>& a) { build_(1, 1, n, a); }
    void upd_add(int ql, int qr, ll x) { upd_add_(1, 1, n, ql, qr, x); }
    void upd_divide(int ql, int qr, ll x) { upd_divide_(1, 1, n, ql, qr, x); }
    ll query_sum(int ql, int qr) { return query_sum_(1, 1, n, ql, qr); }
    ll query_min(int ql, int qr) { return query_min_(1, 1, n, ql, qr); }
};
\end{lstlisting}

\subsection*{SEGTree Beats Bit and Gcd}
{ \footnotesize
\textbf{Description:}
\begin{enumerate}[leftmargin=*, nosep]
    \item \textbf{Bitwise:} Range AND/OR/Set using \texttt{tree[v].all\_or} and \texttt{tree[v].all\_and} to detect if update affects range.
    \item \textbf{GCD:} Range GCD + Min/Max/Add.
\end{enumerate}
}
\begin{lstlisting}

const ll INF=1e18;
const ll NINF=-1e18;

struct STBeats_Bit {
private:
    struct node {
        ll sum;
        int len;
        ll all_and;
        ll all_or;
        ll max_val;
        ll min_val;
        ll lazy_set;

        node() : sum(0),len(0),all_and(~0LL),all_or(0LL),
                 max_val(NINF),min_val(INF),lazy_set(INF) {}
    };
    int n;
    vector<node> tree;
    node merge(const node& left,const node& right) {
        node res;
        res.sum=left.sum+right.sum;
        res.len=left.len+right.len;
        res.all_and=left.all_and & right.all_and;
        res.all_or=left.all_or | right.all_or;
        res.max_val=max(left.max_val,right.max_val);
        res.min_val=min(left.min_val,right.min_val);
        res.lazy_set=INF;
        return res;
    }
    void apply_set(int v,ll x) {
        tree[v].sum=tree[v].len*x;
        tree[v].all_and=x;
        tree[v].all_or=x;
        tree[v].max_val=x;
        tree[v].min_val=x;
        tree[v].lazy_set=x;
    }
    void push_down(int v,int tl,int tr) {
        if(tl==tr || tree[v].lazy_set==INF) return;
        apply_set(2*v,tree[v].lazy_set);
        apply_set(2*v+1,tree[v].lazy_set);
        tree[v].lazy_set=INF;
    }
    void build_(int v,int tl,int tr,const vector<ll>& a) {
        if(tl==tr) {
            tree[v].len=1;
            tree[v].sum=a[tl];
            tree[v].all_and=a[tl];
            tree[v].all_or=a[tl];
            tree[v].max_val=a[tl];
            tree[v].min_val=a[tl];
        }else {
            int mid=(tl+tr)>>1;
            build_(2*v,tl,mid,a);
            build_(2*v+1,mid+1,tr,a);
            tree[v]=merge(tree[2*v],tree[2*v+1]);
        }
    }
    void upd_or_(int v,int tl,int tr,int ql,int qr,ll x) {
        push_down(v,tl,tr);
        if(qr<tl || tr<ql) return;
        if((tree[v].all_and & x)==x) return;
        if(tl==tr) {
            apply_set(v,tree[v].sum | x);
            return;
        }
        int mid=(tl+tr)>>1;
        upd_or_(2*v,tl,mid,ql,qr,x);
        upd_or_(2*v+1,mid+1,tr,ql,qr,x);
        tree[v]=merge(tree[2*v],tree[2*v+1]);
    }
    void upd_and_(int v,int tl,int tr,int ql,int qr,ll x) {
        push_down(v,tl,tr);
        if(qr<tl || tr<ql) return;
        if((tree[v].all_or | x)==x) return;
        if(tl==tr) {
            apply_set(v,tree[v].sum & x);
            return;
        }
        int mid=(tl+tr)>>1;
        upd_and_(2*v,tl,mid,ql,qr,x);
        upd_and_(2*v+1,mid+1,tr,ql,qr,x);
        tree[v]=merge(tree[2*v],tree[2*v+1]);
    }
    void upd_set_(int v,int tl,int tr,int ql,int qr,ll x) {
        push_down(v,tl,tr);
        if(qr<tl || tr<ql) return;
        if(ql<=tl && tr<=qr) {
            apply_set(v,x);
            return;
        }
        int mid=(tl+tr)>>1;
        upd_set_(2*v,tl,mid,ql,qr,x);
        upd_set_(2*v+1,mid+1,tr,ql,qr,x);
        tree[v]=merge(tree[2*v],tree[2*v+1]);
    }
    ll query_sum_(int v,int tl,int tr,int ql,int qr) {
        if(qr<tl || tr<ql) return 0;
        push_down(v,tl,tr);
        if(ql<=tl && tr<=qr) return tree[v].sum;
        int mid=(tl+tr)>>1;
        return query_sum_(2*v,tl,mid,ql,qr) +
                query_sum_(2*v+1,mid+1,tr,ql,qr);
    }
    ll query_and_(int v,int tl,int tr,int ql,int qr) {
        if(qr<tl || tr<ql) return ~0LL;
        push_down(v,tl,tr);
        if(ql<=tl && tr<=qr) return tree[v].all_and;
        int mid=(tl+tr)>>1;
        return query_and_(2*v,tl,mid,ql,qr) &
                query_and_(2*v+1,mid+1,tr,ql,qr);
    }
    ll query_or_(int v,int tl,int tr,int ql,int qr) {
        if(qr<tl || tr<ql) return 0LL;
        push_down(v,tl,tr);
        if(ql<=tl && tr<=qr) return tree[v].all_or;
        int mid=(tl+tr)>>1;
        return query_or_(2*v,tl,mid,ql,qr) |
                query_or_(2*v+1,mid+1,tr,ql,qr);
    }
    ll query_max_(int v,int tl,int tr,int ql,int qr) {
        if(qr<tl || tr<ql) return NINF;
        push_down(v,tl,tr);
        if(ql<=tl && tr<=qr) return tree[v].max_val;
        int mid=(tl+tr)>>1;
        return max(query_max_(2*v,tl,mid,ql,qr) ,
                query_max_(2*v+1,mid+1,tr,ql,qr));
    }
    ll query_min_(int v,int tl,int tr,int ql,int qr) {
        if(qr<tl || tr<ql) return INF;
        push_down(v,tl,tr);
        if(ql<=tl && tr<=qr) return tree[v].min_val;
        int mid=(tl+tr)>>1;
        return min(query_min_(2*v,tl,mid,ql,qr) ,
                query_min_(2*v+1,mid+1,tr,ql,qr));
    }

public:
    STBeats_Bit(int n) : n(n) { tree.resize(4*n+4); }
    void build(const vector<ll>& a) { build_(1,1,n,a); }
    void upd_or(int ql,int qr,ll x) { upd_or_(1,1,n,ql,qr,x); }
    void upd_and(int ql,int qr,ll x) { upd_and_(1,1,n,ql,qr,x); }
    void upd_set(int ql,int qr,ll x) { upd_set_(1,1,n,ql,qr,x); }
    ll query_sum(int ql,int qr) { return query_sum_(1,1,n,ql,qr); }
    ll query_and(int ql,int qr) { return query_and_(1,1,n,ql,qr); }
    ll query_or(int ql,int qr) { return query_or_(1,1,n,ql,qr); }
    ll query_max(int ql,int qr) { return query_max_(1,1,n,ql,qr); }
    ll query_min(int ql,int qr) { return query_min_(1,1,n,ql,qr); }
};
int32_t main() {
    ios_base :: sync_with_stdio(0); cin.tie(0);

    int n,q; cin>>n>>q;
    vector<ll> a(n+1);
    for(int i=1;i<=n;i++) cin>>a[i];
    STBeats_Bit t(n);
    t.build(a);
}

/*
this code is for gcd and multiple update like cmax cmin add set 
*/

#include<bits/stdc++.h>

using namespace std;

const long long MX = 1e18;

struct node {
    long long max, max2, min, min2, sum, gcd, add = 0, set = 0, updmin = 0, updmax = 0;
    int cntmax, cntmin;
    node() {}
    node(long long x) {
        sum = max = min = x, cntmax = cntmin = 1;
        gcd = 0;
        max2 = -MX, min2 = MX;
    }
};

vector<node> t;
vector<long long> a;

void merge(node& res, node& a, node& b) {
    // max
    res.max = max(a.max, b.max);
    res.max2 = -MX;
    res.cntmax = 0;
    if (a.max == res.max) {
        res.cntmax += a.cntmax;
        res.max2 = max(res.max2, a.max2);
    } else {
        res.max2 = max(res.max2, a.max);
    }
    if (b.max == res.max) {
        res.cntmax += b.cntmax;
        res.max2 = max(res.max2, b.max2);
    } else {
        res.max2 = max(res.max2, b.max);
    }

    // min
    res.min = min(a.min, b.min);
    res.min2 = MX;
    res.cntmin = 0;
    if (a.min == res.min) {
        res.cntmin += a.cntmin;
        res.min2 = min(res.min2, a.min2);
    } else {
        res.min2 = min(res.min2, a.min);
    }
    if (b.min == res.min) {
        res.cntmin += b.cntmin;
        res.min2 = min(res.min2, b.min2);
    } else {
        res.min2 = min(res.min2, b.min);
    }

    //sum
    res.sum = a.sum + b.sum;

    //gcd
    res.gcd = __gcd(a.gcd, b.gcd);
    long long x = -1, y = -1;
    if (a.max2 != -MX && a.max2 != a.min) {
        x = a.max2;
    }
    if (b.max2 != -MX && b.max2 != b.min) {
        y = b.max2;
    }
    if (x != -1 && y != -1) {
        res.gcd = __gcd(res.gcd, abs(x - y));
    }
    for (long long z : {a.max, a.min, b.max, b.min}) {
        if (z == res.max) {
            continue;
        }
        if (z == res.min) {
            continue;
        }
        if (x != -1) {
            res.gcd = __gcd(res.gcd, abs(x - z));
        } else if (y != -1) {
            res.gcd = __gcd(res.gcd, abs(y - z));
        } else {
            x = z;
        }
    }
}

void push_add(int v, long long x) {
    if (t[v].set != 0) {
        t[v].set += x;
    } else {
        if (t[v].updmin != 0) {
            t[v].updmin += x;
        }
        if (t[v].updmax != 0) {
            t[v].updmax += x;
        }
        t[v].add += x;
    }
}

void push_max(int v, long long x) {
    if (t[v].set != 0) {
        t[v].set = min(t[v].set, x);
    } else if (t[v].updmin == 0 || x > t[v].updmin) {
        if (t[v].updmax == 0) {
            t[v].updmax = x;
        } else {
            t[v].updmax = min(t[v].updmax, x);
        }
    } else {
        t[v].set = x;
    }
}

void push_min(int v, long long x) {
    if (t[v].set != 0) {
        t[v].set = max(t[v].set, x);
    } else if (t[v].updmax == 0 || t[v].updmax > x) {
        if (t[v].updmin == 0) {
            t[v].updmin = x;
        } else {
            t[v].updmin = max(t[v].updmin, x);
        }
    } else {
        t[v].set = x;
    }
}

void push(int v, int l, int r) {
    if (t[v].set != 0) {
        if (l + 1 != r) {
            t[v * 2 + 1].set = t[v * 2 + 2].set = t[v].set;
        }
        t[v].max = t[v].min = t[v].set;
        t[v].cntmax = t[v].cntmin = r - l;
        t[v].sum = t[v].set * (long long) (r - l);
        t[v].add = t[v].set = t[v].gcd = t[v].updmin = t[v].updmax = 0;
        t[v].max2 = -MX, t[v].min2 = MX;
    }
    if (t[v].add != 0) {
        if (l + 1 != r) {
            push_add(v * 2 + 1, t[v].add);
            push_add(v * 2 + 2, t[v].add);
        }
        t[v].max += t[v].add;
        t[v].min += t[v].add;
        if (t[v].max2 != -MX) {
            t[v].max2 += t[v].add;
        }
        if (t[v].min2 != MX) {
            t[v].min2 += t[v].add;
        }
        t[v].sum += t[v].add * (long long) (r - l);
        t[v].add = 0;
    }
    if (t[v].updmax != 0) {
        if (l + 1 != r) {
            push_max(v * 2 + 1, t[v].updmax);
            push_max(v * 2 + 2, t[v].updmax);
        }
        if (t[v].max == t[v].min) {
            if (t[v].updmax < t[v].max) {
                t[v].sum = t[v].updmax * (long long) (r - l);
                t[v].max = t[v].min = t[v].updmax;
            }
        } else {
            if (t[v].updmax < t[v].max) {
                t[v].sum -= (t[v].max - t[v].updmax) * (long long) t[v].cntmax;
                if (t[v].max == t[v].min2) {
                    t[v].min2 = t[v].updmax;
                }
                t[v].max = t[v].updmax;
            }
        }
        t[v].updmax = 0;
    }
    if (t[v].updmin != 0) {
        if (l + 1 != r) {
            push_min(v * 2 + 1, t[v].updmin);
            push_min(v * 2 + 2, t[v].updmin);
        }
        if (t[v].max == t[v].min) {
            if (t[v].updmin > t[v].min) {
                t[v].sum = t[v].updmin * (long long) (r - l);
                t[v].max = t[v].min = t[v].updmin;
            }
        } else {
            if (t[v].updmin > t[v].min) {
                t[v].sum += (t[v].updmin - t[v].min) * (long long) t[v].cntmin;
                if (t[v].min == t[v].max2) {
                    t[v].max2 = t[v].updmin;
                }
                t[v].min = t[v].updmin;
            }
        }
        t[v].updmin = 0;
    }
}

void build(int v, int l, int r) {
    if (l + 1 == r) {
        t[v] = node(a[l]);
        return;
    }
    int m = (l + r) / 2;
    build(v * 2 + 1, l, m);
    build(v * 2 + 2, m, r);
    merge(t[v], t[v * 2 + 1], t[v * 2 + 2]);
}

void updatemin(int v, int l, int r, int l1, int r1, long long x) {
    push(v, l, r);
    if (l1 >= r || l >= r1 || t[v].max <= x) return;
    if (l1 <= l && r <= r1 && t[v].max2 < x) {
        t[v].updmax = x;
        push(v, l, r);
        return;
    }
    int m = (l + r) / 2;
    updatemin(v * 2 + 1, l, m, l1, r1, x);
    updatemin(v * 2 + 2, m, r, l1, r1, x);
    merge(t[v], t[v * 2 + 1], t[v * 2 + 2]);
}

void updatemax(int v, int l, int r, int l1, int r1, long long x) {
    push(v, l, r);
    if (l1 >= r || l >= r1 || t[v].min >= x) return;
    if (l1 <= l && r <= r1 && t[v].min2 > x) {
        t[v].updmin = x;
        push(v, l, r);
        return;
    }
    int m = (l + r) / 2;
    updatemax(v * 2 + 1, l, m, l1, r1, x);
    updatemax(v * 2 + 2, m, r, l1, r1, x);
    merge(t[v], t[v * 2 + 1], t[v * 2 + 2]);
}

void updateset(int v, int l, int r, int l1, int r1, long long x) {
    push(v, l, r);
    if (l1 >= r || l >= r1) return;
    if (l1 <= l && r <= r1) {
        t[v].set = x;
        push(v, l, r);
        return;
    }
    int m = (l + r) / 2;
    updateset(v * 2 + 1, l, m, l1, r1, x);
    updateset(v * 2 + 2, m, r, l1, r1, x);
    merge(t[v], t[v * 2 + 1], t[v * 2 + 2]);
}

void updateadd(int v, int l, int r, int l1, int r1, long long x) {
    push(v, l, r);
    if (l1 >= r || l >= r1) return;
    if (l1 <= l && r <= r1) {
        t[v].add = x;
        push(v, l, r);
        return;
    }
    int m = (l + r) / 2;
    updateadd(v * 2 + 1, l, m, l1, r1, x);
    updateadd(v * 2 + 2, m, r, l1, r1, x);
    merge(t[v], t[v * 2 + 1], t[v * 2 + 2]);
}

long long getsum(int v, int l, int r, int l1, int r1) {
    push(v, l, r);
    if (l1 >= r || l >= r1) return 0ll;
    if (l1 <= l && r <= r1) return t[v].sum;
    int m = (l + r) / 2;
    return getsum(v * 2 + 1, l, m, l1, r1) + getsum(v * 2 + 2, m, r, l1, r1);
}

long long getmin(int v, int l, int r, int l1, int r1) {
    push(v, l, r);
    if (l1 >= r || l >= r1) return MX;
    if (l1 <= l && r <= r1) return t[v].min;
    int m = (l + r) / 2;
    return min(getmin(v * 2 + 1, l, m, l1, r1), getmin(v * 2 + 2, m, r, l1, r1));
}

long long getmax(int v, int l, int r, int l1, int r1) {
    push(v, l, r);
    if (l1 >= r || l >= r1) return -MX;
    if (l1 <= l && r <= r1) return t[v].max;
    int m = (l + r) / 2;
    return max(getmax(v * 2 + 1, l, m, l1, r1), getmax(v * 2 + 2, m, r, l1, r1));
}

long long getgcd(int v, int l, int r, int l1, int r1) {
    push(v, l, r);
    if (l1 >= r || l >= r1) return 0ll;
    if (l1 <= l && r <= r1) {
        long long res = __gcd(t[v].max, t[v].min);
        if (t[v].max2 != t[v].min && t[v].max2 != -MX) {
            res = __gcd(res, t[v].gcd);
            res = __gcd(res, t[v].max2);
        }
        return res;
    }
    int m = (l + r) / 2;
    return __gcd(getgcd(v * 2 + 1, l, m, l1, r1), getgcd(v * 2 + 2, m, r, l1, r1));
}
\end{lstlisting}

\subsection*{SEGTree with Hashing}
\begin{lstlisting}

const ll p=137; const ll N=2e5+10; // check range
const pair<ll,ll> mod={127657753,987654319};
ll powerr(ll a,ll b,ll mod){
	ll r=1;
	while(b){
		if(b%2) r=((r%mod) *(a%mod))%mod;
		a=((a%mod)*(a%mod))%mod;
		b/=2;
	}
	return r;
}
ll add(ll a,ll b,ll mod){return ((a%mod)+(b%mod)+mod)%mod;}
ll substract(ll a,ll b,ll mod){return ((a%mod)-(b%mod)+mod)%mod;}
ll mult(ll a,ll b,ll mod) {return ((a%mod)*(b%mod))%mod;}
ll fn(char ch){if(islower(ch)) return ch-'a'+1;if(isupper(ch)) return ch-'A'+1;return ch-'0'+1;}
// ll fn(ll a[i]) return a[i]; //for integer hash

pair<ll,ll> pw[N+10],inv[N+10],inv_p_minus1;
void precal(){
	pw[0].F=pw[0].S=1;
	for(int i=1;i<N;i++){
		pw[i].F=mult(pw[i-1].F,p,mod.F);
		pw[i].S=mult(pw[i-1].S,p,mod.S);
	} 
	ll pw_inv1=powerr(p,mod.F-2,mod.F);
	ll pw_inv2=powerr(p,mod.S-2,mod.S);
	inv[0].F=inv[0].S=1;
	for(int i=1;i<N;i++){
		inv[i].F=mult(inv[i-1].F,pw_inv1,mod.F);
		inv[i].S=mult(inv[i-1].S,pw_inv2,mod.S);
	}	
    inv_p_minus1 = {
        powerr(p-1, mod.F-2, mod.F),
        powerr(p-1, mod.S-2, mod.S)
    };
}
struct hashing {
  vector<pair<ll,ll>> t;
  vector<char>lazy; // lazy of integer for integer hash 
  string s; // integer hash make vector<ll> a
  hashing(){}
  hashing(string _s){
    s=_s;
    ll n=s.size();
    t.resize(n*4);
    lazy.resize(n*4,'?');
  }
  inline void push(int node,int l,int r){
    if(lazy[node]=='?') return;
    ll len=(r-l+1);
    ll sum1 = mult(mult(substract(pw[len].F, 1, mod.F), inv_p_minus1.F, mod.F), pw[l].F, mod.F);
    ll sum2 = mult(mult(substract(pw[len].S, 1, mod.S), inv_p_minus1.S, mod.S), pw[l].S, mod.S);

    t[node].F = mult(sum1, fn(lazy[node]), mod.F);
    t[node].S = mult(sum2, fn(lazy[node]), mod.S);
    if(l!=r){
        lazy[node*2]=lazy[node*2+1]=lazy[node];
    }
    lazy[node]='?';
  }
  inline void here(int node){
      t[node].F=add(t[node*2].F,t[node*2+1].F,mod.F);
      t[node].S=add(t[node*2].S,t[node*2+1].S,mod.S);
  }
  void build(int node,int l,int r){
    if(l==r){
        t[node].F=mult(pw[l].F,fn(s[l]),mod.F);
        t[node].S=mult(pw[l].S,fn(s[l]),mod.S);
        return;
    }
    ll mid=(l+r)>>1;
    build(node*2,l,mid);
    build(node*2+1,mid+1,r);
    here(node);
  }
  void upd(int node,int l,int r,int i,int j,char value){
    push(node,l,r);
    if(l>j || r<i) return;
    if(i<=l && r<=j){
        lazy[node]=value;
        push(node,l,r);
        return;
    }
    ll mid=(l+r)>>1;
    upd(node*2,l,mid,i,j,value);
    upd(node*2+1,mid+1,r,i,j,value);
    here(node);
  }
  pair<ll,ll> query(int node,int l,int r,int i,int j){
    push(node,l,r);
    if(l>j || r<i) return {0,0};           /// check here
    if(i<=l && r<=j) return t[node];
    ll mid=(l+r)>>1;
    pair<ll,ll> x=query(node*2,l,mid,i,j);
    pair<ll,ll> y=query(node*2+1,mid+1,r,i,j);
    return {add(x.F,y.F,mod.F),add(x.S,y.S,mod.S)};
  }
  pair<ll,ll> get_hash(int l,int r,int n){
    pair<ll,ll> ck=query(1,0,n-1,l,r);
    ck.F=mult(ck.F,inv[l].F,mod.F);
    ck.S=mult(ck.S,inv[l].S,mod.S);
    return ck;
  }
}a;
int main(){
	precal();
  ll n,m,x; cin>>n>>m>>x;
  ll q=m+x;
  string s; cin>>s;
  a = hashing(s);
  a.build(1,0,n-1);
  while(q--){
    ll i; cin>>i;
    if(i==1){
      ll l,r; char c; cin>>l>>r>>c; l--,r--;
      a.upd(1,0,n-1,l,r,c);
    }else{
      ll l,r,d; cin>>l>>r>>d;
      --l,--r;
      if(d==(r-l+1) || a.get_hash(l,r-d,n)==a.get_hash(l+d,r,n)) 
      cout<<"YES"<<endl;
      else cout<<"NO"<<endl;
    }
  }
}
\end{lstlisting}
\subsection*{Fast Fourier Transform (FFT)}
{ \footnotesize
\textbf{Description:} Iterative Cooley-Tukey FFT. Computes convolution of two polynomials $A$ and $B$. \\
\textbf{Optimization:} Packs $A$ into real part and $B$ into imaginary part ($P(x) = A(x) + iB(x)$) to compute DFT of both using a \textbf{single forward FFT} call. Total ops: 1 Forward FFT + 1 Inverse FFT. \\
\textbf{Time:} $\mathcal{O}(N \log N)$, where $N$ is the smallest power of 2 $\ge |A| + |B| - 1$. \\
\textbf{Note:} Uses \texttt{complex<double>}. Precision errors may occur for result values $> 10^{14}$.
}
\begin{lstlisting}
class FFT {
    using cd = complex<double>;
    static constexpr double PI = acos(-1.0);
    void fft(vector<cd>& a, bool invert) const {
        int n = (int)a.size();
        for (int i = 1, j = 0; i < n; ++i) {
            int bit = n >> 1;
            for (; j & bit; bit >>= 1) j ^= bit;
            j ^= bit;
            if (i < j) swap(a[i], a[j]);
        }

        for (int len = 2; len <= n; len <<= 1) {
            double ang = 2 * PI / len * (invert ? -1 : 1);
            cd wlen(cos(ang), sin(ang));
            for (int i = 0; i < n; i += len) {
                cd w(1);
                for (int k = 0; k < len/2; ++k) {
                    cd u = a[i + k];
                    cd v = a[i + k + len/2] * w;
                    a[i + k] = u + v;
                    a[i + k + len/2] = u - v;
                    w *= wlen;
                }
            }
        }
        if (invert) for (cd & x : a) x /= n;
    }

    static int next_pow2(int x) {
        int n = 1;
        while (n < x) n <<= 1;
        return n;
    }

public:

    vector<long long> multiply(const vector<ll>& A, const vector<ll>& B ) const {
        if (A.empty() || B.empty()) return {};
        int n = (int)A.size(), m = (int)B.size();
        int need = n + m - 1;
        int sz = next_pow2(need);

        vector<cd> fa(sz);
        for (int i = 0; i < n; ++i) fa[i].real((double)A[i]);
        for (int i = 0; i < m; ++i) fa[i].imag((double)B[i]);

        fft(fa, false);

        vector<cd> fb(sz);
        for (int i = 0; i < sz; ++i) {
            int j = (i == 0 ? 0 : sz - i);
            cd a1 = (fa[i] + conj(fa[j])) * cd(0.5, 0.0);
            cd b1 = (fa[i] - conj(fa[j])) * cd(0.0, -0.5);
            fb[i] = a1 * b1;
        }

        fft(fb, true);

        vector<long long> res(need);
        for (int i = 0; i < need; ++i) res[i] =llround(fb[i].real());
        return res;
    }
};
\end{lstlisting}
% ================= NUMBER THEORY =================
\section*{Number Theory}

\subsection*{nCr \& nPr}
\begin{lstlisting}
const ll mod=1e9+7;
ll fact[69];
ll poW(ll x, ll n){
    ll result = 1;
    while (n > 0){
        if (n & 1LL == 1){
            result = (result * x)%mod;
        }
        x = (x * x)%mod;
        n = n >> 1LL;
    }
    return result%mod;
}
ll nCr(ll n,ll r){
    return (fact[n] * poW((fact[r]*fact[n-r])%mod,mod-2)) % mod;
}
ll nPr(ll n,ll r){
    return (fact[n] * poW(fact[n-r]%mod,mod-2)) % mod;
}
int32_t main(){
    fact[0]=1;
    for(int i=1;i<=60;i++){
        fact[i]=(fact[i-1]*i*1LL)%mod;
    }
}
\end{lstlisting}

\subsection*{Sieve \& Primes}
{ \footnotesize
\textbf{Description:} Linear Sieve (spf), Segmented Sieve, Segmented Factorization, $\phi(n)$ (Euler Totient), Factorization. \\
\textbf{Time:} Sieve $\mathcal{O}(N)$, Factorize $\mathcal{O}(\log N)$ (with spf) or $\mathcal{O}(\sqrt{N})$.
}
\begin{lstlisting}
struct NumberTheory {
  static ll power(ll x, ll n) {
    ll res = 1;
    while (n > 0) {
      if (n & 1)
        res *= x;
      x *= x;
      n >>= 1;
    }
    return res;
  }
  vector<ll> primes;
  vector<int> spf;
  void sieve(ll n) { // O(n)
    spf.assign(n + 1, 0);
    for (int i = 2; i <= n; ++i) {
      if (!spf[i]) {
        spf[i] = i;
        primes.PB(i);
      }
      for (auto j : primes) {
        ll prime = j;
        ll composite_num = 1LL * i * prime;
        if (composite_num > n)
          break;
        spf[composite_num] = prime;
        if (prime == spf[i])
          break;
      }
    }
  }
  vector<ll> segmentedSieve(ll L, ll R) {
    vector<bool> mark(R - L + 1, true);
    if (L == 1)
      mark[0] = false;
    for (auto p : primes) {
      if (1LL * p * p > R)
        break;
      ll base = max(p * p, ((L + p - 1) / p) * p);
      for (ll j = base; j <= R; j += p)
        mark[j - L] = false;
    }
    vector<ll> seg;
    for (ll i = 0; i <= R - L; i++)
      if (mark[i])
        seg.push_back(L + i);
    return seg;
  }
  vector<vector<ll>> segment_factor;
  void segment_fact(ll L, ll R) {
    segment_factor.assign(R - L + 1, vector<ll>());
    vector<ll> range_primes(R - L + 1);
    for (ll i = 0; i <= R - L; i++)
      range_primes[i] = L + i;
    for (auto p : primes) {
      if (1LL * p * p > R)
        break;
      ll base = p * ((L + p - 1) / p);

      for (ll j = base; j <= R; j += p) {
        ll index = j - L;
        while (!(range_primes[index] % p)) {
          segment_factor[index].PB(p);
          range_primes[index] /= p;
        }
      }
    }
    for (ll i = 0; i <= R - L; i++) {
      if (range_primes[i] <= 1)
        continue;

      segment_factor[i].PB(range_primes[i]);
    }
  }
  vector<ll> factorize(ll n) {
    vector<ll> f;
    for (auto p : primes) {
      if (1LL * p * p > n)
        break;
      while (n % p == 0) {
        f.push_back(p);
        n /= p;
      }
    }
    if (n > 1)
      f.push_back(n);
    return f;
  }
  ll phi(ll n) {
    ll res = n;
    for (auto p : primes) {
      if (1LL * p * p > n)
        break;
      if (n % p == 0) {
        while (n % p == 0)
          n /= p;
        res -= res / p;
      }
    }
    if (n > 1)
      res -= res / n;
    return res;
  }
  ll phi2(ll n) {
    vector<ll> v = factorize(n);
    map<ll, ll> mp;
    ll res = 1;
    for (int i = 0; i < v.size(); ++i) {
      ll p = v[i], exp = 0;
      while (i < v.size() && v[i] == p) {
        exp++;
        i++;
      }
      i--;
      res *= power(p, exp - 1) * (p - 1);
    }
    return res;
  }
  static ll xorUpto(ll n) {
    ll x = n % 4;
    if (x == 0)
      return n;
    if (x == 1)
      return 1;
    if (x == 2)
      return n + 1;
    return 0;
  }
  static ll nCr(ll n, ll r) {
    if (r > n)
      return 0;
    r = min(r, n - r);
    ll res = 1;
    for (ll i = 1; i <= r; i++) {
      res = res * (n - i + 1) / i;
    }
    return res;
  }
} P;
\end{lstlisting}

\subsection*{Pollard Rho \& Miller Rabin}
{ \footnotesize
\textbf{Description:} Deterministic Miller-Rabin primality test (up to $10^{18}$) and Pollard's Rho factorization. Requires \texttt{\_\_int128} for modular multiplication to avoid overflow. \\
\textbf{Time:} Primality $\mathcal{O}(k \log^3 N)$, Factorization $\mathcal{O}(N^{1/4})$.
}
\begin{lstlisting}
// this is the topic to find prime fact of a big number 
using ll = unsigned long long;
mt19937_64 rng(chrono::steady_clock::now().time_since_epoch().count());
ll rand(ll n) { return rng() % (n - 2) + 1; }
ll modMul(ll a,ll b,ll mod) {
    return (__int128)a*b%mod;
}
ll modPower(ll base,ll exp,ll mod) {
    ll res=1;
    base%=mod;
    while(exp>0) {
        if(exp%2==1) res=modMul(res,base,mod);
        base=modMul(base,base,mod);
        exp/=2;
    }
    return res;
}
ll gcd(ll a,ll b) {
    while(b) {
        a%=b;
        swap(a,b);
    }
    return a;
}
const int MAX_SIEVE=1000001;
vector<int> spf(MAX_SIEVE);
void init_sieve() {
    vector<int> primes;
    for(int i=2;i<MAX_SIEVE;++i) {
        if(!spf[i]) {
            spf[i]=i;
            primes.PB(i);
        }
        for(int p:primes) {
            if(i*(ll)p>=MAX_SIEVE) break;
            spf[i*p]=p;
            if(!(i%p)) break;
        }
    }
}
bool MillerRabin(ll n,ll a,ll d,int s) {
    ll x=modPower(a,d,n);
    if(x==1 || x==n-1) return true;
    for(int r=1;r<s;r++) {
        x=modMul(x,x,n);
        if(x==1) return false;
        if(x==n-1) return true;
    }
    return false;
}
bool isPrime(ll n) {
    if(n<=1) return false;
    if(n<MAX_SIEVE) return spf[n]==n;
    if(n==2 || n==3) return true;
    if(!(n%2)) return false;
    ll d=n-1;
    int s=0;
    while(!(d%2)) {
        d/=2;
        s++;
    }
    vector<ll> witnesses={2,3,5,7,11,13,17,19,23,29,31,37};
    for(ll a:witnesses) {
        if(n==a) return true;
        if(!(MillerRabin(n,a,d,s))) return false;
    }
    return true;
}
ll pollard_rho(ll n) {
    auto f =[&](ll x,ll c) {
        return (modMul(x,x,n)+c)%n;
    };
    ll c=rand(n);
    ll tortoise=2,hare=2,d=1;
    ll product=1;
    const int BATCH_SIZE=128;
    int count=0;
    while(1) {
        tortoise=f(tortoise,c);
        hare=f(f(hare,c),c);
        if(tortoise==hare) {
            c=rand(n);
            tortoise=2; hare=2; product=1; count=0;
            continue;
        }
        ll prev_product=product,diff;
        if(tortoise>hare) diff=tortoise-hare;
        else diff=hare-tortoise;
        product=modMul(product,diff,n);
        if(!product) {
            d=gcd(prev_product,n);
            if(d==1) d=gcd(diff,n);
            break;
        }
        count++;
        if(count==BATCH_SIZE) {
            d=gcd(product,n);
            if(d>1) break;
            count=0;
            product=1;
        }
    }
    if(d==n || d==1) return pollard_rho(n);
    return d;
}
void factorize(ll n,vector<ll>& primeFactors) {
    if(n<=1) return;
    while(!(n%2)) {
        primeFactors.PB(2);
        n/=2;
    }
    if(n==1) return;
    while(n>1 && n<MAX_SIEVE) {
        primeFactors.PB(spf[n]);
        n/=spf[n];
    }
    if(n==1) return;
    if(isPrime(n)) {
        primeFactors.PB(n);
        return;
    }
    ll d=pollard_rho(n);
    factorize(d,primeFactors);
    factorize(n/d,primeFactors);
}
int32_t main() {
    init_sieve(); // run it before testcase
    ll n; cin>>n;
    vector<ll> ans;
    factorize(n,ans);
}
\end{lstlisting}

\subsection*{Mobius Function}
{ \footnotesize
\textbf{Description:} Linear Sieve to compute $\mu(i)$ and $\phi(i)$. $h[i]$ stores helper values for LCM sums. \\
\textbf{Time:} $\mathcal{O}(N)$.
}
\begin{lstlisting}
const int MX=1000001;
vector<int> mu(MX);
vector<int> phi(MX);
vector<int> spf(MX);
vector<ll> h(MX,0); // for LCM
vector<int> primes;
void mobius_sieve(){
  mu[1]=1; h[1]=1;
  for(int i=2;i<MX;i++){
    if(!spf[i]){
        spf[i]=i;
        mu[i]=-1;
        phi[i]=i-1;
        h[i]=(1-i+MOD);
        primes.PB(i);
    }
    for(int p:primes){
        if(1LL*i*p>=MX) break;
        spf[i*p]=p;
        if(!(i%p)){
          h[i*p]=h[i];
          phi[i*p]=phi[i]*p;
          mu[i*p]=0;
          break;
        }else {
          mu[i*p]=-mu[i];
          phi[i*p]=phi[i]*(p-1);
          h[i*p]=(h[i]*h[p])%MOD;
        }
    }
  }
}
\end{lstlisting}

\subsection*{Mobius Inversion Formulas}
{ \footnotesize
\textbf{Description:}
\begin{enumerate}[leftmargin=*, nosep]
    \item \texttt{count(n,k)}: Pairs with $\text{gcd}(i,j)=k$. Uses $\sum_{d=1}^{\lfloor n/k \rfloor} \mu(d) \lfloor \frac{n}{kd} \rfloor^2$.
    \item \texttt{count(n)}: Sum of $\text{gcd}(i,j)$ for $1 \le i,j \le n$.
    \item \texttt{solve\_lcm}: Count subsequences with $\text{LCM}=k$.
    \item \texttt{primitive}: Count primitive strings.
\end{enumerate}
}
\begin{lstlisting}
// count gcd(i,j)==1 hard
// but count of gcd(i,j)%k==0 is easy cause i%k==0 and j%k==0
// that is N/k this much value can be divide by k and pairs are (N/k)*(N/k)
ll count(int n, int k)
{ // count gcd(i,j)==k i,j<=n
  n /= k;
  if (!n)
    return 0;
  ll ans = 0;
  for (int i = 1; i <= n; i++)
  { // this will find in O(n)
    ll g_i = (n / i) * (n / i);
    ans += 1LL * mu[i] * g_i;
  }
  return ans;
}
ll count_faster(int n, int k)
{ // this will find in O(sqrt n)
  n /= k;
  if (!n)
    return 0;
  ll ans = 0;
  for (int l = 1; l <= n;)
  {
    int val = n / l;
    int r = n / val;
    ll g_val = 1LL * val * val;
    ll mu_sum = mu_pre[r] - mu_pre[l - 1];
    ans += mu_sum * g_val;
    l = r + 1;
  }
  return ans;
}
ll count(ll n)
{
  ll ans = 0;
  for (int i = 1; i <= n;)
  {
    ll val = n / i;
    if (!val)
      break;
    ll r = n / val;
    ll g_val = (val * (val - 1)) / 2;
    ans += g_val * (pre_phi[r] - pre_phi[i - 1]);
    i = r + 1;
  }
  return ans;
}
void solve_lcm()
{ // ans for subsequence LCM=k;
  mobius_sieve();
  pow2[0] = 1;
  for (int i = 1; i < mx; i++)
    pow2[i] = pow2[i - 1] * 2;
  int n;
  cin >> n;
  map<int, int> freq;
  for (int i = 1; i <= n; i++)
  {
    int x;
    cin >> x;
    freq[x]++;
  }
  // now calculating the easy g[k] that is c[k]= count of numbers in A that divides k
  vector<int> c(mx, 0);
  for (auto const &[val, count] : freq)
  {
    for (int k = val; k < mx; k += val)
      c[k] += count;
  }
  vector<mi> g(mx);
  for (int k = 1; k < mx; k++)
  {
    g[k] = pow2[c[k]] - 1;
  }
  // f[n] = sum( g[d] * mu[n/d] )
  vector<mi> f(mx, 0);
  for (int d = 1; d < mx; d++)
  {
    // if(!g[d]) continue;
    for (int n = d; n < mx; n += d)
      f[n] += g[d] * mu[n / d];
  }
  // f[k] is the ans for subsequence LCM=k;
  int k;
  cin >> k;
  cout << f[k] << endl;
}
/*
*****Problem Statement: "Given N, and an alphabet of K letters,
find the number of primitive strings of length n for all n from 1 to N."
(A string is primitive if it's not a repetition of a smaller block,
e.g., "abcab" is primitive, but "ababab" is not).
*/
void solve_primitive_strings()
{
  int n = 100000, k = 26;
  mobius_sieve();
  vector<mi> g(n + 1);
  g[0] = 1;
  for (int i = 1; i <= n; i++)
    g[i] = g[i - 1] * k;
  vector<mi> f(mx, 0);
  for (int d = 1; d < mx; d++)
  {
    // if(!g[d]) continue;
    for (int n = d; n < mx; n += d)
      f[n] += g[d] * mu[n / d];
  }
  // cout << "Primitive strings of length 4 (K=26): " << f[4] << endl;
}
\end{lstlisting}

\subsection*{Mobius LCM Array}
{ \footnotesize
\textbf{Description:} Computes sum of LCM of all pairs in an array. Uses precomputed $h[i]$ from sieve.
}
\begin{lstlisting}
  int n; cin>>n;
  int mx=0;
  vector<int> v(n+1);
  mi sum=0;
  for(int i=1;i<=n;i++) {
    cin>>v[i];
    mx=max(mx,v[i]);
    sum+=v[i];
  }
  vector<int>fre(mx+1,0);
  for(int i=1;i<=n;i++) fre[v[i]]++;
  vector<ll>mp(mx+1,0);
  for(int i=1;i<=mx;i++) {
    for(int j=i;j<=mx;j+=i) {
      ll k=j/i;
      mp[i]+=1LL*k*fre[j];
    }
  }
  mi ans=0;
  for(int i=1;i<=mx;i++) {
    mi term=mi(i)*mi(h[i])*mi(mp[i])*mi(mp[i]);
    ans+=term;
  }
  cout<<ans<<endl; // all pair lcm sum
  cout<<mi(ans-sum)<<endl; // exclude i=j
  mi inv=mi((MOD+1)/2);
  cout<<mi(mi(ans-sum)*inv)<<endl; // all pair lcm i<j
\end{lstlisting}

\subsection*{Fast Prime Count}
{ \footnotesize
\textbf{Description:} Counts $\pi(n)$ (number of primes $\le n$) in sub-linear time. \\
\textbf{Time:} $\mathcal{O}(N^{2/3})$.
}
\begin{lstlisting}
const int N=3e5+9;
namespace pcf {
    #define MAXN 20000010
    #define MAX_PRIMES 2000010
    #define PHI_N 100000
    #define PHI_K 100
    int len=0; // number of prime gen by sieve
    int primes[MAX_PRIMES];
    int pref[MAXN]; // number of primes <=i
    int dp[PHI_N][PHI_K];
    bitset<MAXN> f;
    void sieve(int n) {
        f[1]=true;
        for(int i=4;i<=n;i+=2) f[i]=true;
        for(int i=3;i*i<=n;i+=2) {
            if(!f[i]) {
                for(int j=i*i;j<=n;j+=i<<1) f[j]=true;
            }
        }
        for(int i=1;i<=n;i++) {
            if(!f[i]) primes[len++]=i;
            pref[i]=len;
        }
    }
    void init() {
        sieve(MAXN-1);
        for(int n=0;n<PHI_N;n++) dp[n][0]=n;
        for(int k=1;k<PHI_K;k++) {
            for(int n=0;n<PHI_N;n++) {
                dp[n][k]=dp[n][k-1]-dp[n/primes[k-1]][k-1];
            }
        }
    }
    ll bro(ll n,int k) { // number of int <=n not div by first k primes
        if(n<PHI_N && k<PHI_K) return dp[n][k];
        if(k==1) return ((++n)>>1);
        if(primes[k-1]>=n) return 1;
        return bro(n,k-1)-bro(n/primes[k-1],k-1); 
    }
    ll lehmer(ll n) { // runs under 0.2s for n=1e12
        if(n<MAXN) return pref[n];
        ll w,res=0;
        int b=sqrt(n),c=lehmer(cbrt(n)),a=lehmer(sqrt(b));b=lehmer(b);
        res=bro(n,a)+((1LL*(b+a-2)*(b-a+1))>>1);
        for(int i=a;i<b;i++) {
            w=n/primes[i];
            int lim=lehmer(sqrt(w)); res-=lehmer(w);
            if(i<=c) {
                for(int j=i;j<lim;j++) {
                    res+=j;
                    res-=lehmer(w/primes[j]);
                }
            }
        }
        return res;
    }
}
int32_t main() {
    pcf::init();
    ll n; cin>>n;
    cout<<pcf::lehmer(n)<<endl;
}
\end{lstlisting}

\subsection{Modular Combinatorics}
{ \footnotesize
\textbf{Description:} Solves Combinations when standard Fermat's Little Theorem fails. \\
1. \texttt{lucas}: Use when $n, r$ are huge ($10^{18}$) but $p$ is small ($10^5$). \\
2. \texttt{nCr\_pk}: Use when modulus is a prime power (e.g., $27, 25$) so inverses don't normally exist (removes factor $p$). \\
\textbf{Time:} Lucas $O(p + \log_p n)$. Prime Power $O(p^k \log n)$.
}
\begin{lstlisting}
ll power(ll base,ll exp,ll mod) {
	ll res=1;
	base%=mod;
	while(exp>0) {
		if(exp%2==1) res=(res*base)%mod;
		base=(base*base)%mod;
		exp/=2;
	}
	return res;
}
// Use this for simple primes like 3,5,7..
ll nCr(ll n,ll r,ll p) {
	if(r<0 || r>n) return 0;
	if(!r || r==n) return 1;
	if(r>n/2) r=n-r;
	ll num=1;
	ll den=1;
	for(ll i=1;i<=r;i++) {
		num=(num*(n-i+1))%p;
		den=(den*i)%p;
	}
	return (num*power(den,p-2,p))%p;
}
// Use this to calculate nCr % p when n & r is huge p is small 
ll lucas(ll n,ll r,ll p) {
	if(!r) return 1;
	return (lucas(n/p, r/p, p) * nCr(n%p, r%p, p)) %p;
}
//---> billow part use to calcuate nCr if mod is p^k
ll extended_euclid(ll a,ll b,ll &x,ll &y) { // ax+by=gcd(a,b)
	if(!b) {
		x=1,y=0;
		return a;
	}
	ll x1,y1;
	ll d=extended_euclid(b,a%b,x1,y1);
	x=y1;
	y=x1-y1*(a/b);
	
	return d;
}
ll inverse_pk(ll n,ll mod) {
	ll x,y;
	extended_euclid(n,mod,x,y);
	return (x%mod+mod)%mod;
}
ll fact_no_p(ll n,ll p,ll pk) {
	if(!n) return 1;
	ll ans=1;
	for(ll i=1;i<=pk;i++) {
		if(i%p) ans=(ans*i)%pk;
	}	
	ans=power(ans,n/pk,pk);
	for(ll i=1;i<=n%pk;i++) {
		if(i%p) ans=(ans*i)%pk;
	}
	return (ans*fact_no_p(n/p,p,pk))%pk;
}
ll count_p(ll n,ll p) {
	ll ans=0;
	while(n) {
		ans+=n/p;
		n/=p;
	}
	return ans;
}
// nCr % p^k
// Use this for cases like 27 (3^3), 25 (5^2)
// pk=p^k
ll nCr_pk(ll n,ll r,ll p,ll pk) {
	if(r<0 || r>n) return 0;
	ll num=fact_no_p(n,p,pk);
	ll den1=fact_no_p(r,p,pk);
	ll den2=fact_no_p(n-r,p,pk);
	ll ans=(num*inverse_pk(den1,pk))%pk;
	ans=(ans*inverse_pk(den2,pk))%pk;
	ll pow_p=count_p(n,p)-count_p(r,p)-count_p(n-r,p);
	ans=(ans*power(p,pow_p,pk))%pk;
	
	return ans;
}
\end{lstlisting}

\subsection*{Kth FIB K$\le 10^{18}$}
\begin{lstlisting}
/*
 * Note : If MOD is constant then use (const int mod=Given mod)
 * and remove mod from function variable declare it will make it 
 * 5-10X faster if no need __int128 then remove it from mul more fast
 * For prefix sum f(n+2)-(a+b)
 * sum of f(0)^2+f(1)^2+..+f(n)^2 = f(n)*f(n+1)
 * gcd(f(n),f(m)) = f(gcd(n,m))
 * odd index sum = f(2n)
 * evne index sum f(2n+1)-1
 * THEOREM: Every positive integer N can be uniquely represented as the sum 
    of non-consecutive Fibonacci numbers. (i.e., If you use F[i], you cannot use F[i-1] or F[i+1]).

 2. SEQUENCE: Uses Fib starting 1, 2, 3, 5, 8... (Index: F[0]=1, F[1]=2...)
 3. EXAMPLE: N = 100
    - Largest Fib <= 100 is 89. (Rem = 11)
    - Largest Fib <= 11 is 8.   (Rem = 3)
    - Largest Fib <= 3 is 3.    (Rem = 0)
    -> 100 = 89 + 8 + 3
*/
const int mod=1e8+7;
inline ll mul(ll a,ll b,ll mod) {
	return (__int128)a*b%mod;
}
// Works for any modulo m
pair<ll,ll> FIB(ll n,ll mod) {
	if(!n) return {0,1};
	ll a=0,b=1;
	for(int i=63-__builtin_clzll(n);i>=0;i--) {
		ll c=mul(a,(2*b%mod-a+mod)%mod,mod);	// F(2k) 
		ll d=(mul(a,a,mod)+mul(b,b,mod))%mod;	// F(2k+1) 
		if((n>>i)&1) {
			a=d;		 // F(2k+1)
			b=(c+d)%mod; // F(2k+2)
		}else {
			a=c;		// F(2k) 
			b=d;		// F(2k+1)
		}
	}
	return {a,b};
}
ll kth(ll a,ll b,ll n,ll mod) {
	if(mod==1) return 0;
	if(!n) return a%mod;
	if(n==1) return b%mod;
	pair<ll,ll> fibs=FIB(n-1,mod); // 
	return (mul(a,fibs.F,mod) + mul(b,fibs.S,mod))%mod;
}

void GLITCH_() {
	ll n; cin>>n;
	cout<<kth(0,1,n,mod)<<endl;
}
\end{lstlisting}

\subsection*{Kth FIB n is large}
\begin{lstlisting}
inline ll mul(ll a,ll b,ll mod) {
	return (__int128)a*b%mod;
}
struct Mat {
	ll m[2][2];
	Mat() {m[0][0]=m[0][1]=m[1][0]=m[1][1]=0;}
};
Mat matMul(Mat A,Mat B,ll mod) {
	Mat C;
	for(int i=0;i<2;i++) {
		for(int j=0;j<2;j++) {
			for(int k=0;k<2;k++) {
				C.m[i][j]=(C.m[i][j]+mul(A.m[i][k],B.m[k][j],mod))%mod;
			}
		}
	}
	return C;
}
Mat matPow(Mat A,ll p,ll mod) {
	Mat res; res.m[0][0]=res.m[1][1]=1;
	while(p) {
		if(p&1) res=matMul(res,A,mod);
		A=matMul(A,A,mod);
		p>>=1;
	}
	return res;
}
ll kth_string(ll a,ll b,string n,ll mod) {
	if(mod==1) return 0;
	Mat T;
	T.m[0][0]=1; T.m[0][1]=1;
	T.m[1][0]=1; T.m[1][1]=0;
	Mat res;
	res.m[0][0]=1; res.m[1][1]=1;
	
	for(char c:n) {
		int digit=c-'0';
		res=matPow(res,10,mod);
		res=matMul(res,matPow(T,digit,mod),mod);
	}
	return (mul(a,res.m[1][1],mod)+mul(b,res.m[1][0],mod))%mod;
}
ll kth(ll a,ll b,ll n,ll mod) {
    if(mod==1) return 0;
    if(n==0) return a%mod;
    if(n==1) return b%mod;
    Mat T;
    T.m[0][0]=1; T.m[0][1]=1;
    T.m[1][0]=1; T.m[1][1]=0;

    // Use binary exponentiation directly
    Mat res=matPow(T,n,mod);

    return (mul(a,res.m[1][1],mod) + mul(b,res.m[1][0],mod))%mod;
}
// this can cal nth fib for n is large or <=1e18
void GLITCH_() {
	string n; cin>>n;
	cout<<kth_string(0,1,n,1e8+7)<<endl;
}
\end{lstlisting}

\subsection*{Extended EGCD}
{ \footnotesize
\textbf{Description:} Solves $ax + by = \gcd(a, b)$. Essential for finding Modular Inverse when $M$ is \textbf{not prime} (unlike Fermat's Little Theorem) and solving Linear Diophantine Equations. \\
\textbf{Time:} $O(\log(\min(a, b)))$.
}
\begin{lstlisting}
ll extended_euclid(ll a,ll b,ll &x,ll &y) { // ax+by=gcd(a,b)
	if(!b) {
		x=1,y=0;
		return a;
	}
	ll x1,y1;
	ll d=extended_euclid(b,a%b,x1,y1);
	x=y1;
	y=x1-y1*(a/b);
	
	return d;
}
ll inverse(ll a,ll m) {
	ll x,y; 
	ll g=extended_euclid(a,m,x,y);
	if(g!=1) return -1;
	return (x%m+m)%m;
}
int main() {
	ll a,b; cin>>a>>b;
	ll x,y, gc=extended_euclid(a,b,x,y);
}
\end{lstlisting}

\subsection*{CRT}
{ \footnotesize
\textbf{Description:} Solves the system of congruences $x \equiv a_i \pmod{m_i}$. Works even if moduli are \textbf{not coprime}. Returns $\{x, L\}$ where $x$ is the unique solution modulo $L = \text{lcm}(m_i)$. Returns $\{-1, -1\}$ if no solution exists. \\
\textbf{Time:} $O(N \log(\text{lcm}(M)))$.
}
\begin{lstlisting}
ll extended_euclid(ll a,ll b,ll &x,ll &y) { // ax+by=gcd(a,b)
	if(!b) {
		x=1,y=0; return a;
	}
	ll x1,y1;
	ll d=extended_euclid(b,a%b,x1,y1);
	x=y1;
	y=x1-y1*(a/b);
	return d;
}
/** Works for non-coprime moduli.
 Returns {-1,-1} if solution does not exist or input is invalid.
 Otherwise, returns {x,L}, where x is the solution unique to mod L
*/
pair<ll,ll> CRT(vector<ll>A, vector<ll>M) {
	if(A.size()!=M.size()) return {-1,-1};
	int n=A.size();
	ll a1=A[0];
	ll m1=M[0];
	for(int i=1;i<n;i++) {
		ll a2=A[i];
		ll m2=M[i];
		ll g=__gcd(m1,m2);
		if(a1%g != a2%g) return {-1,-1};
		// Marge two equation 
		ll p,q, d=extended_euclid(m1/g,m2/g,p,q);
		ll mod=m1/g*m2; // LCM of m1,m2
		ll x = ((__int128)a1 * (m2 / g) * q + (__int128)a2 * (m1 / g) * p) % mod;
		
		a1=(x+mod)%mod;
		m1=mod;
	}
	return {a1,m1};
}
int32_t main() {
	vector<ll> A,M;
	pair<ll,ll> ans=CRT(A,M);
}
\end{lstlisting}

\subsection*{Catalan Number}
{ \footnotesize
\textbf{Description:} $C_n = \frac{1}{n+1}\binom{2n}{n}$. Counts valid parenthesis sequences, binary trees, polygon triangulations, etc. \\
\textbf{Time:} $\mathcal{O}(N)$.
}
\begin{lstlisting}
ll dp[M], fac[2 * M];
void fact() {
  fac[0] = 1;
  for (int i = 1; i < 2 * M; i++)
    fac[i] = (fac[i - 1] * i) % mod;
}
void cal() {                    /// O(n*logn)
  dp[0] = dp[1] = 1; /// x = (2*x)!/((x+1)!*x!)
  for (int i = 2; i < M; i++)
    dp[i] = 
    (fac[2*i]*bigmod((fac[i+1]*fac[i])%mod,mod-2,mod))%mod;
}
\end{lstlisting}

\subsection*{Custom Bitset (Dynamic)}
\begin{lstlisting}
// Compact, fast bitset wrapper using uint64_t blocks.
// - b : number of bits the bitset represents (logical length).
// - n : number of uint64_t words used = ceil(b / 64).
// - bits : underlying storage; bits[0] stores bits [0..63], bits[1] -> [64..127], etc.
//
// Notes:
// - Indexing and public methods use 0-based bit indices in range [0, b).
// - _clean() masks off unused high bits in the last word so count()/find_first() behave correctly.
// - left_shift/right_shift implement block+intra-block shifts using OR to accumulate results
//   (your implementation performs |= shifts; if you want pure shift (assignment) semantics,
//   you would need to zero the target before ORing).
struct Cool_Bitset {
  vector<uint64_t> bits;   // storage
  int64_t b, n;            // b = number of bits, n = number of 64-bit words
  // ctor: optional initial bit length
  Cool_Bitset(int64_t _b = 0) {
      init(_b);
  }
  // initialize to hold _b bits (all cleared)
  void init(int64_t _b) {
    b = _b;
    n = (b + 63) / 64;          // number of 64-bit words required
    bits.assign(n, 0);          // zero-initialize
  }
  // completely free storage
  void clear() {
    b = n = 0;
    bits.clear();
  }
  // reset contents to zero but keep size
  void reset() {
    bits.assign(n, 0);
  }
  // mask out unused high bits in the last word (if b is not a multiple of 64).
  // This ensures operations like count() and find_first() don't see garbage bits past 'b'.
  void _clean() {
    if (b != 64 * n) {
      // compute number of valid bits in last word and mask others off
      bits.back() &= (1ULL << (b - 64 * (n - 1))) - 1;
    }
  }
  // read bit at index (0-based). Returns 0/1.
  bool get(int64_t index) const {
    // no bounds check here for speed; caller should ensure 0 <= index < b
    return (bits[index / 64] >> (index % 64)) & 1ULL;
  }
  // write bit at index to 'value' (true => 1, false => 0)
  void set(int64_t index, bool value) {
    assert(0 <= index && index < b);             // debug-only check
    int64_t word = index / 64;
    int shift = index % 64;
    // clear the target bit then set accordingly
    bits[word] &= ~(1ULL << shift);
    bits[word] |= (uint64_t(value) << shift);
  }
  // LEFT shift by 'shift' bits (logical shift). Implementation uses |= so it accumulates bits.
  // Complexity: O(n)
  void left_shift(int64_t shift) {
    int64_t div = shift / 64;    // whole-word shift
    int64_t mod = shift % 64;    // intra-word shift
    if (mod == 0) {
      // shift by whole words: move words upward
      for (int64_t i = n - 1; i >= div; i--)
        bits[i] |= bits[i - div];
      // note: words [0..div-1] are unchanged (ORed with 0)
      return;
    }
    // shift with both whole-word and bit offset
    for (int64_t i = n - 1; i >= div + 1; i--) {
      // combine higher-part and lower-part of source words
      bits[i] |= (bits[i - (div + 1)] >> (64 - mod)) | (bits[i - div] << mod);
    }
    // handle the boundary word (if any)
    if (div < n)
      bits[div] |= bits[0] << mod;
    _clean(); // ensure we didn't set bits past 'b'
  }
  // RIGHT shift by 'shift' bits (logical). Implementation uses |= so it accumulates bits.
  // Complexity: O(n)
  void right_shift(int64_t shift) {
    int64_t div = shift / 64;
    int64_t mod = shift % 64;
    if (mod == 0) {
      for (int64_t i = div; i < n; i++)
        bits[i - div] |= bits[i];
      return;
    }
    for (int64_t i = 0; i < n - (div + 1); i++)
      bits[i] |= (bits[i + (div + 1)] << (64 - mod)) | (bits[i + div] >> mod);
    if (div < n)
      bits[n - div - 1] |= bits[n - 1] >> mod;
    _clean();
  }
  // population count (number of set bits). Uses builtin popcountll on each word.
  int64_t count() const {
    int64_t res = 0;
    for (int64_t i = 0; i < n; i++)
      res += __builtin_popcountll(bits[i]);
    return res;
  }
  // find index of first set bit (lowest index). Returns -1 if none.
  // Complexity: O(n) in worst case, but fast because it scans word-by-word and uses ctz.
  int64_t find_first() const {
      for (int64_t i = 0; i < n; i++)
        if (bits[i] != 0)
          return 64 * i + __builtin_ctzll(bits[i]); // ctz: count trailing zeros
      return -1;
  }
  // find next set bit strictly after x (i.e., search from x+1).
  // Safety: original loop could read past 'b', so we added a guard that stops at 'b'.
  // Returns -1 if none.
  int64_t find_next(int64_t x) const {
    // first scan in the same word (from x+1 up to end of that word)
    int64_t start = x + 1;
    if (start < b) {
      int64_t end_same_word = min<int64_t>( (x / 64) * 64 + 64, b ); // exclusive bound
      for (int64_t i = start; i < end_same_word; ++i) {
          if (get(i)) return i;
      }
    }
    // then scan entire following words
    for (int64_t i = x / 64 + 1; i < n; i++)
      if (bits[i] != 0)
        return 64 * i + __builtin_ctzll(bits[i]);

    return -1;
  }
  // in-place AND with another bitset (must be same size)
  Cool_Bitset& operator&=(const Cool_Bitset &other) {
    assert(b == other.b);
    for (int64_t i = 0; i < n; i++)
        bits[i] &= other.bits[i];
    return *this;
  }
  // return new bitset = this & other
  Cool_Bitset operator&(const Cool_Bitset &other) const {
    assert(b == other.b);
    Cool_Bitset res(b);
    for (int64_t i = 0; i < n; i++) res.bits[i] = bits[i] & other.bits[i];
    return res;
  }
};
\end{lstlisting}

\subsection*{XOR Basis}
\begin{lstlisting}
const int MX=301;
struct bigxorBasis {
	bitset<MX> basis[MX];
	bool has_basis[MX];
	int sz;
	bigxorBasis() {
		for(int i=0;i<MX;i++) has_basis[i]=false;
		sz=0;
	}
	void insert(bitset<MX> mask) {
		for(int i=MX-1;i>=0;i--) {
			if(!mask.test(i)) continue;
			if(!has_basis[i]) {
				basis[i]=mask;
				has_basis[i]=true;
				sz++;
				return;
			}
			mask^=basis[i];
		}
	}
	ll zeros(ll n) {
		return (n-sz);
	}
};

const int LOG_K=64;
struct xorBasis {
	ll basis[LOG_K];
	int sz;
	bool dirty;
	xorBasis() {
		fill(basis,basis+LOG_K,0);
		sz=0;
	}
	bool insert(ll x) {
		for(int i=LOG_K-1;i>=0;i--) {
			if(!(x&(1LL<<i))) continue;
			if(!basis[i]) {
				basis[i]=x;
				sz++;
				dirty=true;
				return true;
			}
			x^=basis[i];
		}
		return false; // it means x got 0 and it is makeable by others
	}
	void RREF() { // Reduced row echekon form 
		for(int i=LOG_K-1;i>=0;i--) {
			if(basis[i]) {
				for(int j=i-1;j>=0;j--) {
					if(basis[j] && basis[i]&(1LL<<j))
						basis[i]^=basis[j];
				}
			}
		}
	}
	ll unique(ll n) {
		return (1LL<<sz);
	}
	ll how_many_can_make(ll n) {
		return (1LL<<(n-sz));
		//return n-sz;
	}
	ll can_make_x(ll x) {
		for(int i=LOG_K-1;i>=0;i--) {
			if(x&(1LL<<i)) x^=basis[i];
		}
		if(!x) return 1;
		else return 0;
	}
	ll kth(ll k) {
		if(dirty) RREF();
		vector<ll> v;
		for(int i=0;i<LOG_K;i++) if(basis[i]) v.PB(basis[i]);
		if((1LL<<sz)<k) return -1;
		k--;
		ll ans=0;
		for(int i=0;i<LOG_K;i++) {
			if(k&(1LL<<i)) ans^=v[i];
		} 
		return ans;
	}
	ll max() {
		RREF();
		ll ans=0;
		for(int i=0;i<LOG_K;i++) ans^=basis[i];
		return ans;
	}
};
\end{lstlisting}
\subsection*{XOR Basis range query}
\begin{lstlisting}
const int LOG_K=60;
struct xorBasis {
	ll basis[LOG_K];
	ll pos[LOG_K];
	int sz;
	bool dirty;
	xorBasis() {
		fill(basis,basis+LOG_K,0);
		fill(pos,pos+LOG_K,0);
		sz=0;
	}
	bool insert(ll x,ll ind) {
		for(int i=LOG_K-1;i>=0;i--) {
			if(!(x&(1LL<<i))) continue;
			if(!basis[i]) {
				basis[i]=x;
				sz++;
				pos[i]=ind;
				//dirty=true;
				return true;
			}
			if(pos[i]<ind) {
				swap(basis[i],x);
				swap(pos[i],ind);
			}
			x^=basis[i];
		}
		return false; // it means x got 0 and it is makeable by others
	}
	//int MAX(int L) {
		//int ans=0;
		//for (int i = LOG_K - 1; i >= 0; i--) {
			//if(pos[i]>=L) {
				//ans=max(ans,basis[i]^ans);
			//}
		//}
		//return ans;
	//}
	ll can_make_x(ll x,ll L) {
		for(int i=LOG_K-1;i>=0;i--) {
			if(pos[i]>=L)
				if(x&(1LL<<i)) x^=basis[i];
		}
		return (x==0);
	}
};
vector<xorBasis> prefix_basis;
void GLITCH_() {
	int n; cin>>n;
	prefix_basis.resize(n+1);
	for(int i=1;i<=n;i++) {
		ll val; cin>>val;
		prefix_basis[i]=prefix_basis[i-1];
		prefix_basis[i].insert(val,i);
	}
	int q; cin>>q;
	for(int i=1;i<=q;i++) {
		int l,r; cin>>l>>r;
		ll x; cin>>x;
		if(prefix_basis[r].can_make_x(x,l)) ha();
		else na();
		//cout<<prefix_basis[r].can_make_x(x,l)<<endl;
	}
}
\end{lstlisting}
\subsection*{XOR Basis subset print}
\begin{lstlisting}
const int LOG_K=60;

struct Filter {
	ll basis[LOG_K];
	Filter() {
		fill(basis,basis+LOG_K,0);
	}
	bool insert(ll val) {
		for(int i=LOG_K-1;i>=0;i--) {
			if(!(val & (1LL<<i))) continue;
			if(!basis[i]) {
				basis[i]=val;
				return true;
			}
			val^=basis[i];
		} 
		return false;
	}
};

struct construct {
	ll basis[LOG_K];
	ll mask[LOG_K];
	construct() {
		fill(basis,basis+LOG_K,0);
		fill(mask,mask+LOG_K,0);
	}
	void insert(ll val,int pivort) {
		ll current_mask=(1LL<<pivort);
		for(int i=LOG_K-1;i>=0;i--) {
			if(!(val & (1LL<<i))) continue;
			if(!basis[i]) {
				basis[i]=val;
				mask[i]=current_mask;
				return;
			}
			val^=basis[i];
			current_mask^=mask[i];
		}
	}
	ll get_mask(ll terget) {
		ll ans_mask=0;
		for(int i=LOG_K-1;i>=0;i--) {
			if((terget>>i) & 1) {
				if(!basis[i]) return -1; // 
				terget^=basis[i];
				ans_mask^=mask[i];
			}
		}
		return ans_mask;
	}
};
void GLITCH_() {
	Filter filter;
	construct solver;
	int n; cin>>n;
	vector<int> ind;
	int cnt=0;
	for(int i=1;i<=n;i++) {
		ll val;
		cin>>val;
		if(filter.insert(val)) {
			solver.insert(val,cnt);
			cnt++;
			ind.PB(i);
		}
	}
	int q; cin>>q;
	while(q--) {
		ll x; cin>>x;
		ll used_mask=solver.get_mask(x);
		ll ans[n+1] {};
		for(int i=0;i<ind.size();i++) {
			if((used_mask>>i) & 1) {
				ans[ind[i]]=1;
			}
		}
		for(int i=1;i<=n;i++) cout<<ans[i]; cout<<endl;
	}
}
\end{lstlisting}

\subsection*{Matrix Multiplication \& Matrix Exponentiation}
\begin{lstlisting}
const int MOD = 1e9 + 7; const int SZ = 2;
struct Matrix {
    long long mat[SZ][SZ];
    Matrix() { memset(mat, 0, sizeof(mat)); }
    static Matrix identity() {
        Matrix res;
        for (int i = 0; i < SZ; i++)
            res.mat[i][i] = 1;
        return res;
    }
    Matrix operator*(const Matrix& other) const { // Matrix Mul: A * B
        Matrix res;
        for (int i = 0; i < SZ; i++) {
            for (int k = 0; k < SZ; k++) {
                if (mat[i][k] == 0) continue; 
                for (int j = 0; j < SZ; j++) {
                    res.mat[i][j] = (res.mat[i][j] + mat[i][k] * other.mat[k][j]) % MOD;
                }
            }
        }
        return res;
    }
};
Matrix power(Matrix a, long long p) {
    Matrix res = Matrix::identity();
    while (p > 0) {
        if (p & 1) res = res * a;
        a = a * a; p >>= 1;
    }
    return res;
}
int main() {
    long long n; cin >> n;
    if (n == 0) {
        cout << 0 << endl; return 0;
    }
    Matrix T;
    T.mat[0][0] = 1; T.mat[0][1] = 1;
    T.mat[1][0] = 1; T.mat[1][1] = 0;
    T = power(T, n - 1);
// The answer is T[0][0] * F(1) + T[0][1]*F(0)
//Since F(1)=1 and F(0)=0, answer is just T[0][0]
    cout << T.mat[0][0] << endl;
}
\end{lstlisting}

\section*{Geo Template}
% \vspace{-.4cm}
\begin{lstlisting}

const int N = 3e5 + 9;

const double inf = 1e100;
const double eps = 1e-9;
const double PI = acos((double)-1.0);
int sign(double x) { return (x > eps) - (x < -eps); }
struct PT {
    double x, y;
    PT() { x = 0, y = 0; }
    PT(double x, double y) : x(x), y(y) {}
    PT(const PT &p) : x(p.x), y(p.y)    {}
    PT operator + (const PT &a) const { return PT(x + a.x, y + a.y); }
    PT operator - (const PT &a) const { return PT(x - a.x, y - a.y); }
    PT operator * (const double a) const { return PT(x * a, y * a); }
    friend PT operator * (const double &a, const PT &b) { return PT(a * b.x, a * b.y); }
    PT operator / (const double a) const { return PT(x / a, y / a); }
    bool operator == (PT a) const { return sign(a.x - x) == 0 && sign(a.y - y) == 0; }
    bool operator != (PT a) const { return !(*this == a); }
    bool operator < (PT a) const { return sign(a.x - x) == 0 ? y < a.y : x < a.x; }
    bool operator > (PT a) const { return sign(a.x - x) == 0 ? y > a.y : x > a.x; }
    double norm() { return sqrt(x * x + y * y); }
    double norm2() { return x * x + y * y; }
    PT perp() { return PT(-y, x); }
    double arg() { return atan2(y, x); }
    PT truncate(double r) { // returns a vector with norm r and having same direction
        double k = norm();
        if (!sign(k)) return *this;
        r /= k;
        return PT(x * r, y * r);
    }
};
istream &operator >> (istream &in, PT &p) { return in >> p.x >> p.y; }
ostream &operator << (ostream &out, PT &p) { return out << "(" << p.x << "," << p.y << ")"; }
inline double dot(PT a, PT b) { return a.x * b.x + a.y * b.y; }
inline double dist2(PT a, PT b) { return dot(a - b, a - b); }
inline double dist(PT a, PT b) { return sqrt(dot(a - b, a - b)); }
inline double cross(PT a, PT b) { return a.x * b.y - a.y * b.x; }
inline double cross2(PT a, PT b, PT c) { return cross(b - a, c - a); }
inline int orientation(PT a, PT b, PT c) { return sign(cross(b - a, c - a)); }
PT perp(PT a) { return PT(-a.y, a.x); }
PT rotateccw90(PT a) { return PT(-a.y, a.x); }
PT rotatecw90(PT a) { return PT(a.y, -a.x); }
PT rotateccw(PT a, double t) { return PT(a.x * cos(t) - a.y * sin(t), a.x * sin(t) + a.y * cos(t)); }
PT rotatecw(PT a, double t) { return PT(a.x * cos(t) + a.y * sin(t), -a.x * sin(t) + a.y * cos(t)); }
double SQ(double x) { return x * x; }
double rad_to_deg(double r) { return (r * 180.0 / PI); }
double deg_to_rad(double d) { return (d * PI / 180.0); }
double get_angle(PT a, PT b) {
    double costheta = dot(a, b) / a.norm() / b.norm();
    return acos(max((double)-1.0, min((double)1.0, costheta)));
}
bool is_point_in_angle(PT b, PT a, PT c, PT p) { // does point p lie in angle <bac
    assert(orientation(a, b, c) != 0);
    if (orientation(a, c, b) < 0) swap(b, c);
    return orientation(a, c, p) >= 0 && orientation(a, b, p) <= 0;
}
bool half(PT p) {
    return p.y > 0.0 || (p.y == 0.0 && p.x < 0.0);
}
void polar_sort(vector<PT> &v) { // sort points in counterclockwise
    sort(v.begin(), v.end(), [](PT a,PT b) {
        return make_tuple(half(a), 0.0, a.norm2()) < make_tuple(half(b), cross(a, b), b.norm2());
    });
}
void polar_sort(vector<PT> &v, PT o) { // sort points in counterclockwise with respect to point o
    sort(v.begin(), v.end(), [&](PT a,PT b) {
        return make_tuple(half(a - o), 0.0, (a - o).norm2()) < make_tuple(half(b - o), cross(a - o, b - o), (b - o).norm2());
    });
}
struct line {
    PT a, b; // goes through points a and b
    PT v; double c;  //line form: direction vec [cross] (x, y) = c
    line() {}
    //direction vector v and offset c
	line(PT v, double c) : v(v), c(c) {
        auto p = get_points();
        a = p.first; b = p.second;
	}
	// equation ax + by + c = 0
	line(double _a, double _b, double _c) : v({_b, -_a}), c(-_c) {
		auto p = get_points();
        a = p.first; b = p.second;
	}
	// goes through points p and q
	line(PT p, PT q) : v(q - p), c(cross(v, p)), a(p), b(q) {}
    	pair<PT, PT> get_points() { //extract any two points from this line
		PT p, q; double a = -v.y, b = v.x; // ax + by = c
		if (sign(a) == 0) {
		    p = PT(0, c / b);
		    q = PT(1, c / b);
		}
		else if (sign(b) == 0) {
		    p = PT(c / a, 0);
		    q = PT(c / a, 1);
		}
		else {
		    p = PT(0, c / b);
		    q = PT(1, (c - a) / b);
		}
		return {p, q};
    	}
    // ax + by + c = 0
    array<double, 3> get_abc() {
        double a = -v.y, b = v.x;
        return {a, b, -c};
    }
    // 1 if on the left, -1 if on the right, 0 if on the line
    int side(PT p) { return sign(cross(v, p) - c); }
    // line that is perpendicular to this and goes through point p
    line perpendicular_through(PT p) { return {p, p + perp(v)}; }
    // translate the line by vector t i.e. shifting it by vector t
    line translate(PT t) { return {v, c + cross(v, t)}; }
    // compare two points by their orthogonal projection on this line
    // a projection point comes before another if it comes first according to vector v
    bool cmp_by_projection(PT p, PT q) { return dot(v, p) < dot(v, q); }
	line shift_left(double d) {
		PT z = v.perp().truncate(d);
		return line(a + z, b + z);
	}
};
// find a point from a through b with distance d
PT point_along_line(PT a, PT b, double d) {
    assert(a != b);
    return a + (((b - a) / (b - a).norm()) * d);
}
// projection point c onto line through a and b  assuming a != b
PT project_from_point_to_line(PT a, PT b, PT c) {
    return a + (b - a) * dot(c - a, b - a) / (b - a).norm2();
}
// reflection point c onto line through a and b  assuming a != b
PT reflection_from_point_to_line(PT a, PT b, PT c) {
    PT p = project_from_point_to_line(a,b,c);
    return p + p - c;
}
// minimum distance from point c to line through a and b
double dist_from_point_to_line(PT a, PT b, PT c) {
    return fabs(cross(b - a, c - a) / (b - a).norm());
}
// returns true if  point p is on line segment ab
bool is_point_on_seg(PT a, PT b, PT p) {
    if (fabs(cross(p - b, a - b)) < eps) {
        if (p.x < min(a.x, b.x) - eps || p.x > max(a.x, b.x) + eps) return false;
        if (p.y < min(a.y, b.y) - eps || p.y > max(a.y, b.y) + eps) return false;
        return true;
    }
    return false;
}
// minimum distance point from point c to segment ab that lies on segment ab
PT project_from_point_to_seg(PT a, PT b, PT c) {
    double r = dist2(a, b);
    if (sign(r) == 0) return a;
    r = dot(c - a, b - a) / r;
    if (r < 0) return a;
    if (r > 1) return b;
    return a + (b - a) * r;
}
// minimum distance from point c to segment ab
double dist_from_point_to_seg(PT a, PT b, PT c) {
    return dist(c, project_from_point_to_seg(a, b, c));
}
// 0 if not parallel, 1 if parallel, 2 if collinear
int is_parallel(PT a, PT b, PT c, PT d) {
    double k = fabs(cross(b - a, d - c));
    if (k < eps){
        if (fabs(cross(a - b, a - c)) < eps && fabs(cross(c - d, c - a)) < eps) return 2;
        else return 1;
    }
    else return 0;
}
// check if two lines are same
bool are_lines_same(PT a, PT b, PT c, PT d) {
    if (fabs(cross(a - c, c - d)) < eps && fabs(cross(b - c, c - d)) < eps) return true;
    return false;
}
// bisector vector of <abc
PT angle_bisector(PT &a, PT &b, PT &c){
    PT p = a - b, q = c - b;
    return p + q * sqrt(dot(p, p) / dot(q, q));
}
// 1 if point is ccw to the line, 2 if point is cw to the line, 3 if point is on the line
int point_line_relation(PT a, PT b, PT p) {
    int c = sign(cross(p - a, b - a));
    if (c < 0) return 1;
    if (c > 0) return 2;
    return 3;
}
// intersection point between ab and cd assuming unique intersection exists
bool line_line_intersection(PT a, PT b, PT c, PT d, PT &ans) {
    double a1 = a.y - b.y, b1 = b.x - a.x, c1 = cross(a, b);
    double a2 = c.y - d.y, b2 = d.x - c.x, c2 = cross(c, d);
    double det = a1 * b2 - a2 * b1;
    if (det == 0) return 0;
    ans = PT((b1 * c2 - b2 * c1) / det, (c1 * a2 - a1 * c2) / det);
    return 1;
}
// intersection point between segment ab and segment cd assuming unique intersection exists
bool seg_seg_intersection(PT a, PT b, PT c, PT d, PT &ans) {
    double oa = cross2(c, d, a), ob = cross2(c, d, b);
    double oc = cross2(a, b, c), od = cross2(a, b, d);
    if (oa * ob < 0 && oc * od < 0){
        ans = (a * ob - b * oa) / (ob - oa);
        return 1;
    }
    else return 0;
}
// intersection point between segment ab and segment cd assuming unique intersection may not exists
// se.size()==0 means no intersection
// se.size()==1 means one intersection
// se.size()==2 means range intersection
set<PT> seg_seg_intersection_inside(PT a,  PT b,  PT c,  PT d) {
    PT ans;
    if (seg_seg_intersection(a, b, c, d, ans)) return {ans};
    set<PT> se;
    if (is_point_on_seg(c, d, a)) se.insert(a);
    if (is_point_on_seg(c, d, b)) se.insert(b);
    if (is_point_on_seg(a, b, c)) se.insert(c);
    if (is_point_on_seg(a, b, d)) se.insert(d);
    return se;
}
// intersection  between segment ab and line cd
// 0 if do not intersect, 1 if proper intersect, 2 if segment intersect
int seg_line_relation(PT a, PT b, PT c, PT d) {
    double p = cross2(c, d, a);
    double q = cross2(c, d, b);
    if (sign(p) == 0 && sign(q) == 0) return 2;
    else if (p * q < 0) return 1;
    else return 0;
}
// intersection between segament ab and line cd assuming unique intersection exists
bool seg_line_intersection(PT a, PT b, PT c, PT d, PT &ans) {
    bool k = seg_line_relation(a, b, c, d);
    assert(k != 2);
    if (k) line_line_intersection(a, b, c, d, ans);
    return k;
}
// minimum distance from segment ab to segment cd
double dist_from_seg_to_seg(PT a, PT b, PT c, PT d) {
    PT dummy;
    if (seg_seg_intersection(a, b, c, d, dummy)) return 0.0;
    else return min({dist_from_point_to_seg(a, b, c), dist_from_point_to_seg(a, b, d),
        dist_from_point_to_seg(c, d, a), dist_from_point_to_seg(c, d, b)});
}
// minimum distance from point c to ray (starting point a and direction vector b)
double dist_from_point_to_ray(PT a, PT b, PT c) {
    b = a + b;
    double r = dot(c - a, b - a);
    if (r < 0.0) return dist(c, a);
    return dist_from_point_to_line(a, b, c);
}
// starting point as and direction vector ad
bool ray_ray_intersection(PT as, PT ad, PT bs, PT bd) {
    double dx = bs.x - as.x, dy = bs.y - as.y;
    double det = bd.x * ad.y - bd.y * ad.x;
    if (fabs(det) < eps) return 0;
    double u = (dy * bd.x - dx * bd.y) / det;
    double v = (dy * ad.x - dx * ad.y) / det;
    if (sign(u) >= 0 && sign(v) >= 0) return 1;
    else return 0;
}
double ray_ray_distance(PT as, PT ad, PT bs, PT bd) {
    if (ray_ray_intersection(as, ad, bs, bd)) return 0.0;
    double ans = dist_from_point_to_ray(as, ad, bs);
    ans = min(ans, dist_from_point_to_ray(bs, bd, as));
    return ans;
}
struct circle {
    PT p; double r;
    circle() {}
    circle(PT _p, double _r): p(_p), r(_r) {};
    // center (x, y) and radius r
    circle(double x, double y, double _r): p(PT(x, y)), r(_r) {};
    // circumcircle of a triangle
    // the three points must be unique
    circle(PT a, PT b, PT c) {
        b = (a + b) * 0.5;
        c = (a + c) * 0.5;
        line_line_intersection(b, b + rotatecw90(a - b), c, c + rotatecw90(a - c), p);
        r = dist(a, p);
    }
    // inscribed circle of a triangle
    // pass a bool just to differentiate from circumcircle
    circle(PT a, PT b, PT c, bool t) {
        line u, v;
        double m = atan2(b.y - a.y, b.x - a.x), n = atan2(c.y - a.y, c.x - a.x);
        u.a = a;
        u.b = u.a + (PT(cos((n + m)/2.0), sin((n + m)/2.0)));
        v.a = b;
        m = atan2(a.y - b.y, a.x - b.x), n = atan2(c.y - b.y, c.x - b.x);
        v.b = v.a + (PT(cos((n + m)/2.0), sin((n + m)/2.0)));
        line_line_intersection(u.a, u.b, v.a, v.b, p);
        r = dist_from_point_to_seg(a, b, p);
    }
    bool operator == (circle v) { return p == v.p && sign(r - v.r) == 0; }
    double area() { return PI * r * r; }
    double circumference() { return 2.0 * PI * r; }
};
//0 if outside, 1 if on circumference, 2 if inside circle
int circle_point_relation(PT p, double r, PT b) {
    double d = dist(p, b);
    if (sign(d - r) < 0) return 2;
    if (sign(d - r) == 0) return 1;
    return 0;
}
// 0 if outside, 1 if on circumference, 2 if inside circle
int circle_line_relation(PT p, double r, PT a, PT b) {
    double d = dist_from_point_to_line(a, b, p);
    if (sign(d - r) < 0) return 2;
    if (sign(d - r) == 0) return 1;
    return 0;
}
//compute intersection of line through points a and b with
//circle centered at c with radius r > 0
vector<PT> circle_line_intersection(PT c, double r, PT a, PT b) {
    vector<PT> ret;
    b = b - a; a = a - c;
    double A = dot(b, b), B = dot(a, b);
    double C = dot(a, a) - r * r, D = B * B - A * C;
    if (D < -eps) return ret;
    ret.push_back(c + a + b * (-B + sqrt(D + eps)) / A);
    if (D > eps) ret.push_back(c + a + b * (-B - sqrt(D)) / A);
    return ret;
}
//5 - outside and do not intersect
//4 - intersect outside in one point
//3 - intersect in 2 points
//2 - intersect inside in one point
//1 - inside and do not intersect
int circle_circle_relation(PT a, double r, PT b, double R) {
    double d = dist(a, b);
    if (sign(d - r - R) > 0)  return 5;
    if (sign(d - r - R) == 0) return 4;
    double l = fabs(r - R);
    if (sign(d - r - R) < 0 && sign(d - l) > 0) return 3;
    if (sign(d - l) == 0) return 2;
    if (sign(d - l) < 0) return 1;
    assert(0); return -1;
}
// returns area of intersection between two circles
double circle_circle_area(PT a, double r1, PT b, double r2) {
    double d = (a - b).norm();
    if(r1 + r2 < d + eps) return 0;
    if(r1 + d < r2 + eps) return PI * r1 * r1;
    if(r2 + d < r1 + eps) return PI * r2 * r2;
    double theta_1 = acos((r1 * r1 + d * d - r2 * r2) / (2 * r1 * d)),
    	theta_2 = acos((r2 * r2 + d * d - r1 * r1)/(2 * r2 * d));
    return r1 * r1 * (theta_1 - sin(2 * theta_1)/2.) + r2 * r2 * (theta_2 - sin(2 * theta_2)/2.);
}
vector<PT> convex_hull(vector<PT> &p) {
	if (p.size() <= 1) return p;
	vector<PT> v = p;
    sort(v.begin(), v.end());
    vector<PT> up, dn;
    for (auto& p : v) {
        while (up.size() > 1 && orientation(up[up.size() - 2], up.back(), p) >= 0) {
            up.pop_back();
        }
        while (dn.size() > 1 && orientation(dn[dn.size() - 2], dn.back(), p) <= 0) {
            dn.pop_back();
        }
        up.push_back(p);
        dn.push_back(p);
    }
    v = dn;
    if (v.size() > 1) v.pop_back();
    reverse(up.begin(), up.end());
    up.pop_back();
    for (auto& p : up) {
        v.push_back(p);
    }
    if (v.size() == 2 && v[0] == v[1]) v.pop_back();
    return v;
}
 //checks if convex or not
bool is_convex(vector<PT> &p) {
    bool s[3]; s[0] = s[1] = s[2] = 0;
    int n = p.size();
    for (int i = 0; i < n; i++) {
        int j = (i + 1) % n;
        int k = (j + 1) % n;
        s[sign(cross(p[j] - p[i], p[k] - p[i])) + 1] = 1;
        if (s[0] && s[2]) return 0;
    }
    return 1;
}
// -1 if strictly inside, 0 if on the polygon, 1 if strictly outside
// it must be strictly convex, otherwise make it strictly convex first
int is_point_in_convex(vector<PT> &p, const PT& x) { // O(log n)
    int n = p.size(); assert(n >= 3);
    int a = orientation(p[0], p[1], x), b = orientation(p[0], p[n - 1], x);
    if (a < 0 || b > 0) return 1;
    int l = 1, r = n - 1;
    while (l + 1 < r) {
        int mid = l + r >> 1;
        if (orientation(p[0], p[mid], x) >= 0) l = mid;
        else r = mid;
    }
    int k = orientation(p[l], p[r], x);
    if (k <= 0) return -k;
    if (l == 1 && a == 0) return 0;
    if (r == n - 1 && b == 0) return 0;
    return -1;
}
\end{lstlisting}
\subsection*{Closest Pair of Points}
{ \footnotesize
\textbf{Description:} Finds the minimum distance between any two points in a set.
\begin{itemize}[leftmargin=*, nosep]
    \item \textbf{Algorithm:} Divide \& Conquer.
    \item \textbf{Logic:}
    1. Sort points by X-coordinate.
    2. Divide into left/right halves. Recurse to find $d = \min(d_L, d_R)$.
    3. \textbf{Merge Step:} The closest pair might span the dividing line. Gather points within distance $d$ of the middle X-line into a "strip".
    4. Sort strip by Y-coordinate. For each point, check neighbors in the strip. (Geometry guarantees we only need to check the next $\approx 7$ points).
    \item \textbf{Time:} $\mathcal{O}(N \log N)$ (if we merge-sort by Y during recursion) or $\mathcal{O}(N \log^2 N)$ (if we sort strip explicitly). The code below uses \texttt{inplace\_merge} for $\mathcal{O}(N \log N)$.
\end{itemize}
}
\begin{lstlisting}
// Auxiliary function for recursion
ld closestPairRec(vector<P>& pts, int l, int r, vector<P>& aux) {
    if (r - l <= 3) {
        ld best = numeric_limits<ld>::infinity();
        for (int i = l; i < r; ++i)
            for (int j = i+1; j < r; ++j) best = min(best, dist(pts[i], pts[j]));
        // Sort by Y for the merge step
        sort(pts.begin()+l, pts.begin()+r, [](const P& a, const P& b){ return a.y < b.y; });
        return best;
    }
    int m = (l + r) >> 1;
    ld midx = pts[m].x;
    ld d = min(closestPairRec(pts, l, m, aux), closestPairRec(pts, m, r, aux));
    
    // Merge both sorted halves by Y-coordinate
    inplace_merge(pts.begin()+l, pts.begin()+m, pts.begin()+r, 
                  [](const P& a, const P& b){ return a.y < b.y; });
    
    // Create strip: only keep points within 'd' horizontal distance from midx
    int sz = 0;
    for (int i = l; i < r; ++i) {
        if (fabsl(pts[i].x - midx) < d + EPS) aux[sz++] = pts[i];
    }
    // Check points in strip against their neighbors (within vertical distance d)
    for (int i = 0; i < sz; ++i) {
        for (int j = i+1; j < sz && (aux[j].y - aux[i].y) < d + EPS; ++j) {
            d = min(d, dist(aux[i], aux[j]));
        }
    }
    return d;
}
inline ld closestPair(vector<P> pts) {
    sort(pts.begin(), pts.end(), point_cmp); // Sort by X initially
    vector<P> aux(pts.size());
    return closestPairRec(pts, 0, pts.size(), aux);
}
\end{lstlisting}
\section*{DP}
\subsection*{Binary Optimization}
{ \footnotesize
\textbf{Description:} Solves Bounded Knapsack (limited count of items) by decomposing counts into powers of 2 ($1, 2, 4, \dots, rem$). Turns $\mathcal{O}(W \cdot \text{count})$ into $\mathcal{O}(W \cdot \log(\text{count}))$. \\
\textbf{Time:} $\mathcal{O}(W \cdot \sum \log(\text{count}))$.
}
\begin{lstlisting}
map<int, int> mp;
for (auto it : vec)
  mp[it]++;
vector<int> dp(n + 1, 1e9);
dp[0] = 0;
for (auto [w, cnt] : mp) {
  int cur = 1;
  while (cnt > 0) {
    int use = min(cnt, cur);
    for (int i = n; i >= w * use; i--) {
      dp[i] = min(dp[i], dp[i - w * use] + use);
    }
    cnt -= use;
    cur *= 2;
  }
}
\end{lstlisting}

\section*{Mathematics}

% --- SECTION: Equations ---
\subsection*{Equations}
{ \footnotesize
The extremum of a quadratic is given by $x = -b/2a$.

\textbf{Cramer's Rule}: Given an equation $Ax = b$, the solution to a variable $x_i$ is given by
\begin{flalign*}
x_i &= \frac{\det A_i'}{\det A} &&  
   \text{\parbox{2.5cm}{\tiny [where $A_i'$ is $A$ with the $i$'th column replaced by $b$.]}}
\end{flalign*}
\textbf{Example (3x3):}
{ \footnotesize
\begin{align*}
    2x + 3y - 5z &= 1 \\
    x + y - z &= 2 \\
    2y + z &= 8
\end{align*}
$D = \begin{vmatrix} 2 & 3 & -5 \\ 1 & 1 & -1 \\ 0 & 2 & 1 \end{vmatrix} = -7$
$D_x = \begin{vmatrix} 1 & 3 & -5 \\ 2 & 1 & -1 \\ 8 & 2 & 1 \end{vmatrix} = -7$
$D_y = \begin{vmatrix} 2 & 1 & -5 \\ 1 & 2 & -1 \\ 0 & 8 & 1 \end{vmatrix} = -21$
$D_z = \begin{vmatrix} 2 & 3 & 1 \\ 1 & 1 & 2 \\ 0 & 2 & 8 \end{vmatrix} = 14$
$x = \frac{D_x}{D} = 1$, $y = \frac{D_y}{D} = 3$, $z = \frac{D_z}{D} = -2$
}

\textbf{Vieta's Formulas}: Let $P(x) = a_nx^n +...+a_0$, be a polynomial with complex coefficients and degree $n$, having complex roots $r_n,...,r_1$. Then for any integer $0 \leq k \leq n$,
\[
  \sum_{1 \leq i_1 < i_2 < ... < i_k \leq n} r_{i_1}r_{i_2}...r_{i_k} = (-1)^k\frac{a_{n-k}}{a_n}
\]
\textbf{Rational Root Theorem}: If $\frac{p}{q}$ is a reduced rational root of a polynomial with \textbf{integer coeffs}, then $p \mid a_0$ and $q \mid a_n$.
}
% --- SECTION: Number Theory ---
\subsection*{Number Theory}
{\footnotesize
\textbf{Sum of Divisors (S.O.D):}
If $N = a^p \cdot b^q \cdot c^r \dots$
\[ \text{S.O.D} = \frac{a^{p+1}-1}{a-1} \cdot \frac{b^{q+1}-1}{b-1} \cdot \frac{c^{r+1}-1}{c-1} \dots \]

\textbf{Number of Divisors (N.O.D):}
If $N = a^p \cdot b^q \cdot c^r \dots$
\[ \text{N.O.D} = (p+1)(q+1)(r+1) \dots \]

% --- NEW: Product of Divisors ---
\textbf{Product of Divisors (P.O.D):}
If $N$ has $D = \text{N.O.D}(N)$ divisors:
\[ \text{P.O.D}(N) = N^{D/2} = (\sqrt{N})^D \]

\textbf{Euclidean Algorithm Property:}
\[ \gcd(a, b) = \gcd(a, a - b) \quad [a > b] \]

\textbf{Fibonacci GCD:}
\[ \gcd(F(a), F(b)) = F(\gcd(a,b)) \]

\textbf{Euler's Totient Theorem:}
\[ a^{\phi(n)} \equiv 1 \pmod{n} \]
where $\phi(n)$ is Euler's Totient Function.

\textbf{Modular Exponentiation:}
\[ a^b \pmod m \equiv a^{b \pmod{\phi(m)}} \pmod m \]
(if $a$ and $m$ are coprime)

\textbf{Primitive roots} modulo $n$ exists iff $n = 1, 2, 4$ or, $n = p^k, 2p^k$ where $p$ is an odd prime. Furthermore, the number of roots are $\phi (\phi (n))$.

\textbf{To Find Generator} $g$ of $M$, factor $M - 1$ and get the distinct primes $p_i$. If $g^{(M - 1) / p_i } \neq 1 (MOD M)$ for each $p_i$ then $g$ is a valid root. Try all $g$ until a hit is found (usually found very quick).
}
{ \footnotesize
\begin{itemize}[leftmargin=*, nosep]
    % --- Divisibility & GCD ---
    \item \textbf{Euclidean Step:} For $i > j$, $\gcd(i, j) = \gcd(i-j, j) \le (i-j)$
    \item \textbf{Lattice Points:} Points on segment $(x_1, y_1)$ to $(x_2, y_2)$ is $\gcd(|x_1-x_2|, |y_1-y_2|) + 1$
    \item \textbf{Power Divisibility:} Count $x \le n$ such that $d | x^k$:
    \[ \sum_{x=1}^n [d | x^k] = \left\lfloor \frac{n}{\prod p_i^{\lceil e_i/k \rceil}} \right\rfloor \quad \text{where } d = \prod p_i^{e_i} \]

    % --- Divisor Counts & Sums ---
    \item \textbf{Odd Divisor Count:} $d(n)$ is odd $\iff n$ is a perfect square.
    \item \textbf{Odd Divisor Sum:} $\sigma(n)$ is odd $\iff n = 2^r k^2$ ($n$ is square or twice a square).
    \item \textbf{Triple Divisor Sum:} Sum of $d(ijk)$ for $i \le A, j \le B, k \le C$:
    \[ \sum_{i,j,k} d(ijk) = \sum_{\gcd(i,j)=\gcd(j,k)=\gcd(k,i)=1} \lfloor \frac{A}{i} \rfloor \lfloor \frac{B}{j} \rfloor \lfloor \frac{C}{k} \rfloor \]

    % --- Modular Arithmetic (Wilson's & Factorials) ---
    \item \textbf{Factorial Modulo $n$:} $(n-1)! \pmod n = \begin{cases} n-1 & n \text{ is prime} \\ 2 & n=4 \\ 0 & \text{otherwise} \end{cases}$
    \item \textbf{Generalized Wilson:} Product of integers coprime to $m$ modulo $m$:
    \[ \prod_{\substack{1 \le k < m \\ \gcd(k,m)=1}} k \equiv \begin{cases} -1 & m = 4, p^\alpha, 2p^\alpha \\ 1 & \text{otherwise} \end{cases} \pmod m \]
    \hfill \textit{($-1$ iff primitive root exists)}

    % --- Linear Diophantine ---
    \item \textbf{Linear Representations:} Number of solutions to $n = ax + by$ ($x, y \ge 0$, $\gcd(a,b)=1$):
    \[ \frac{n}{ab} - \left\{ \frac{b' n}{a} \right\} - \left\{ \frac{a' n}{b} \right\} + 1 \]
    \textit{($\{x\} =$ fractional part. $a', b'$ are inverses: $aa' \equiv 1 \pmod b$, $bb' \equiv 1 \pmod a$)}
\end{itemize}
}

{ \footnotesize
\textbf{Description:} Properties of $F_0=0, F_1=1, F_n=F_{n-1}+F_{n-2}$. Key for tiling, GCD, and modular periodicity.
\begin{itemize}[leftmargin=*, nosep]
    % --- Closed Forms & Sums ---
    \item \textbf{Binet's Formula:} $F_n = \frac{1}{\sqrt{5}}\left( (\frac{1+\sqrt{5}}{2})^n - (\frac{1-\sqrt{5}}{2})^n \right)$ \hfill \textit{(Closed Form)}
    \item \textbf{Combinatorial Sum:} $F_n = \sum_{k=0}^{\lfloor\frac{n-1}{2}\rfloor}\binom{n-k-1}{k}$ \hfill \textit{(Sum of shallow diagonals)}
    \item \textbf{Sum of Odd Indices:} $\sum_{i=0}^{n-1}F_{2i+1} = F_{2n}$

    % --- Addition & Doubling Identities ---
    \item \textbf{Addition Identity:} $F_{m+n} = F_{m-1}F_n + F_m F_{n+1}$
    \item \textbf{Shifted Addition:} $F_{m+n-1} = F_m F_n + F_{m-1} F_{n-1}$
    \item \textbf{Doubling Identity:} $F_{2n} = F_n(F_{n+1} + F_{n-1}) = F_{n+1}^2 - F_{n-1}^2$ \hfill \textit{(Fast doubling)}
    \item \textbf{General Subtraction:} $F_mF_{n+1} - F_{m-1}F_n = (-1)^n F_{m-n}$

    % --- Number Theoretic Properties ---
    \item \textbf{Square Check:} $n$ is Fib $\iff 5n^2+4$ or $5n^2-4$ is a perfect square.
    \item \textbf{Strong Divisibility:} $F_k | F_n \iff k | n$. \hfill \textit{(Every $k^{th}$ Fib is a multiple of $F_k$)}
    \item \textbf{Coprimality:} $\gcd(F_n, F_{n+1}) = 1$. Any 3 consecutive are pairwise coprime.
    \item \textbf{Pisano Period:} Sequence modulo $n$ is periodic with period $\pi(n) \le 6n$.
\end{itemize}
}
% --- SECTION: Mobius Function ---
\subsection*{Mobius Function}
{ \footnotesize
\[
    \mu(n) = \begin{cases} 0 & n \textrm{ is not square free}\\ 1 & n \textrm{ has even number of prime factors}\\ -1 & n \textrm{ has odd number of prime factors}\\\end{cases}
\]
}
 Mobius Inversion:
 \[ g(n) = \sum_{d|n} f(d) \Leftrightarrow f(n) = \sum_{d|n} \mu(d)g(n/d) \]
 Other useful formulas/forms:

 $ \sum_{d | n} \mu(d) = [ n = 1] $,
 $ \phi(n) = \sum_{d | n} \mu(d)\frac{n}{d}$

 $ g(n) = \sum_{n|d} f(d) \Leftrightarrow f(n) = \sum_{n|d} \mu(\frac{d}{n})g(d)$

 $ g(n) = \sum_{1 \leq m \leq n} f(\left\lfloor\frac{n}{m}\right \rfloor ) \Leftrightarrow f(n) = \sum_{1\leq m\leq n} \mu(m)g(\left\lfloor\frac{n}{m}\right\rfloor)$

 If $f$ multiplicative, $\sum_{d | n} \mu(d)f(d) = \prod_{\textnormal{prime } p | n}(1 - f(p))$ and $\sum_{d | n}\mu^2(d)f(d) = \prod_{\textnormal{prime } p | n} (1 + f(p))$. 

 If $s_f(n) = \sum_{i = 1}^{n} f(i)$ is a prefix sum of mulitplicative $f$ then $s_{f * g}(n) = \sum_{1 \leq xy \leq n}f(x)g(y)$. Then $s_f(n) = \{s_{f * g}(n) - \sum_{d = 2}^{n} s_f(\lfloor n / d \rfloor) g(d)\} / g(1)$ where $f * g(n) = \sum_{d | n} f(d)g(n / d)$ (Dirichlet). Precompute (linear sieve) $O(n^{2/3})$ first values of $s_f$ for complexity $O(n^{2/3})$.

 Useful sums and convolutions: $\epsilon = \mu * \textnormal{\textbf{1}}$, id = $\phi * \textnormal{\textbf{1}}$, id = $g * \textnormal{id}_2$, where $\epsilon(n) = [n = 1]$, $\textnormal{\textbf{1}}(n) = 1$, id$(n) = n$, id$_k(n) = n^k$, $g(n) = \sum_{d | n}\mu(d)nd$.\\
 coprime pairs in $[1,n]$ is $\sum_{d = 1}^{n}\mu(d)\lfloor n / d \rfloor ^2$. Sum of GCD pairs in $[1, n]$ is $\sum_{d = 1}^{n}\phi(d)\lfloor n / d \rfloor ^2$. Sum of LCM pairs in $[1, n]$ is $\sum_{d = 1}^{n}(\frac{\lfloor n / d \rfloor (1 + \lfloor n / d \rfloor)}{2})^2 g(d)$, where $g$ is defined above with $g(p^k) = p^k - p^{k+1}$.

\subsection*{GCD and LCM}
{ \footnotesize
\textbf{Description:} Identities for simplifying GCD/LCM sums, counting coprime pairs, and optimizing range queries.

{ \footnotesize
\begin{itemize}[leftmargin=*, nosep]
    % --- Basic Properties ---
    \item \textbf{Euclidean \& Base:} $\gcd(a, b) = \gcd(b, a \pmod b)$, $\quad \gcd(a, 0) = a$
    \item \textbf{Product Relation:} $\gcd(a, b) \cdot \operatorname{lcm}(a, b) = |a \cdot b|$
    \item \textbf{Linear Combination:} $\gcd(a + m \cdot b, b) = \gcd(a, b)$
    \item \textbf{Distributivity:} $\gcd(a, \operatorname{lcm}(b, c)) = \operatorname{lcm}(\gcd(a, b), \gcd(a, c))$
    \item \textbf{GCD of Exponents:} $\gcd(n^a - 1, n^b - 1) = n^{\gcd(a,b)} - 1$
    \item \textbf{Difference Trick:} $\gcd(A_L, \dots, A_R) = \gcd(A_L, A_{L+1}{-}A_L, \dots, A_R{-}A_{R-1})$

    % --- Single Summation Identities ---
    \item \textbf{Gauss' Identity:} $\sum_{d|n} \phi(d) = n \quad \text{and} \quad \gcd(a, b) = \sum_{k|a, k|b} \phi(k)$
    \item \textbf{Sum of LCM(1\dots n, n):} $\sum_{i=1}^n \operatorname{lcm}(i, n) = \frac{n}{2} \left( 1 + \sum_{d|n} d \cdot \phi(d) \right)$
    \item \textbf{Count GCD=k:} $\sum_{i=1}^n [\gcd(i, n) = k] = \phi(n/k)$
    \item \textbf{Sum of GCD:} $\sum_{k=1}^n \gcd(k, n) = \sum_{d|n} d \cdot \phi(n/d)$
    \item \textbf{Power of GCD:} $\sum_{k=1}^n x^{\gcd(k,n)} = \sum_{d|n} x^d \cdot \phi(n/d)$

    % --- Inverse GCD Sums ---
    \item \textbf{Inverse GCD Sum:} $\sum_{k=1}^n \frac{1}{\gcd(k, n)} = \frac{1}{n} \sum_{d|n} d \cdot \phi(d)$
    \item \textbf{Weighted Inverse:} $\sum_{k=1}^n \frac{k}{\gcd(k, n)} = \frac{1}{2} \sum_{d|n} d \cdot \phi(d)$
    \item \textbf{Relation:} $\sum_{k=1}^n \frac{n}{\gcd(k, n)} = 2 \sum_{k=1}^n \frac{k}{\gcd(k, n)} - 1 \quad (n > 1)$

    % --- Double Summation (1 to n) ---
    \item \textbf{Coprime Pairs:} $\sum_{i=1}^n \sum_{j=1}^n [\gcd(i, j) = 1] = \sum_{d=1}^n \mu(d) \lfloor \frac{n}{d} \rfloor^2$
    \item \textbf{Sum of GCD(i, j):} $\sum_{i=1}^n \sum_{j=1}^n \gcd(i, j) = \sum_{d=1}^n \phi(d) \lfloor \frac{n}{d} \rfloor^2$
    \item \textbf{Weighted Coprime:} $\sum_{i=1}^n \sum_{j=1}^n i \cdot j [\gcd(i, j) = 1] = \sum_{i=1}^n \phi(i) i^2$
    \item \textbf{Sum of LCM(i, j):} $\sum_{i=1}^n \sum_{j=1}^n \operatorname{lcm}(i, j) = \sum_{l=1}^n \left( \frac{\lfloor n/l \rfloor (\lfloor n/l \rfloor + 1)}{2} \right)^2 \sum_{d|l} \mu(d)ld$
\end{itemize}
}
}
\subsection*{Euler’s Totient Function $\phi(n)$}
{ \footnotesize
\textbf{Description:} $\phi(n)$ counts positive integers $\le n$ that are relatively prime to $n$. Key for modular arithmetic and GCD counting.

{ \footnotesize
\begin{itemize}[leftmargin=*, nosep]
    % --- Core Computation ---
    \item \textbf{Definition:} $\displaystyle \phi(n) = n \prod_{p|n} \left(1 - \frac{1}{p}\right)$
    \item \textbf{Prime Power:} $\phi(p^k) = p^k - p^{k-1} = p^k(1 - \frac{1}{p})$
    \item \textbf{Multiplicative:} $\phi(mn) = \phi(m)\phi(n)$ \hfill \textit{(If $\gcd(m, n) = 1$)}
    \item \textbf{General Product:} $\phi(mn) = \frac{\phi(m)\phi(n)d}{\phi(d)}$ \hfill \textit{($d = \gcd(m, n)$)}
    \item \textbf{LCM Relation:} $\phi(\operatorname{lcm}(m, n)) \cdot \phi(\gcd(m, n)) = \phi(m) \cdot \phi(n)$
    \item \textbf{Radical Identity:} $\frac{\phi(n)}{n} = \frac{\phi(\operatorname{rad}(n))}{\operatorname{rad}(n)}$ \hfill \textit{($\operatorname{rad}(n) = \prod_{p|n} p$)}

    % --- Summation Identities ---
    \item \textbf{Gauss' Identity:} $\sum_{d|n} \phi(d) = n$
    \item \textbf{Möbius Inversion:} $\phi(n) = \sum_{d|n} \mu(d)\frac{n}{d} = \sum_{d|n} d \cdot \mu(\frac{n}{d})$
    \item \textbf{Sum $\phi \times$ Floor:} $\sum_{i=1}^n \phi(i) \lfloor \frac{n}{i} \rfloor = \frac{n(n+1)}{2}$
    \item \textbf{Sum Odd Indices:} $\sum_{i \text{ odd}}^n \phi(i) \lfloor \frac{n}{i} \rfloor = \sum_{k \ge 1} [ \frac{n}{2^k} ]^2$ \hfill \textit{($[\cdot]$ is round)}
    \item \textbf{Inverse Phi Sum:} $\sum_{d|n} \frac{\mu^2(d)}{\phi(d)} = \frac{n}{\phi(n)}$
    
    % --- Coprime Sums & Counting ---
    \item \textbf{Sum of Coprimes:} $\sum_{k=1, \gcd(k,n)=1}^n k = \frac{1}{2}n\phi(n)$ \hfill \textit{(Avg $= n/2$)}
    \item \textbf{Weighted Double Sum:} $\sum_{i=1}^n \sum_{j=1}^n ij[\gcd(i, j)=1] = \sum_{i=1}^n \phi(i)i^2$
    \item \textbf{Shifted GCD Sum:} $\sum_{\gcd(k, n)=1} \gcd(k-1, n) = \phi(n)d(n)$
    \item \textbf{Count GCD=d:} There are exactly $\phi(n/d)$ integers $i \le n$ such that $\gcd(i, n) = d$.

    % --- Properties & Divisibility ---
    \item \textbf{Divisibility I:} $a|b \implies \phi(a) | \phi(b)$
    \item \textbf{Divisibility II:} $n | \phi(a^n - 1)$ \hfill \textit{(For $a, n > 1$)}
    \item \textbf{Evenness:} $\phi(n)$ is even ($n \ge 3$). If $n$ has $r$ odd primes, $2^r | \phi(n)$.
    \item \textbf{Power Tower:} $a^x \equiv a^{\phi(m) + (x \pmod{\phi(m)})} \pmod m$ \hfill \textit{(If $x \ge \log_2 m$)}
    \item \textbf{Lower Bound:} $\phi(n) \ge \sqrt{n/2}$ \hfill \textit{(Roughly; $\phi(n) \ge \log_2 n$)}

    % --- Jordan Totient (Generalization) ---
    \item \textbf{Jordan Function $J_k(n)$:} Counts $k$-tuples $\le n$ forming coprime $(k+1)$-tuple with $n$.
    \item \textbf{Jordan Formula:} $J_k(n) = n^k \prod_{p|n} (1 - p^{-k})$ \hfill \textit{($J_1(n) = \phi(n)$)}
    \item \textbf{Jordan Sum:} $\sum_{d|n} J_k(d) = n^k$
\end{itemize}
}
}
% --- UPDATED SECTION: Partition Function ---
\subsection*{Partition Function: $p(n)$}
{ \footnotesize
\textbf{Pattern:} Form a sum $n$ where the \textbf{order does not matter}.
\begin{itemize}
    \item "How many ways to write $n$ as a sum of positive integers?"
    \item "How many ways to put $n$ *identical* balls into *identical* boxes?"
\end{itemize}
\hrule
\textbf{Definition:} Number of ways of writing $n$ as a sum of positive integers, disregarding order.
\textbf{Sequence $p(n)$ for $n=0, 1, 2, \dots$:}
\[
1, 1, 2, 3, 5, 7, 11, 15, 22, 30, 42, 56, 77, \dots
\]
\textbf{Recurrence (Pentagonal Number Theorem):}
\begin{align*}
p(n) &= \sum_{k \in \mathbb{Z} \setminus \{0\}} (-1)^{k-1} p(n - k(3k-1)/2) \\
     &= p(n-1) + p(n-2) - p(n-5) - p(n-7) + \dots \\
\end{align*}
}
% --- SECTION: Ceils and Floors ---
\subsection*{Ceils and Floors}
For $x, y \in \mathbb{R}$, $m, n \in \mathbb{Z}$:
{ \footnotesize
\begin{itemize}
  \item $\lfloor x \rfloor \leq x < \lfloor x \rfloor + 1\text{;    }\lceil x \rceil - 1 < x \leq \lceil x \rceil$
  \item $- \lfloor x \rfloor = \lceil -x \rceil \text{;    } - \lceil x \rceil = \lfloor -x \rfloor$
  \item $\lfloor x + n \rfloor = \lfloor x \rfloor + n \text{, } \lceil x + n \rceil = \lceil x \rceil + n$
  \item $\lfloor x \rfloor = m \Leftrightarrow x - 1 < m \leq x < m + 1$
  \item $\lceil x \rceil = n \Leftrightarrow n - 1 < x \leq n < x + 1$
  \item If $n > 0$, $\lfloor \frac{\lfloor x \rfloor + m}{n} \rfloor = \lfloor \frac{x + m}{n} \rfloor$
  \item If $n > 0$, $\lceil \frac{\lceil x \rceil + m}{n} \rceil = \lceil \frac{x + m}{n} \rceil$
  \item If $n > 0$, $\lfloor \frac{\lfloor \frac{x}{m} \rfloor }{n} \rfloor = \lfloor \frac{x}{mn} \rfloor$
  \item If $n > 0$, $\lceil \frac{\lceil \frac{x}{m} \rceil }{n} \rceil = \lceil \frac{x}{mn} \rceil$
  \item For $m, n > 0$, $\sum_{k = 1}^{n - 1} \lfloor \frac{km}{n} \rfloor = \frac{(m - 1)(n - 1) + \gcd(m, n) - 1}{2}$
  \item $\lfloor n/j \rfloor = x \text{ for } j \in [\lfloor n/(x+1) \rfloor + 1, \lfloor n/x \rfloor]$
  \item Modulo definition: $a \pmod m = a - m \lfloor a/m \rfloor$
\end{itemize}
}

% --- SECTION: Recurrences ---
\subsection*{Recurrences}
{ \footnotesize
If $a_n = c_1 a_{n-1} + \dots + c_k a_{n-k}$, and $r_1, \dots, r_k$ are distinct roots of $x^k - c_1 x^{k-1} - \dots - c_k$, there are $d_1, \dots, d_k$ s.t.
\[a_n = d_1r_1^n + \dots + d_kr_k^n. \]
Non-distinct roots $r$ become polynomial factors, e.g. $a_n = (d_1n + d_2)r^n$.
}
% --- SECTION: Trigonometry ---
\subsection*{Trigonometry}
{ \footnotesize
\begin{align*}
\sin(v+w)&{}=\sin v\cos w+\cos v\sin w\\
\cos(v+w)&{}=\cos v\cos w-\sin v\sin w\\
\tan(v+w)&{}=\dfrac{\tan v+\tan w}{1-\tan v\tan w}\\
\sin v+\sin w&{}=2\sin\dfrac{v+w}{2}\cos\dfrac{v-w}{2}\\
\cos v+\cos w&{}=2\cos\dfrac{v+w}{2}\cos\dfrac{v-w}{2}
\end{align*}
}
\[ (V+W)\tan(\frac{v-w}{2}){}=(V-W)\tan(\frac{v+w}{2}) \]
$V, W$ are sides opposite to angles $v, w$.
   $a\cos x+b\sin x=r\cos(x-\phi)$\\
   $a\sin x+b\cos x=r\sin(x+\phi)$\\
where $r=\sqrt{a^2+b^2}, \phi=\operatorname{atan2}(b,a)$.

% --- SECTION: Geometry ---
\subsection*{Geometry}

\subsubsection*{Rectangles and Squares}
{ \footnotesize
\begin{itemize}
    \item Area of a rectangle: $A = l \cdot w$
    \item Perimeter of a rectangle: $P = 2l + 2w$
    \item Diagonal of a rectangle: $d = \sqrt{l^2 + w^2}$
    \item Area of a square: $A = \text{side}^2$
    \item Perimeter of a square: $P = 4 \cdot \text{side}$
    \item Diagonal of a square: $d = \sqrt{2} \cdot \text{side}$
\end{itemize}
}
\subsubsection*{Triangles}
{ \footnotesize
Side lengths: $a,b,c$; Semiperimeter: $p=\dfrac{a+b+c}{2}$
\begin{itemize}
    \item Area: $A = \frac{1}{2} \cdot b \cdot h$
    \item Perimeter: $P = a + b + c$
    \item Heron's Area: $A=\sqrt{p(p-a)(p-b)(p-c)}$
    \item Circumradius: $R=\dfrac{abc}{4A}$
    \item Inradius: $r=\dfrac{A}{p}$
    \item Length of median: $m_a=\tfrac{1}{2}\sqrt{2b^2+2c^2-a^2}$
    \item Length of bisector: $s_a=\sqrt{bc\left[ 1 - (a / (b+c))^2 \right]}$
    \item Law of Sines: $\dfrac{\sin\alpha}{a}=\dfrac{\sin\beta}{b}=\dfrac{\sin\gamma}{c}=\dfrac{1}{2R}$
    \item Law of Cosines: $a^2=b^2+c^2-2bc\cos\alpha$
    \item Law of Tangents: $\dfrac{a+b}{a-b}=\dfrac{\tan((\alpha + \beta)/2)}{\tan((\alpha - \beta)/2)}$
\end{itemize}
}
\subsubsection*{Circles}
{ \footnotesize
\begin{itemize}
    \item Area: $A = \pi \cdot r^2$
    \item Circumference: $C = 2 \pi \cdot r$
    \item Sector Area: $A_{\text{sector}} = \frac{\theta}{360^\circ} \cdot \pi \cdot r^2$ (in degrees)
    \item Arc Length: $l = \frac{\theta}{360^\circ} \cdot 2 \pi \cdot r$ (in degrees)
\end{itemize}
}
\subsubsection*{Polygons (n-sided)}
{ \footnotesize
\begin{itemize}
    \item Sum of interior angles: $(n-2) \times 180^\circ$
    \item A single angle (regular): $\frac{(n-2) \times 180^\circ}{n}$
    \item Amount of diagonals: $\frac{n(n-3)}{2}$
    \item Sum of exterior angles: $360^\circ$
    \item Area (regular): $\frac{1}{4} n s^2 \cot(\frac{\pi}{n})$
    \item Area (with apothem): $\frac{1}{2} \cdot n \cdot s \cdot a$
\end{itemize}
}
\subsubsection*{3D Shapes}
{ \footnotesize
\begin{itemize}
    \item \textbf{Cube:} Volume $V = s^3$, Surface Area $SA = 6s^2$
    \item \textbf{Sphere:} Volume $V = \frac{4}{3}\pi r^3$, Surface Area $SA = 4\pi r^2$
    \item \textbf{Cylinder:} Volume $V = \pi r^2 h$, Surface Area $SA = 2\pi r^2 + 2\pi rh$
    \item \textbf{Cone:} Volume $V = \frac{1}{3}\pi r^2 h$, Surface Area $SA = \pi r s + \pi r^2$, where $s = \sqrt{h^2+r^2}$
    \item \textbf{Cuboid:} Volume $V = lwh$, Surface Area $SA = 2(wh + lw + lh)$
\end{itemize}
}
\subsubsection*{Quadrilaterals}
{ \footnotesize
With side lengths $a,b,c,d$, diagonals $e, f$, diagonals angle $\theta$, area $A$ and
magic flux $F=b^2+d^2-a^2-c^2$:
\[ 4A = 2ef \cdot \sin\theta = F\tan\theta = \sqrt{4e^2f^2-F^2} \]
 For cyclic quadrilaterals the sum of opposite angles is $180^\circ$,
$ef = ac + bd$, and $A = \sqrt{(p-a)(p-b)(p-c)(p-d)}$
}
\subsubsection*{Pick's Theorem}
{ \footnotesize
For a polygon on a grid:
\[ A = I + \frac{B}{2} - 1 \]
$A$ = Area, $I$ = Interior points, $B$ = Boundary points.
}
\subsubsection*{Spherical coordinates}
{ \footnotesize
\[\begin{array}{cc}
x = r\sin\theta\cos\phi & r = \sqrt{x^2+y^2+z^2}\\
y = r\sin\theta\sin\phi & \theta = \textrm{acos}(z/\sqrt{x^2+y^2+z^2})\\
z = r\cos\theta & \phi = \textrm{atan2}(y,x)
\end{array}\]
}
\subsection*{Geometry}
{ \footnotesize
\textbf{Description:} Essential formulas for 2D/3D shapes, triangle properties, and lattice points.

{ \footnotesize
\begin{itemize}[leftmargin=*, nosep]
    % --- 2D Shapes & Circles ---
    \item \textbf{Ellipse Area:} $A = \pi a b$ \hfill \textit{($a, b$ are semi-axes)}
    \item \textbf{Regular Polygon Area:} $A = \frac{1}{2}nRr$ \hfill \textit{($R=$ circumradius, $r=$ apothem)}
    \item \textbf{Sector Area:} $A = \frac{\theta}{2} r^2$ \hfill \textit{($\theta$ in radians)}
    \item \textbf{Chord Length:} $d = 2r \sin(\frac{\theta}{2}) = 2\sqrt{r^2 - x^2}$ \hfill \textit{($x=$ dist from center)}
    \item \textbf{Pick's Theorem:} $A = I + \frac{B}{2} - 1$ \hfill \textit{($I=$ interior, $B=$ boundary points)}

    % --- 3D Volumes ---
    \item \textbf{Sphere Volume:} $V = \frac{4}{3}\pi r^3$
    \item \textbf{Cone Volume:} $V = \frac{1}{3}\pi r^2 h$
    \item \textbf{Pyramid Volume:} $V = \frac{1}{3} B h$ \hfill \textit{($B=$ base area)}
    \item \textbf{Rectangular Prism:} $V = lwh$

    % --- Triangle Laws ---
    \item \textbf{Law of Sines:} $\frac{a}{\sin A} = \frac{b}{\sin B} = \frac{c}{\sin C} = 2R$
    \item \textbf{Law of Cosines:} $c^2 = a^2 + b^2 - 2ab\cos C$

    % --- Triangle Properties ---
    \item \textbf{Altitude ($h_a$):} $h_a = \frac{2A}{a} = c \sin B = b \sin C$
    \item \textbf{Median ($m_a$):} $m_a = \frac{1}{2}\sqrt{2b^2 + 2c^2 - a^2}$
    \item \textbf{Angle Bisector Theorem:} $\frac{BD}{DC} = \frac{AB}{AC}$ \hfill \textit{(Divides opposite side)}
    
    % --- Triangle Radii ---
    \item \textbf{Circumradius ($R$):} $R = \frac{abc}{4A} = \frac{a}{2\sin A}$
    \item \textbf{Inradius ($r$):} $r = \frac{A}{s} = \frac{\sqrt{(s-a)(s-b)(s-c)}}{s}$ \hfill \textit{($s=$ semi-perimeter)}
\end{itemize}
}
}
\subsection*{Coordinate Geometry}
{ \footnotesize
{ \footnotesize
\begin{itemize}
    \item \textbf{Distance (2 points):} $(x_1, y_1), (x_2, y_2)$
    $D = \sqrt{(x_2-x_1)^2 + (y_2-y_1)^2}$
    \item \textbf{Midpoint:} $M = \left(\frac{x_1+x_2}{2}, \frac{y_1+y_2}{2}\right)$
    \item \textbf{Slope (2 points):} $m = \frac{y_2-y_1}{x_2-x_1}$
    \item \textbf{Line (point-slope):} $y - y_1 = m(x - x_1)$
    \item \textbf{Line (slope-intercept):} $y = mx + b$
    \item \textbf{Line (two-point):} $y - y_1 = \frac{y_2-y_1}{x_2-x_1}(x - x_1)$
    \item \textbf{Line (general):} $Ax + By + C = 0$
    \item \textbf{Slope (from general):} $m = -A/B$
    \item \textbf{Parallel lines:} have the same slope ($m_1 = m_2$)
    \item \textbf{Perpendicular lines:} $m_1 = -1/m_2$
    \item \textbf{Distance (point to line):} Point $(x_0, y_0)$ to line $Ax+By+C=0$. $D = \frac{|Ax_0+By_0+C|}{\sqrt{A^2+B^2}}$
    \item \textbf{Area of Triangle (vertices):} $(x_1, y_1), (x_2, y_2), (x_3, y_3)$
    $A = \frac{1}{2} |x_1(y_2-y_3) + x_2(y_3-y_1) + x_3(y_1-y_2)|$
    \item \textbf{Circle:} Center $(h, k)$, radius $r$. $(x-h)^2 + (y-k)^2 = r^2$
    \item \textbf{Distance (2 circle centers):} $D = \sqrt{(h_2-h_1)^2 + (k_2-k_1)^2}$
    \item \textbf{Tangent slope on circle:} At point $(x_0, y_0)$ on circle $x^2+y^2=r^2$. $m = -x_0/y_0$
    \item \textbf{Area of Parallelogram (vertices):} $(x_1, y_1), \dots, (x_4, y_4)$
    $A = |x_1y_2 + x_2y_3 + x_3y_4 + x_4y_1 - x_2y_1 - x_3y_2 - x_4y_3 - x_1y_4|$
    \item \textbf{Ellipse:} $\frac{x^2}{a^2} + \frac{y^2}{b^2} = 1$
    \item \textbf{Hyperbola:} $\frac{x^2}{a^2} - \frac{y^2}{b^2} = 1$
    \item \textbf{Parabola:} Vertex $(h, k)$, focus $(h+p, k)$. $(x-h) = 4p(y-k)$
\end{itemize}
}
}

\subsection*{Derivatives/Integrals}
{ \footnotesize
{ \footnotesize
\begin{align*}
  \dfrac{d}{dx}\arcsin x = \dfrac{1}{\sqrt{1-x^2}} &&& \dfrac{d}{dx}\arccos x = -\dfrac{1}{\sqrt{1-x^2}} \\
  \dfrac{d}{dx}\tan x = 1+\tan^2 x &&& \dfrac{d}{dx}\arctan x = \dfrac{1}{1+x^2} \\
  \int\tan ax = -\dfrac{\ln|\cos ax|}{a} &&& \int xe^{ax}dx = \frac{e^{ax}}{a^2}(ax-1) \\
  \int e^{-x^2} = \frac{\sqrt \pi}{2} \text{erf}(x) &&& \int x\sin ax = \frac{\sin ax-ax \cos ax}{a^2}
\end{align*}}
\[\int_a^bf(x)g(x)dx = [F(x)g(x)]_a^b-\int_a^bF(x)g'(x)dx\]
}
\subsection*{Sums}
\subsubsection*{Basic Sums}
{ \footnotesize
\begin{itemize}
    \item $\sum_{i=1}^{n} 1 = n$
    \item $\sum_{i=1}^{n} i = \frac{n(n+1)}{2}$
    \item $\sum_{i=1}^{n} i^2 = \frac{n(n+1)(2n+1)}{6}$
    \item $\sum_{i=1}^{n} i^3 = \left(\frac{n(n+1)}{2}\right)^2 = \frac{n^2(n+1)^2}{4}$
    \item $\sum_{i=1}^{n} i^4 = \frac{n(n+1)(2n+1)(3n^2+3n-1)}{30}$
    \item Sum of first $n$ odd: $\sum_{i=1}^{n} (2i-1) = n^2$
    \item Sum of first $n$ even: $\sum_{i=1}^{n} 2i = n(n+1)$
\end{itemize}
}
\subsubsection*{Arithmetic Progression (AP)}
{ \footnotesize
$a_n = a_1 + (n-1)d$
$S_n = \frac{n}{2}(2a_1 + (n-1)d) = \frac{n}{2}(a_1 + a_n)$
$a_n = a_m + (n-m)d$
}
\subsubsection*{Geometric Progression (GP)}
{ \footnotesize
$a_n = a_1 r^{(n-1)}$
$S_n = \frac{a_1(r^n - 1)}{r-1}$ (finite)
$S_\infty = \frac{a_1}{1-r}$ (for $|r|<1$)
$P_n = a_1^n r^{n(n-1)/2}$
$c^a + c^{a+1} + \dots + c^{b} = \frac{c^{b+1} - c^a}{c-1}, c \neq 1$
}
\subsubsection*{Bernoulli Numbers \& Sum of Powers}
{ \footnotesize
\textbf{Pattern:} Compute $\sum_{i=1}^n i^k$ where \textbf{$n$ is large} but \textbf{$k$ is small}.
\begin{itemize}
    \item "Find $(1^5 + 2^5 + \dots + n^5) \pmod{10^9+7}$ for $n=10^{18}$."
\end{itemize}
\textbf{Sequence $B_k$ for $k=0, 1, 2, \dots$:}
\[
1, \frac{1}{2}, \frac{1}{6}, 0, -\frac{1}{30}, 0, \frac{1}{42}, 0, -\frac{1}{30}, \dots
\]
\textit{(Note: Using $B_1 = +1/2$. The $B_1 = -1/2$ convention also exists.)}

\textbf{EGF for $B_k$ (using $B_1 = -1/2$):}
\[
\frac{x}{e^x - 1} = \sum_{k=0}^\infty B_k \frac{x^k}{k!}
\]
\textbf{Faulhaber's Formula (Sum of Powers):}
\[
\sum_{i=0}^{n-1} i^m = \frac{1}{m+1} \sum_{k=0}^m \binom{m+1}{k} B_k n^{m+1-k}
\]
}
\section*{Combinatorics}

\subsection*{Binomial Theorem}
{ \footnotesize
\textbf{Description:} Used for expanding powers of binomials $(a+b)^p$. The coefficients $\binom{p}{k}$ give the number of ways to choose $k$ items from $p$.

\textbf{Formula:}
\[ (a+b)^p = \sum_{k=0}^{p} \binom{p}{k} a^k b^{p-k} \]

\subsection*{Stars and Bars}
{ \footnotesize
\textbf{Description:} Used to find the number of ways to distribute \textbf{identical (unlabeled)} objects ($n$) into \textbf{distinct} bins ($k$).

\textbf{Formulas:}
\begin{itemize}[leftmargin=*, nosep]
    \item \textbf{Empty bins NOT valid (Positive Integer Solutions):} $\binom{n-1}{k-1}$
    \item \textbf{Empty bins VALID (Non-Negative Integer Solutions):} $\binom{n+k-1}{k-1}$
    \item \textbf{Bounded Constraints (via Inclusion-Exclusion):} \\
    Formula: $\displaystyle \sum_{j=0}^{k} (-1)^j \binom{k}{j} \binom{\text{Top}}{k-1}$ \\
    \textit{(Stop summation when $\text{Top} < k-1$)}
    \begin{itemize}[nosep]
        \item \textbf{$0 \le x_i \le v$:} $\text{Top} = n - j(v+1) + k - 1$
        \item \textbf{$0 \le x_i < v$:} $\text{Top} = n - j(v) + k - 1$
        \item \textbf{$0 < x_i \le v$:} $\text{Top} = n - j(v) - 1$
        \item \textbf{$0 < x_i < v$:} $\text{Top} = n - j(v-1) - 1$
    \end{itemize}
\end{itemize}
}
}
\subsection*{Binomial Coefficients $\binom{n}{k}$}
{ \footnotesize
\textbf{Description:} $\binom{n}{k}$ is the number of ways to choose $k$ elements from $n$ distinct elements. Essential for DP, probability, and modular arithmetic.

{ \footnotesize
\begin{itemize}[leftmargin=*, nosep]
    % --- Core Definition & Computation ---
    \item \textbf{Definition:} $\binom{n}{k} = \frac{n!}{k!(n-k)!}$
    \item \textbf{Symmetry:} $\binom{n}{k} = \binom{n}{n-k}$
    \item \textbf{Multiplicative ($\mathcal{O}(k)$):} $\binom{n}{k} = \prod_{i=1}^k \frac{n-i+1}{i}$
    \item \textbf{Base Cases:} $\binom{n}{0} = 1, \quad \binom{n}{n} = 1$
    
    % --- Identities (Recurrence & Addition) ---
    \item \textbf{Pascal's Identity ($\mathcal{O}(1)$ DP):} $\binom{n}{k} = \binom{n-1}{k} + \binom{n-1}{k-1}$
    \item \textbf{Absorption Identity:} $\binom{n}{k} = \frac{n}{k} \binom{n-1}{k-1}$
    \item \textbf{Shifted Recurrence I:} $\binom{n}{k} = \frac{n-k+1}{k} \binom{n}{k-1}$
    \item \textbf{Shifted Recurrence II:} $\binom{n+1}{k} = \frac{n+1}{n-k+1} \binom{n}{k}$
    \item \textbf{Vandermonde's Identity:} $\binom{m+n}{r} = \sum_{k=0}^r \binom{m}{k} \binom{n}{r-k}$
    % \item \textbf{Hockey-Stick Identity:} $\sum_{i=k}^{n} \binom{i}{k} = \binom{n+1}{k+1}$
    
    % --- Summation Formulas ---
    \item \textbf{Sum of Row (Total Subsets):} $\sum_{k=0}^{n} \binom{n}{k} = 2^n$
    \item \textbf{Sum of K (Weighted Sum):} $\sum_{k=1}^n k \binom{n}{k} = n 2^{n-1}$
    \item \textbf{Sum of $K^2$ (Weighted Sum II):} $\sum_{k=1}^n k^2 \binom{n}{k} = n(n+1) 2^{n-2}$
     

    \item \textbf{Extraction Identity:} $\binom{n}{k} = \frac{n}{k}\binom{n-1}{k-1}$ \hfill \textit{(Isolate $k$)}

    % --- Row & Index Sums ---

    \item \textbf{Even/Odd Index Sum:} $\sum_{i \ge 0} \binom{n}{2i} = \sum_{i \ge 0} \binom{n}{2i+1} = 2^{n-1}$
    \item \textbf{Alternating Partial Sum:} $\sum_{i=0}^k (-1)^i \binom{n}{i} = (-1)^k \binom{n-1}{k}$
    \item \textbf{Partial Row Sum:} $\sum_{i=0}^n \binom{2n}{i} = 2^{2n-1} + \frac{1}{2}\binom{2n}{n}$

    % --- Weighted Sums & Expansions ---
    % \item \textbf{Weighted Sum (Expectation):} $\sum_{k=1}^n k\binom{n}{k} = n2^{n-1}$
    % \item \textbf{Variance Sum ($2^{nd}$ Moment):} $\sum_{k=1}^n k^2\binom{n}{k} = (n+n^2)2^{n-2}$
    \item \textbf{Binomial Expansion:} $\sum_{i=0}^n k^i \binom{n}{i} = (k+1)^n$

    % --- Diagonal & Matrix Sums ---
    \item \textbf{Hockey-Stick Identity:} $\sum_{i=r}^n \binom{i}{r} = \binom{n+1}{r+1}$ \hfill \textit{(Col sum: fix $r$, vary $n$)}
    \item \textbf{Parallel Summation:} $\sum_{i=0}^k \binom{n+i}{i} = \binom{n+k+1}{k}$ \hfill \textit{(Diag sum: vary both)}
    \item \textbf{Fibonacci Sum:} $\sum_{k=0}^n \binom{n-k}{k} = Fib_{n+1}$ \hfill \textit{(Shallow diagonals)}

    % --- Vandermonde Convolutions ---
    % \item \textbf{Vandermonde's Identity:} $\sum_{k=0}^r \binom{m}{k}\binom{n}{r-k} = \binom{m+n}{r}$
    \item \textbf{Sum of Squares:} $\sum_{i=0}^k \binom{k}{i}^2 = \binom{2k}{k}$ \hfill \textit{(Case $m=n=r=k$)}
    \item \textbf{Convolution Product:} $\sum_{k=0}^n \binom{n}{k}\binom{n}{n-k} = \binom{2n}{n}$
    \item \textbf{Fixed Element Convolution:} $\sum_{i=1}^n \binom{n}{i}\binom{n-1}{i-1} = \binom{2n-1}{n-1}$
    \item \textbf{Subset of a Subset:} $\sum_{k=q}^n \binom{n}{k}\binom{k}{q} = 2^{n-q}\binom{n}{q}$
    \item \textbf{Partial Sum of Squares:} $\sum_{i=0}^n \binom{2n}{i}^2 = \frac{1}{2}\left[ \binom{4n}{2n} + \binom{2n}{n}^2 \right]$
\end{itemize}
}
}
\subsection*{Stirling Numbers of the First Kind: $c(n, k)$}
{ \footnotesize
\textbf{Pattern:} Count permutations in terms of their \textbf{cycle structure}.
\begin{itemize}
    \item "Arrange $n$ people around $k$ identical round tables."
    \item "Count permutations of $n$ elements with exactly $k$ cycles."
    \item Lets $[n, k]$ be the stirling number of the first kind, then
\[\displaystyle \bigl[\!\begin{smallmatrix} n \\ n\ -\ k \end{smallmatrix}\!\bigr] = \sum_{0 \leq i_1 < i_2 < i_k < n}{i_1i_2....i_k.}\]
\end{itemize}
\textbf{Definition:} Number of permutations of $n$ items with $k$ cycles.
\begin{align*}
c(n, k) &= (n-1)c(n-1, k) + c(n-1, k-1) \\
c(n, 0) &= 0 \quad (n > 0), \quad c(0, 0) = 1 \\
\sum_{k=0}^n c(n, k)x^k &= x(x+1)\dots(x+n-1)
\end{align*}
\textbf{Sequence $c(n, 2)$ for $n=0, 1, 2, \dots$:}
\[
0, 0, 1, 3, 11, 50, 274, 1764, 13068, \dots
\]
% \begin{itemize}
% \item Lets $[n, k]$ be the stirling number of the first kind, then
% \[\displaystyle \bigl[\!\begin{smallmatrix} n \\ n\ -\ k \end{smallmatrix}\!\bigr] = \sum_{0 \leq i_1 < i_2 < i_k < n}{i_1i_2....i_k.}\]
% \end{itemize}
}
\subsection*{Stirling Numbers of the Second Kind: $S(n, k) \text{ or } \genfrac{\{}{\}}{0pt}{}{n}{k}$}
{ \footnotesize
\textbf{Pattern:} Partition $n$ \textbf{distinct items} into $k$ \textbf{identical, non-empty boxes}.
\begin{itemize}
    \item "How many ways to put $n$ *labeled* balls into $k$ *unlabeled* boxes?"
    \item "Count ways to partition a set of $n$ elements into $k$ non-empty subsets."
\end{itemize}
\textbf{Definition:} Number of partitions of $n$ distinct elements into exactly $k$ non-empty subsets.
\begin{align*}
S(n, k) &= S(n-1, k-1) + k \cdot S(n-1, k) \\
S(n, 1) &= 1, \quad S(n, n) = 1 \\
S(n, k) &= \frac{1}{k!} \sum_{j=0}^k (-1)^{k-j} \binom{k}{j} j^n
\end{align*}
\begin{itemize}
\item $S(n,2)=2^{n-1}-1$
\item $S(n,k) \cdot k!$ = number of ways to color $n$ nodes using colors from $\displaystyle 1$ to $\displaystyle \displaystyle k$ such that each color is used at least once.
\item An $r$-associated Stirling number of the second kind is the number of ways to partition a set of $n$ objects into $\displaystyle \displaystyle k$ subsets, with each subset containing at least $r$ elements. It is denoted by $S_r( n , k )$ and obeys the recurrence relation.
$\displaystyle \displaystyle S_r(n+1,k) = k S_r(n,k) + \binom{n}{r-1} S_r(n-r+1,k-1)$
\item Denote the n objects to partition by the integers $\displaystyle 1, 2, …., n$. Define the reduced Stirling numbers of the second kind, denoted $S^d(n, k)$, to be the number of ways to partition the integers $\displaystyle 1, 2, …., n$ into k nonempty subsets such that all elements in each subset have pairwise distance at least d.
That is, for any integers i and j in a given subset, it is required that $|i - j| \geq d$. It has been shown that these numbers satisfy,
\(S^d(n, k) = S(n - d + 1, k - d + 1), n \geq k \geq d\)
\end{itemize}
}
\subsection*{Bell Numbers: $B(n)$}
{ \footnotesize
\textbf{Pattern:} Total ways to partition $n$ \textbf{distinct items} (number of boxes doesn't matter).
\begin{itemize}
    \item "Find the total number of equivalence relations on a set of $n$ elements."
    \item "How many ways to put $n$ *labeled* balls into *unlabeled* boxes?"
    \item Counts the number of partitions of a set.
\end{itemize}
\textbf{Definition:} Total number of partitions of $n$ distinct elements.
\[
B(n) = \sum_{k=0}^n S(n, k)
\]
\textbf{Sequence $B(n)$ for $n=0, 1, 2, \dots$:}
\[
1, 1, 2, 5, 15, 52, 203, 877, 4140, 21147, \dots
\]
\textbf{Recurrence (P-set construction):}
\[
B(n+1) = \sum_{k=0}^n \binom{n}{k} B(k)
\]
}
\subsection*{Catalan Numbers: $C_n$}
{ \footnotesize
\textbf{Pattern:} One of the most famous sequences. Look for:
\begin{itemize}
    \item \textbf{Balanced sequences:} "Valid (balanced) parenthesis strings of length $2n$."
    \item \textbf{Recursive splitting:} "Number of full binary trees with $n$ nodes."
    \item \textbf{Non-crossing paths:} "Paths from (0, 0) to (n, n) on a grid that do not go above $y=x$."
    \item \textbf{Polygon triangulation:} "Ways to triangulate a convex polygon with $n+2$ sides."
\end{itemize}
\textbf{Sequence $C_n$ for $n=0, 1, 2, \dots$:}
\[
1, 1, 2, 5, 14, 42, 132, 429, 1430, 4862, 16796, \dots
\]
\textbf{Closed Form:}
\[
C_n = \frac{1}{n+1} \binom{2n}{n} = \binom{2n}{n} - \binom{2n}{n+1}
\]
\textbf{Recurrence Relations:}
\begin{align*}
C_0 &= 1, \quad C_{n+1} = \sum_{i=0}^n C_i C_{n-i} \\
C_0 &= 1, \quad C_{n+1} = \frac{2(2n+1)}{n+2} C_n
\end{align*}
\begin{itemize}
\item \textbf{Balanced Parentheses count with prefix:}
The count of balanced parentheses sequences consisting of $n+k$ pairs of parentheses where the first $\displaystyle \displaystyle k$ symbols are open brackets. Let the number be $\displaystyle C_n^{(k)}$, then
\[\displaystyle \displaystyle C_n^{(k)} = \frac{k+1}{n+k+1} \binom{2n+k}{n}\]
\end{itemize}
}
\subsection*{Eulerian Numbers: $E(n, k)$}
{ \footnotesize
\textbf{Pattern:} Count permutations based on their \textbf{"runs"} or \textbf{"ascents/descents"}.
\begin{itemize}
    \item "Count permutations of $\{1, \dots, n\}$ with exactly $k$ ascents ($p_i < p_{i+1}$)."
\end{itemize}
\textbf{Definition:} Number of $n$-permutations with exactly $k$ rises (positions $i$ with $p_i > p_{i-1}$).
\begin{align*}
E(n, k) &= (n-k)E(n-1, k-1) + (k+1)E(n-1, k) \\
E(n, 0) &= E(n, n-1) = 1 \\
E(n, k) &= \sum_{j=0}^{k} (-1)^j \binom{n+1}{j} (k-j+1)^n
\end{align*}
}
\subsection*{Derangements: $D(n) \text{ or } !n$}
{ \footnotesize
\textbf{Pattern:} The "mixed-up hats" or "secret santa" problem.
\begin{itemize}
    \item "Count permutations of $n$ elements where \textbf{no element is in its original position}."
    \item "Find the number of permutations with \textbf{no fixed points} ($p_i \neq i$ for all $i$)."
\end{itemize}
\textbf{Definition:} Permutations of a set such that no element appears in its original position.
\textbf{Sequence $D(n)$ for $n=0, 1, 2, \dots$:}
\[
1, 0, 1, 2, 9, 44, 265, 1854, 14833, \dots
\]
\textbf{Recurrence Relations:}
\begin{align*}
D(n) &= (n-1) (D(n-1) + D(n-2)) \\
D(n) &= n \cdot D(n-1) + (-1)^n \\
D(n) &= \left\lfloor \frac{n!}{e} + \frac{1}{2} \right\rfloor = \left\lceil \frac{n!}{e} \right\rceil \quad (n \ge 1)
\end{align*}
}
\subsection*{Burnside's Lemma}
{ \footnotesize
\textbf{Pattern:} Count "distinct" objects under \textbf{symmetry} (rotations, reflections).
\begin{itemize}
    \item "Count distinct ways to color a necklace/bracelet/cube under rotation."
    \item The key is "up to symmetry," "distinct under rotation," etc.
\end{itemize}
\textbf{Definition:} Given a group $G$ of symmetries acting on a set $X$. The number of distinct elements of $X$ up to symmetry (number of orbits) is:
\[
|X/G| = \frac{1}{|G|} \sum_{g \in G} |X^g|
\]
where $X^g = \{x \in X \mid g \cdot x = x\}$ are the elements fixed by $g$.

\textbf{Special Case (Necklaces):}
For $k$ colors and $n$ beads, with $G = \mathbb{Z}_n$ (rotations):
\[
\text{Count} = \frac{1}{n} \sum_{d | n} \phi(d) \cdot k^{n/d}
\]
}
\subsection*{Permutation Cycles (EGF)}
{ \footnotesize
\textbf{Pattern:} Count permutations where \textbf{cycle lengths are restricted} to a set $S$.
\begin{itemize}
    \item "Count permutations of $n$ elements that consist *only* of cycles of length 2 (involutions)."
\end{itemize}
\textbf{Definition:} Let $g_S(n)$ be the number of $n$-permutations whose cycle lengths all belong to $S$. The Exponential Generating Function (EGF) is:
\[
\sum_{n \ge 0} g_S(n) \frac{x^n}{n!} = \exp \left( \sum_{n \in S} \frac{x^n}{n} \right)
\]
}
,\subsection*{Lucas's Theorem}
{ \footnotesize
\textbf{Pattern:} Compute $\binom{n}{k} \pmod p$ where \textbf{$n, k$ are large} but \textbf{$p$ is a small prime}.
\begin{itemize}
    \item "Calculate $\binom{10^{18}}{10^9} \pmod 7$."
\end{itemize}
\textbf{Definition:} Let $n, m$ be non-negative integers and $p$ a prime.
Write $n$ and $m$ in base $p$:
\begin{align*}
n &= n_k p^k + \dots + n_1 p + n_0 \\
m &= m_k p^k + \dots + m_1 p + m_0
\end{align*}
Then:
\[
\binom{n}{m} \equiv \prod_{i=0}^k \binom{n_i}{m_i} \pmod{p}
\]
(Note: $\binom{a}{b} = 0$ if $a < b$)
}
\subsection*{Series}
{ \footnotesize
$$e^x = 1+x+\frac{x^2}{2!}+\frac{x^3}{3!}+\dots,\,(-\infty<x<\infty)$$
$$\ln(1+x) = x-\frac{x^2}{2}+\frac{x^3}{3}-\frac{x^4}{4}+\dots,\,(-1<x\leq1)$$
$$\sqrt{1+x} = 1+\frac{x}{2}-\frac{x^2}{8}+\frac{2x^3}{32}-\frac{5x^4}{128}+\dots,\,(-1\leq x\leq1)$$
$$\sin x = x-\frac{x^3}{3!}+\frac{x^5}{5!}-\frac{x^7}{7!}+\dots,\,(-\infty<x<\infty)$$
$$\cos x = 1-\frac{x^2}{2!}+\frac{x^4}{4!}-\frac{x^6}{6!}+\dots,\,(-\infty<x<\infty)$$
$$(1-x)^{-r} = \sum_{i = 0}^{\infty} \binom{r + i - 1}{i}x^i, (r \in \mathbb{R})$$
}

\subsection*{Bitwise Formulas}
{ \footnotesize \begin{align*}
  a | b = a \oplus b + a \& b \\
  a \oplus(a \& b) = (a | b) \oplus b &&& a \oplus b = (a \& b) \oplus (a | b) \\
  a + b = a | b + a \& b &&& a + b = a \oplus b + 2(a \& b)
\end{align*}
  $a - b = (a \oplus (a \& b)) - ((a | b) \oplus a) = ((a | b) \oplus b) - ((a | b) \oplus a) = (a \oplus (a \& b)) - (b \oplus (a \& b)) = ((a | b) \oplus b) - (b \oplus (a \& b)) $
  \begin{itemize}
  \item $\displaystyle \displaystyle k_{th}$ bit is set in $\displaystyle x$ iff $\displaystyle x \mod 2^{k - 1} - x \mod 2^k \neq\ 0$ ($ = 2^k$ to be exact). It comes handy when you need to look at the bits of the numbers which are pair sums or subset sums etc.
\item $n \mod 2^i = n \& (2^i - 1)$
\item $\displaystyle 1\oplus 2 \oplus 3 \oplus \cdots \oplus (4k - 1) = 0$ for any $\displaystyle \displaystyle k \ge 0$
\end{itemize}
}

\subsection*{Algorithms}
{ \footnotesize
\textbf{Rotation of a n*m matrix:}
$(i,j) \to (j, n-i-1) \to (n-i-1, m-j-1) \to (m-j-1, i)$
}
\subsection*{Probability theory}
{ \footnotesize
Let $X$ be a discrete random variable with probability $p_X(x)$ of assuming the value $x$. It will then have an expected value (mean) $\mu=\mathbb{E}(X)=\sum_xxp_X(x)$ and variance $\sigma^2=V(X)=\mathbb{E}(X^2)-(\mathbb{E}(X))^2=\sum_x(x-\mathbb{E}(X))^2p_X(x)$ where $\sigma$ is the standard deviation. If $X$ is instead continuous it will have a probability density function $f_X(x)$ and the sums above will instead be integrals with $p_X(x)$ replaced by $f_X(x)$.

Expectation is linear:
\[\mathbb{E}(aX+bY) = a\mathbb{E}(X)+b\mathbb{E}(Y)\]
For independent $X$ and $Y$, \[V(aX+bY) = a^2V(X)+b^2V(Y).\]
}
\subsubsection*{Discrete distributions}
{ \footnotesize
\textbf{Binomial distribution:}
The number of successes in $n$ independent yes/no experiments, each which yields success with probability $p$ is $\textrm{Bin}(n,p),\,n=1,2,\dots,\, 0\leq p\leq1$.
\[p(k)=\binom{n}{k}p^k(1-p)^{n-k}\]
\[\mu = np,\,\sigma^2=np(1-p)\]
$\textrm{Bin}(n,p)$ is approximately $\textrm{Po}(np)$ for small $p$.

\textbf{First success distribution:}
The number of trials needed to get the first success in independent yes/no experiments, each which yields success with probability $p$ is $\textrm{Fs}(p),\,0\leq p\leq1$.
\[p(k)=p(1-p)^{k-1},\,k=1,2,\dots\]
\[\mu = \frac1p,\,\sigma^2=\frac{1-p}{p^2}\]

\textbf{Poisson distribution:}
The number of events occurring in a fixed period of time $t$ if these events occur with a known average rate $\kappa$ and independently of the time since the last event is $\textrm{Po}(\lambda),\,\lambda=t\kappa$.
\[p(k)=e^{-\lambda}\frac{\lambda^k}{k!}, k=0,1,2,\dots\]
\[\mu=\lambda,\,\sigma^2=\lambda\]
}
\subsubsection*{Continuous distributions}
{ \footnotesize
\textbf{Uniform distribution:}
If the probability density function is constant between $a$ and $b$ and 0 elsewhere it is $\textrm{U}(a,b),\,a<b$.
\[f(x) = \left\{
\begin{array}{cl}
\frac{1}{b-a} & a<x<b\\
0 & \textrm{otherwise}
\end{array}\right.\]
\[\mu=\frac{a+b}{2},\,\sigma^2=\frac{(b-a)^2}{12}\]

\textbf{Exponential distribution:}
The time between events in a Poisson process is $\textrm{Exp}(\lambda),\,\lambda>0$.
\[f(x) = \left\{
\begin{array}{cl}
\lambda e^{-\lambda x} & x\geq0\\
0 & x<0
\end{array}\right.\]
\[\mu=\frac{1}{\lambda},\,\sigma^2=\frac{1}{\lambda^2}\]

\textbf{Normal distribution:}
Most real random values with mean $\mu$ and variance $\sigma^2$ are well described by $\mathcal{N}(\mu,\sigma^2),\,\sigma>0$.
\[ f(x) = \frac{1}{\sqrt{2\pi\sigma^2}}e^{-\frac{(x-\mu)^2}{2\sigma^2}} \]
If $X_1 \sim \mathcal{N}(\mu_1,\sigma_1^2)$ and $X_2 \sim \mathcal{N}(\mu_2,\sigma_2^2)$ then
\[ aX_1 + bX_2 + c \sim \mathcal{N}(\mu_1+\mu_2+c,a^2\sigma_1^2+b^2\sigma_2^2) \]
}
\subsection*{Graph Theory}

\subsubsection*{Cayley's Formula}
{ \footnotesize
\textbf{Pattern:} Count \textbf{spanning trees} on $n$ \textbf{labeled} vertices in a \textbf{complete graph $K_n$}.
\begin{itemize}
    \item "How many trees can be formed using $n$ labeled nodes?"
\end{itemize}
\textbf{Definition:} The number of spanning trees on $n$ labeled vertices (in $K_n$) is $n^{n-2}$.
\textbf{Sequence $n^{n-2}$ for $n=1, 2, 3, \dots$:}
\[
1, 1, 3, 16, 125, 1296, 16807, \dots
\]
\textbf{Generalizations:}
\begin{itemize}
    \item \# with degrees $d_i$: $\frac{(n-2)!}{(d_1-1)!(d_2-1)!\dots(d_n-1)!}$ (Prufer Sequence)
\end{itemize}
}
\subsubsection*{Kirchhoff's Matrix Tree Theorem}
{ \footnotesize
\textbf{Pattern:} Count \textbf{spanning trees} in a \textbf{general graph $G$} (not complete).
\begin{itemize}
    \item "Given a grid graph, find the number of spanning trees."
\end{itemize}
\textbf{Definition:} Counts spanning trees in a graph $G$.
\begin{enumerate}
    \item Create the \textbf{Laplacian Matrix} $L = D - A$:
    \begin{itemize}
        \item $D$ = Degree Matrix (diagonal, $D_{ii} = \text{deg}(i)$)
        \item $A$ = Adjacency Matrix
    \end{itemize}
    Or, $L_{ij} =
    \begin{cases}
        \deg(i) & \text{if } i = j \\
        -1        & \text{if } i \ne j \text{ and } (i, j) \in E \\
        0         & \text{otherwise}
    \end{cases}$

    \item Remove \textbf{any} row $i$ and \textbf{any} column $j$ to get $L_{i,j}$.
    \item The number of spanning trees is $\det(L_{i,j})$.
\end{enumerate}
}
\subsubsection*{Erd\H{o}s–Gallai Theorem}
{ \footnotesize
\textbf{Pattern:} Given a sequence of numbers, can it be the \textbf{degree sequence} of a \textbf{simple graph}?
\begin{itemize}
    \item "Is the sequence $d_1, \dots, d_n$ a valid graphic sequence?"
\end{itemize}
\textbf{Definition:} A simple graph with node degrees $d_1 \ge \dots \ge d_n$ exists iff:
\begin{enumerate}
    \item $\sum_{i=1}^n d_i$ is even.
    \item For every $k \in [1, n]$:
    \[
    \sum_{i=1}^k d_i \le k(k-1) + \sum_{i=k+1}^n \min(d_i, k)
    \]
\end{enumerate}
}

\subsection*{Game Theory}

\subsubsection*{Sprague–Grundy Theorem}
{ \footnotesize
\textbf{Pattern:} An \textbf{impartial game} (moves depend on position, not player).
\begin{itemize}
    \item \textbf{Classic Nim:} "A game with multiple piles of stones."
    \item \textbf{Sum of games:} Game breaks into independent sub-games.
\end{itemize}
\textbf{Definition:} For impartial games.
\begin{itemize}
    \item \textbf{Grundy Value (G-value) / Nim-sum:}
    \[
    G(v) = \text{mex}(\{ G(v_i) \mid v \to v_i \text{ is a valid move} \})
    \]
    where $\text{mex}(S)$ is the Minimum Excluded value.
    
    \item \textbf{Losing Position:} $G(v) = 0$.
    \item \textbf{Winning Position:} $G(v) > 0$.
    
    \item \textbf{Sum of Games:} If a game is a sum of independent games $g_1, \dots, g_k$:
    \[
    G_{\text{total}} = G(g_1) \oplus G(g_2) \oplus \dots \oplus G(g_k)
    \]
    where $\oplus$ is the bitwise XOR operator.
\end{itemize}
}
\subsection*{Trivia}
{ \footnotesize
\textbf{Pythagorean triples}: The Pythagorean triples are uniquely generated by $a=k\cdot (m^{2}-n^{2})$, $b=k\cdot (2mn)$, $c=k\cdot (m^{2}+n^{2})$ with $m > n > 0$, $k > 0$, $\gcd(m,n) = 1$, both $m, n$ not odd.

\textbf{Primes}: $p=962592769$ is such that $2^{21} \mid p-1$, which may be useful. For hashing use 970592641 (31-bit number), 31443539979727 (45-bit), 3006703054056749 (52-bit). There are 78498 primes less than 1\,000\,000.

\textbf{Estimates}: $\sum_{d|n} d = O(n \log \log n)$. % <-- Typo fixed here

\textbf{Prime Gaps}: For primes $> 10^{12}$, the max gap is not definitively known, but a gap of 1600 is a safe upper bound for practical purposes. (The largest known gap is 1550).

\textbf{Prime count}: 5133 upto 5e4. 9592 upto 1e5. 17984 upto 2e5. 78498 upto 1e6. 5761455 upto 1e8.

\textbf{max NOD} $\leq n$: 100 for $n = 5e4$. 500 for $n = 1e7$. 2000 for $n = 1e10$. 200\,000 for $n = 1e19$.

\textbf{max Unique Prime Factors}: 6 upto 5e5. 7 upto 9e6. 8 upto 2e8. 9 upto 6e9. 11 upto 7e12. 15 upto 3e19.

\textbf{Quadratic Residue}: $(\frac{a}{p})$ is 0 if $p | a$, 1 if $a$ is a quadratic residue, -1 otherwise. Euler: $(\frac{a}{p}) = a^{(p - 1) / 2} (\mod p)$ (prime). Jacobi: if $n = p_1^{e_1}\cdots p_k^{e_k}$ then $(\frac{a}{n}) = \prod (\frac{a}{p_i})^{e_i}$.

\textbf{Chicken McNugget:} If $a, b$ coprime, there are $\frac{1}{2}(a - 1)(b - 1)$ numbers not of form $ax + by$ $(x, y \geq 0)$, the largest being $ab - a - b$.
}
\section*{Extra Formulas}
\subsection*{Math Identities \& Algebra}
{ \footnotesize
\textbf{Description:} Fundamental algebraic identities, inequalities, and optimization theorems.
}
{ \footnotesize
\begin{itemize}[leftmargin=*, nosep]
    % --- Sums & Expansions ---
    \item \textbf{Factorial Sum:} $\displaystyle \sum_{i=0}^n i \cdot i! = (n+1)! - 1$
    \item \textbf{Difference of Powers:} $a^n - b^n = (a-b)(a^{n-1} + a^{n-2}b + \dots + b^{n-1})$
    \item \textbf{Weighted Geo. Sum:} $\displaystyle \sum_{i=1}^n i a^i = \frac{a(n a^{n+1} - (n+1)a^n + 1)}{(a-1)^2}$
    \item \textbf{Lagrange's Identity:} $(\sum a_k^2)(\sum b_k^2) - (\sum a_k b_k)^2 = \sum_{i<j} (a_i b_j - a_j b_i)^2$
    \item \textbf{Subset Product Sum:} Sum of products of all subsets of $A$ is $\prod_{i=1}^n (a_i + 1)$.

    % --- Inequalities & Floors ---
    \item \textbf{Min/Max Identity:} $\min(a+b, c) = a + \min(b, c-a)$
    \item \textbf{Abs Diff Identity:} $|a-b| + |b-c| + |c-a| = 2(\max(a,b,c) - \min(a,b,c))$
    \item \textbf{Floor Inequality:} $ab \le c \iff a \le \lfloor c/b \rfloor$ \hfill \textit{(Also valid for $\ge, >, <$)}
    \item \textbf{Nested Floor:} $\lfloor \frac{\lfloor x/m \rfloor}{n} \rfloor = \lfloor \frac{x}{mn} \rfloor$ \hfill \textit{(Same for $\lceil \cdot \rceil$)}

    % --- Polynomials & Optimization ---
    \item \textbf{Vieta's Formulas:} $\sum_{1 \le i_1 < \dots < i_k \le n} \left(\prod_{j=1}^k r_{i_j}\right) = (-1)^k \frac{a_{n-k}}{a_n}$
    \item \textbf{Min Absolute Error:} $\min_x \sum |a_i - x| \implies x = \text{median}(a)$
    \item \textbf{Min Squared Error:} $\min_x \sum (a_i - x)^2 \implies x = \text{mean}(a)$
\end{itemize}
}

\subsection*{Pythagorean Triples \& Sum of Squares}
{ \footnotesize
\textbf{Description:} Generating $a^2+b^2=c^2$ and counting ways to write integers as sums of squares.
}
{ \footnotesize
\begin{itemize}[leftmargin=*, nosep]
    % --- Pythagorean Triples ---
    \item \textbf{Euclid's Formula:} $a = k(m^2-n^2), b = k(2mn), c = k(m^2+n^2)$ \\
    \textit{($m>n$, $\gcd(m,n)=1$, distinct parity generates primitive triples)}
    \item \textbf{Count Hypotenuse $n$:} $\frac{1}{2} \left( \prod_{p|n, p \equiv 1 \pmod 4} (2\alpha_p + 1) - 1 \right)$ \hfill \textit{($n = \prod p^{\alpha_p}$)}

    % --- Sum of Squares Function r_k(n) ---
    \item \textbf{2 Squares ($r_2(n)$):} $4(d_1(n) - d_3(n))$ \hfill \textit{($d_1, d_3$ count divisors $\equiv 1, 3 \pmod 4$)}
    \item \textbf{4 Squares ($r_4(n)$):} $8 \sum_{d|n, 4 \nmid d} d$
    \item \textbf{8 Squares ($r_8(n)$):} $16 \sum_{d|n} (-1)^{n+d} d^3$
\end{itemize}
}

\subsection*{Divisor Functions $\sigma_x(n)$}
{ \footnotesize
\textbf{Description:} Properties of $\sigma_x(n) = \sum_{d|n} d^x$. $\sigma_0$ is count, $\sigma_1$ is sum.
}
{ \footnotesize
\begin{itemize}[leftmargin=*, nosep]
    \item \textbf{Computation:} $\sigma_x(p^a) = \frac{p^{(a+1)x}-1}{p^x-1}$. Multiplicative: $\sigma_x(ab)=\sigma_x(a)\sigma_x(b)$.
    \item \textbf{Divisor Product:} $\prod_{d|n} d = n^{\sigma_0(n)/2}$
    \item \textbf{Summatory $\sigma_0$:} $\sum_{i=1}^x \sigma_0(i) = 2\sum_{k=1}^{\sqrt{x}} \lfloor \frac{x}{k} \rfloor - \lfloor \sqrt{x} \rfloor^2$ \hfill \textit{(Hyperbola Method)}
    \item \textbf{Summatory $\sigma_1$:} $\sum_{i=1}^x \sigma_1(i) = \sum_{k=1}^x k \lfloor \frac{x}{k} \rfloor$
\end{itemize}
}

\subsection*{Modular Arithmetic Properties}
{ \footnotesize
\textbf{Description:} Essential identities for modular operations.
}
{ \footnotesize
\begin{itemize}[leftmargin=*, nosep]
    \item \textbf{Cancellation Law:} $ac \equiv bc \pmod m \implies a \equiv b \pmod{m/\gcd(c,m)}$
    \item \textbf{Freshman's Dream:} $(x+y)^p \equiv x^p + y^p \pmod p$ \hfill \textit{($p$ is prime)}
    \item \textbf{Modulo Distributivity:} $ab \equiv a(b \bmod c) \pmod{ac}$
\end{itemize}
}
\subsection*{Narayana Numbers $N(n,k)$}
{ \footnotesize
\textbf{Description:} Counts Dyck paths of length $2n$ with $k$ peaks, or valid parentheses with $k$ distinct nestings `()`.
}

{ \footnotesize
\begin{itemize}[leftmargin=*, nosep]
    \item \textbf{Formula:} $N(n,k) = \frac{1}{n}\binom{n}{k}\binom{n}{k-1}$
    \item \textbf{Usage:} Number of expressions with $n$ pairs of parentheses containing exactly $k$ immediate `()` sub-patterns.
    \item \textbf{Example:} $N(4, 2) = 6$ (6 valid sequences of 4 pairs have exactly two `()` nestings).
    \item \textbf{Relation:} $\sum_{k=1}^n N(n,k) = C_n$ \hfill \textit{(Sums to $n$-th Catalan number)}
\end{itemize}
}
\subsection*{Combinatorics}
{ \footnotesize
\textbf{Description:} Counting sequences, Pascal properties, inversions, and permutation restrictions.
}

{ \footnotesize
\begin{itemize}[leftmargin=*, nosep]
    % --- Divisibility & Pascal ---
    \item \textbf{Power of 2 in $\binom{2n}{n}$:} Highest power is $2^x$, where $x$ is the number of $1$s in binary $n$.
    \item \textbf{Pascal Parity:} Odd terms in row $n$ is $2^x$ ($x=$ count of $1$s in binary $n$). Row $2^n-1$ is all odd.
    \item \textbf{Pascal Prime Row:} For prime $p$, all $\binom{p}{k}$ ($1 \le k < p$) are divisible by $p$.
    \item \textbf{Primality Test:} $n \ge 2$ is prime $\iff n \mid \binom{n}{k}$ for all $1 \le k < n$.
    \item \textbf{Kummer’s Theorem:} Largest power of $p$ dividing $\binom{n}{m}$ is the number of carries adding $m$ to $n-m$ in base $p$.

    % --- Counting Formulas ---
    \item \textbf{No Adjacent 0s:} Binary sequences of length $n$ with no adjacent $0$s $= Fib_{n+1}$.
    \item \textbf{Comb. with Repetition:} Choose $k$ from $n$ elements with replacement $= \binom{n+k-1}{k}$.
    \item \textbf{Equal Group Division:} Ways to divide $n$ into $n/k$ groups of size $k$: $\frac{n!}{(k!)^{n/k} (n/k)!}$.
    \item \textbf{Integer Solutions:} Non-negative solutions to $x_1+\dots+x_k=n$ is $\binom{n+k-1}{n}$.
    \item \textbf{Separated Selection:} Choose $n$ ids from $b$ with dist $\ge k$: $\binom{b-(n-1)(k-1)}{n}$.

    % --- Sums & Transformations ---
    \item \textbf{Alternating Sum:} $\sum_{i \text{ odd}}^n \binom{n}{i} a^{n-i}b^i = \frac{1}{2} ((a+b)^n-(a-b)^n)$.
    \item \textbf{Quotient Sum:} $\sum_{i=0}^{n} \frac{\binom{k}{i}}{\binom{n}{i}} = \frac{\binom{n+1}{n-k+1}}{\binom{n}{k}}$.
    \item \textbf{Finite Difference:} If $x_{i+1}$ is sum of prev row $n$ times, $n$-th row first col is $p(n)=\sum_{k=0}^{n}\binom{n}{k} x(k)$.
    \item \textbf{Binomial Inversion I:} $P(n)=\sum_{k=0}^{n}\binom{n}{k} Q(k) \iff Q(n)=\sum_{k=0}^{n}(-1)^{n-k}\binom{n}{k} P(k)$.
    \item \textbf{Binomial Inversion II:} $P(n)=\sum_{k=0}^{n}(-1)^{k}\binom{n}{k} Q(k) \iff Q(n)=\sum_{k=0}^{n}(-1)^{k}\binom{n}{k} P(k)$.

    % --- Permutations ---
    \item \textbf{Derangements:} $d(n)=(n-1)(d(n-1)+d(n-2))$ with $d(0)=1, d(1)=0$.
    \item \textbf{Involutions:} Permutations where $p^2=id$. $a_n=a_{n-1} + (n-1)a_{n-2}$.
    \item \textbf{Restricted Cycles $T(n,k)$:} Permutations size $n$ with all cycles $\le k$:
    \[ T(n, k) = \begin{cases} n! & n \le k \\ n T(n-1, k) - \frac{n!}{(n-k)!} T(n-k-1, k) & n > k \end{cases} \]
\end{itemize}
}

\section*{Template \& Utils}

\subsection*{PBDS (Ordered Set \& Hash Map)}
\begin{lstlisting}
#include <ext/pb_ds/assoc_container.hpp>
#include <ext/pb_ds/tree_policy.hpp>
using namespace __gnu_pbds;
using orderS = tree<ll,null_type,less<ll>,rb_tree_tag,tree_order_statistics_node_update>;

struct custom_hash {
    static uint64_t splitmix64(uint64_t x) {
        x += 0x9e3779b97f4a7c15;
        x = (x ^ (x >> 30)) * 0xbf58476d1ce4e5b9;
        x = (x ^ (x >> 27)) * 0x94d049bb133111eb;
        return x ^ (x >> 31);
    }
    size_t operator()(uint64_t x) const {
        static const uint64_t FIXED_RANDOM =
            chrono::steady_clock::now().time_since_epoch().count();
        return splitmix64(x + FIXED_RANDOM);
    }
};
template <typename K, typename V>
using hash_map = gp_hash_table<K, V, custom_hash>;
\end{lstlisting}

\subsection*{Pragmas \& Optimization}
{ \footnotesize
\textbf{Description:} Aggressive GCC optimizations. \texttt{Ofast} ignores strict IEEE floating point standards (be careful with geometry precision).
}
\begin{lstlisting}
#pragma GCC optimize("O3")
#pragma GCC optimize("Ofast,unroll-loops")
#pragma GCC optimize("tree-vectorize")
#pragma GCC target("avx2,sse4.2,popcnt")
\end{lstlisting}

\subsection*{Random Number Generator}
{ \footnotesize
\textbf{Description:} Mersenne Twister (\texttt{mt19937}) seeded with high-resolution clock. Much better than \texttt{rand()}.
}
\begin{lstlisting}
mt19937 rng(chrono::high_resolution_clock::now().time_since_epoch().count());
inline ll getrandom(ll a,ll b) { return uniform_int_distribution<ll>(a,b)(rng); }
\end{lstlisting}

\subsection*{Basic Math Utils}
{ \footnotesize
\textbf{Description:} 
1. \texttt{bigmod}: Modular Exponentiation $\mathcal{O}(\log P)$.
2. \texttt{inversemod}: Modular Inverse using Fermat's Little Theorem (Requires Prime Mod).
3. \texttt{sqrtt}: Integer Square Root (avoids precision errors of \texttt{sqrt}).
}
\begin{lstlisting}
ll bigmod(ll base, ll power) {
    ll res = 1; ll p = base % mod;
    while (power > 0) {
        if (power % 2 == 1) res = ((res % mod) * (p % mod)) % mod;
        power /= 2;
        p = ((p % mod) * (p % mod)) % mod;
    }
    return res;
}
ll inversemod(ll base) { return bigmod(base, mod - 2); }

int gcd(ll a, ll b) {
    while (b) { a %= b; swap(a, b); }
    return a;
}
ll sqrtt(ll a) {
    long long x = sqrt(a) + 2;
    while (x * x > a) x--;
    return x;
}
\end{lstlisting}

\subsection*{Grid Moves (2D)}
\begin{lstlisting}
int dx[]={-1, 1 , 0 , 0 , -1 ,-1, 1, 1};
int dy[]={ 0, 0 ,-1 , 1 , -1 , 1,-1, 1};
constexpr ld PI = 3.14159265358979323846264338327950288L;
\end{lstlisting}

\section*{CP Environment Setup}
\subsection*{C++ Library Header (\texttt{stdc.h})}
{ \footnotesize
\textbf{Description:}  Run \texttt{g++ stdc.h -o stdc.h.gch} once to enable fast precompilation.
}
\subsection*{Base Solution File (\texttt{template.cpp})}
{ \footnotesize
\textbf{Description:} The starting file for every problem. Includes the necessary macros and I/O setup.
}
\begin{lstlisting}[language=C++]
// IIUC_MARK_US
#include "bits/stdc++.h"
using namespace std;

#define sz(x) (int) (x).size()
#define all(x) begin(x), end(x)
#define rep(i, a, b) for (int i = a; i < (b); ++i)
using ll = long long; using pii = pair<int, int>;
using pll = pair<ll, ll>; using vi = vector<int>;
template<class T> using V = vector<T>;

inline void file() {
#ifndef ONLINE_JUDGE  
    freopen("input.txt", "r", stdin);
    freopen("output.txt", "w", stdout);
#endif
}
int main() {
    ios_base::sync_with_stdio(0);
    cin.tie(0); cout.tie(0);
clock_t start= clock();
cerr << "Time: " <<((double)(clock() - start) / CLOCKS_PER_SEC)<<el;
}
\end{lstlisting}

\subsection*{Fast Compile \& Run (\texttt{cf})}
{ \footnotesize
\textbf{Description:} Saves as \texttt{cf}. Setup: \texttt{chmod +x cf}. \\
\textbf{Usage:} Run \texttt{./cf A} (no need for .cpp). Compiles with \textbf{-O2} and \textbf{C++17}, runs against \texttt{input.txt}, and shows execution time.
}
\begin{lstlisting}[language=Bash]
#!/bin/bash
g++ -o sol -Wall -Wextra -std=c++17 -O2 $1.cpp
if [ $? -eq 0 ]; then
    time ./sol < input.txt
fi
\end{lstlisting}

\subsection*{Runtime Error Check (\texttt{rte})}
{ \footnotesize
\textbf{Description:} Saves as \texttt{rte}. Setup: \texttt{chmod +x rte}. \\
\textbf{Usage:} Run \texttt{./rte A} (no need for .cpp). Compiles with \textbf{AddressSanitizer} and \textbf{UBSan} to catch out-of-bounds, overflows, and memory leaks.
}
\begin{lstlisting}[language=Bash]
#!/bin/bash
g++ -o sol -std=c++17 -O2 -fsanitize=address,undefined $1.cpp
if [ $? -eq 0 ]; then
    ./sol < input.txt
fi
\end{lstlisting}
\subsection{Stress Test Script (run.sh)}
{ \footnotesize
\textbf{Description:} Bash script to compare your solution against a brute force solution using a generator. Stops on the first mismatch. \\
\textbf{Usage:} Save as \texttt{run.sh}, give permission (\texttt{chmod +x run.sh}), and run (\texttt{./run.sh}).
}
\begin{lstlisting}[language=bash]
set -e
g++ code.cpp -o code
g++ gen.cpp -o gen
g++ brute.cpp -o brute

for((i=1;; ++i)); do
    ./gen $i > input_file
    ./code < input_file > myAnswer
    ./brute < input_file > correctAnswer

    # -Z ignores trailing whitespace
    diff -Z myAnswer correctAnswer > /dev/null || break
    echo "Passed test: " $i
done

echo "WA on the following test:"
cat input_file
echo "Your answer is:"
cat myAnswer
echo "Correct answer is:"
cat correctAnswer
\end{lstlisting}

\subsection*{Generator (gen.cpp)}
\begin{lstlisting}
#include <bits/stdc++.h>
using namespace std;
mt19937 rng; 
long long rand(long long l, long long r) {
    uniform_int_distribution<long long> dist(l, r);
    return dist(rng);
}
int main(int argc, char* argv[]) {
    // Seed rng with test case number
    rng.seed(atoi(argv[1])); 
    int n = rand(1, 10); 
    cout << n << endl;
    for(int i = 0; i < n; i++) {
        cout << rand(1, 100) << (i == n-1 ? "" : " ");
    }
    cout << endl;
}
\end{lstlisting}

\end{multicols*}
\end{document}
