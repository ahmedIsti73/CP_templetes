\documentclass[8pt,a4paper,landscape]{extarticle} % 8pt base font for 4-column fit

% --- PACKAGES ---
\usepackage[utf8]{inputenc}
\usepackage[T1]{fontenc}
\usepackage[english]{babel}
\usepackage{graphicx}
\usepackage{amsmath}
\usepackage{amssymb}
\usepackage{amsfonts}
\usepackage{extsizes} % Guarantees 8pt base font size works
\usepackage{microtype} % Improves text density and justification
\usepackage{multicol} % For multi-column layout
\usepackage{xcolor}
\usepackage{fancyhdr}
\usepackage{enumitem} % For compact lists
\usepackage{titlesec} % For custom section formatting
\usepackage{listings} % For code formatting

% --- GEOMETRY: 4 COLUMNS CONFIGURATION ---
\usepackage[top=0.3in, bottom=0.2in, left=0.2in, right=0.2in, includehead]{geometry}

\setlength{\columnsep}{0.2in}
\setlength{\columnseprule}{0.2pt}

% --- HEADER CONFIGURATION ---
\pagestyle{fancy}
\fancyhf{}
\renewcommand{\headrulewidth}{0pt}
\lhead{\scriptsize \textcolor{blue}{\textbf{International Islamic University Chittagong}}}
\chead{\scriptsize \textcolor{blue}{\textbf{| \quad IIUC\_MARK\_US \quad |}}}
\rhead{\thepage} % Added page number for reference

% --- COMPRESSION & LAYOUT TWEAKS ---
\setlength{\parindent}{0pt}
\setlength{\parskip}{0pt} % Remove paragraph skip completely
\setlist[itemize]{nosep, leftmargin=*} % Ultra-compact item lists
\setlist[enumerate]{nosep, leftmargin=*} % Ultra-compact numbered lists

% --- TITLE FORMATTING (Visual Separator for Algorithms) ---
\definecolor{darkgray}{rgb}{0.3,0.3,0.3}
\titleformat{\subsection}
    {\color{darkgray}\normalfont\bfseries}
    {}
    {0em}
    {}
    [\color{darkgray}\titlerule] % Adds a horizontal rule under the title
\titlespacing*{\subsection}{0pt}{5pt}{3pt} % Minimal vertical spacing

% --- CODE LISTING STYLE (Final Optimized Version) ---
\definecolor{codegreen}{rgb}{0,0.6,0}
\definecolor{codegray}{rgb}{0.5,0.5,0.5}
\definecolor{codepurple}{rgb}{0.58,0,0.82}
\definecolor{backcolour}{rgb}{0.98,0.98,0.98}
\lstset{
    language=C++,
    backgroundcolor=\color{backcolour},
    keywordstyle=\bfseries\color{magenta},
    commentstyle=\itshape\color{codegreen},
    stringstyle=\color{codepurple},
    numberstyle=\tiny\color{codegray},
    basicstyle=\fontsize{7}{8}\selectfont\ttfamily,
    tabsize=2,
    breaklines=true,
    breakatwhitespace=false,
    showstringspaces=false,
    numbers=none,
    frame=t, % Only top frame
    aboveskip=2pt, % Minimal vertical space
    belowskip=2pt,
    xleftmargin=0pt,
    postbreak=\mbox{\textcolor{red}{$\hookrightarrow$}\space}, % Arrow for wrapped lines
    columns=fixed
}
% --- DOCUMENT START ---
\begin{document}

% --- FRONT PAGE ---
\begin{titlepage}
    \centering
    \vspace*{\fill}
    
    \includegraphics[height=4cm]{iiuclogo.png} 
    \\[1cm]
    
    {\Huge \textbf{\textcolor{blue}{International Islamic University Chittagong}}} 
    \\[1.5cm]
    
    {\fontsize{40}{48}\selectfont \textbf{IIUC\_MARK\_US}} 
    \\[1cm]
    
    {\LARGE \textbf{Arman, Istiaque, Mizan}}
    
    \vspace*{\fill}
\end{titlepage}
\newpage

% --- 4 COLUMN LAYOUT START ---
\begin{multicols*}{4}

\section*{Mathematics}

% --- SECTION: Equations ---
\subsection*{Equations}
The extremum of a quadratic is given by $x = -b/2a$.

\textbf{Cramer's Rule}: Given an equation $Ax = b$, the solution to a variable $x_i$ is given by
\begin{flalign*}
x_i &= \frac{\det A_i'}{\det A} &&  
   \text{\parbox{2.5cm}{\tiny [where $A_i'$ is $A$ with the $i$'th column replaced by $b$.]}}
\end{flalign*}
\textbf{Example (3x3):}
{\tiny
\begin{align*}
    2x + 3y - 5z &= 1 \\
    x + y - z &= 2 \\
    2y + z &= 8
\end{align*}
$D = \begin{vmatrix} 2 & 3 & -5 \\ 1 & 1 & -1 \\ 0 & 2 & 1 \end{vmatrix} = -7$
$D_x = \begin{vmatrix} 1 & 3 & -5 \\ 2 & 1 & -1 \\ 8 & 2 & 1 \end{vmatrix} = -7$
$D_y = \begin{vmatrix} 2 & 1 & -5 \\ 1 & 2 & -1 \\ 0 & 8 & 1 \end{vmatrix} = -21$
$D_z = \begin{vmatrix} 2 & 3 & 1 \\ 1 & 1 & 2 \\ 0 & 2 & 8 \end{vmatrix} = 14$
$x = \frac{D_x}{D} = 1$, $y = \frac{D_y}{D} = 3$, $z = \frac{D_z}{D} = -2$
}

\textbf{Vieta's Formulas}: Let $P(x) = a_nx^n +...+a_0$, be a polynomial with complex coefficients and degree $n$, having complex roots $r_n,...,r_1$. Then for any integer $0 \leq k \leq n$,
\[
  \sum_{1 \leq i_1 < i_2 < ... < i_k \leq n} r_{i_1}r_{i_2}...r_{i_k} = (-1)^k\frac{a_{n-k}}{a_n}
\]
\textbf{Rational Root Theorem}: If $\frac{p}{q}$ is a reduced rational root of a polynomial with \textbf{integer coeffs}, then $p \mid a_0$ and $q \mid a_n$.

% --- SECTION: Number Theory ---
\subsection*{Number Theory}
\textbf{Sum of Divisors (S.O.D):}
If $N = a^p \cdot b^q \cdot c^r \dots$
\[ \text{S.O.D} = \frac{a^{p+1}-1}{a-1} \cdot \frac{b^{q+1}-1}{b-1} \cdot \frac{c^{r+1}-1}{c-1} \dots \]

\textbf{Number of Divisors (N.O.D):}
If $N = a^p \cdot b^q \cdot c^r \dots$
\[ \text{N.O.D} = (p+1)(q+1)(r+1) \dots \]

% --- NEW: Product of Divisors ---
\textbf{Product of Divisors (P.O.D):}
If $N$ has $D = \text{N.O.D}(N)$ divisors:
\[ \text{P.O.D}(N) = N^{D/2} = (\sqrt{N})^D \]

\textbf{Euclidean Algorithm Property:}
\[ \gcd(a, b) = \gcd(a, a - b) \quad [a > b] \]

\textbf{Fibonacci GCD:}
\[ \gcd(F(a), F(b)) = F(\gcd(a,b)) \]

\textbf{Euler's Totient Theorem:}
\[ a^{\phi(n)} \equiv 1 \pmod{n} \]
where $\phi(n)$ is Euler's Totient Function.

\textbf{Modular Exponentiation:}
\[ a^b \pmod m \equiv a^{b \pmod{\phi(m)}} \pmod m \]
(if $a$ and $m$ are coprime)

\textbf{Primitive roots} modulo $n$ exists iff $n = 1, 2, 4$ or, $n = p^k, 2p^k$ where $p$ is an odd prime. Furthermore, the number of roots are $\phi (\phi (n))$.

\textbf{To Find Generator} $g$ of $M$, factor $M - 1$ and get the distinct primes $p_i$. If $g^{(M - 1) / p_i } \neq 1 (MOD M)$ for each $p_i$ then $g$ is a valid root. Try all $g$ until a hit is found (usually found very quick).

% --- SECTION: Mobius Function ---
\subsection*{Mobius Function}
{\tiny
\[
    \mu(n) = \begin{cases} 0 & n \textrm{ is not square free}\\ 1 & n \textrm{ has even number of prime factors}\\ -1 & n \textrm{ has odd number of prime factors}\\\end{cases}
\]
}
 Mobius Inversion:
 \[ g(n) = \sum_{d|n} f(d) \Leftrightarrow f(n) = \sum_{d|n} \mu(d)g(n/d) \]
 Other useful formulas/forms:

 $ \sum_{d | n} \mu(d) = [ n = 1] $,
 $ \phi(n) = \sum_{d | n} \mu(d)\frac{n}{d}$

 $ g(n) = \sum_{n|d} f(d) \Leftrightarrow f(n) = \sum_{n|d} \mu(\frac{d}{n})g(d)$

 $ g(n) = \sum_{1 \leq m \leq n} f(\left\lfloor\frac{n}{m}\right \rfloor ) \Leftrightarrow f(n) = \sum_{1\leq m\leq n} \mu(m)g(\left\lfloor\frac{n}{m}\right\rfloor)$

 If $f$ multiplicative, $\sum_{d | n} \mu(d)f(d) = \prod_{\textnormal{prime } p | n}(1 - f(p))$ and $\sum_{d | n}\mu^2(d)f(d) = \prod_{\textnormal{prime } p | n} (1 + f(p))$. 

 If $s_f(n) = \sum_{i = 1}^{n} f(i)$ is a prefix sum of mulitplicative $f$ then $s_{f * g}(n) = \sum_{1 \leq xy \leq n}f(x)g(y)$. Then $s_f(n) = \{s_{f * g}(n) - \sum_{d = 2}^{n} s_f(\lfloor n / d \rfloor) g(d)\} / g(1)$ where $f * g(n) = \sum_{d | n} f(d)g(n / d)$ (Dirichlet). Precompute (linear sieve) $O(n^{2/3})$ first values of $s_f$ for complexity $O(n^{2/3})$.

 Useful sums and convolutions: $\epsilon = \mu * \textnormal{\textbf{1}}$, id = $\phi * \textnormal{\textbf{1}}$, id = $g * \textnormal{id}_2$, where $\epsilon(n) = [n = 1]$, $\textnormal{\textbf{1}}(n) = 1$, id$(n) = n$, id$_k(n) = n^k$, $g(n) = \sum_{d | n}\mu(d)nd$.\\
 coprime pairs in $[1,n]$ is $\sum_{d = 1}^{n}\mu(d)\lfloor n / d \rfloor ^2$. Sum of GCD pairs in $[1, n]$ is $\sum_{d = 1}^{n}\phi(d)\lfloor n / d \rfloor ^2$. Sum of LCM pairs in $[1, n]$ is $\sum_{d = 1}^{n}(\frac{\lfloor n / d \rfloor (1 + \lfloor n / d \rfloor)}{2})^2 g(d)$, where $g$ is defined above with $g(p^k) = p^k - p^{k+1}$.

% --- UPDATED SECTION: Partition Function ---
\subsection*{Partition Function: $p(n)$}
\textbf{Pattern:} Form a sum $n$ where the \textbf{order does not matter}.
\begin{itemize}
    \item "How many ways to write $n$ as a sum of positive integers?"
    \item "How many ways to put $n$ *identical* balls into *identical* boxes?"
\end{itemize}
\hrule
\textbf{Definition:} Number of ways of writing $n$ as a sum of positive integers, disregarding order.
\textbf{Sequence $p(n)$ for $n=0, 1, 2, \dots$:}
\[
1, 1, 2, 3, 5, 7, 11, 15, 22, 30, 42, 56, 77, \dots
\]
\textbf{Recurrence (Pentagonal Number Theorem):}
\begin{align*}
p(n) &= \sum_{k \in \mathbb{Z} \setminus \{0\}} (-1)^{k-1} p(n - k(3k-1)/2) \\
     &= p(n-1) + p(n-2) - p(n-5) - p(n-7) + \dots \\
\end{align*}

% --- SECTION: Ceils and Floors ---
\subsection*{Ceils and Floors}
For $x, y \in \mathbb{R}$, $m, n \in \mathbb{Z}$:
{\tiny
\begin{itemize}
  \item $\lfloor x \rfloor \leq x < \lfloor x \rfloor + 1\text{;    }\lceil x \rceil - 1 < x \leq \lceil x \rceil$
  \item $- \lfloor x \rfloor = \lceil -x \rceil \text{;    } - \lceil x \rceil = \lfloor -x \rfloor$
  \item $\lfloor x + n \rfloor = \lfloor x \rfloor + n \text{, } \lceil x + n \rceil = \lceil x \rceil + n$
  \item $\lfloor x \rfloor = m \Leftrightarrow x - 1 < m \leq x < m + 1$
  \item $\lceil x \rceil = n \Leftrightarrow n - 1 < x \leq n < x + 1$
  \item If $n > 0$, $\lfloor \frac{\lfloor x \rfloor + m}{n} \rfloor = \lfloor \frac{x + m}{n} \rfloor$
  \item If $n > 0$, $\lceil \frac{\lceil x \rceil + m}{n} \rceil = \lceil \frac{x + m}{n} \rceil$
  \item If $n > 0$, $\lfloor \frac{\lfloor \frac{x}{m} \rfloor }{n} \rfloor = \lfloor \frac{x}{mn} \rfloor$
  \item If $n > 0$, $\lceil \frac{\lceil \frac{x}{m} \rceil }{n} \rceil = \lceil \frac{x}{mn} \rceil$
  \item For $m, n > 0$, $\sum_{k = 1}^{n - 1} \lfloor \frac{km}{n} \rfloor = \frac{(m - 1)(n - 1) + \gcd(m, n) - 1}{2}$
  \item $\lfloor n/j \rfloor = x \text{ for } j \in [\lfloor n/(x+1) \rfloor + 1, \lfloor n/x \rfloor]$
  \item Modulo definition: $a \pmod m = a - m \lfloor a/m \rfloor$
\end{itemize}
}

% --- SECTION: Recurrences ---
\subsection*{Recurrences}
If $a_n = c_1 a_{n-1} + \dots + c_k a_{n-k}$, and $r_1, \dots, r_k$ are distinct roots of $x^k - c_1 x^{k-1} - \dots - c_k$, there are $d_1, \dots, d_k$ s.t.
\[a_n = d_1r_1^n + \dots + d_kr_k^n. \]
Non-distinct roots $r$ become polynomial factors, e.g. $a_n = (d_1n + d_2)r^n$.

% --- SECTION: Trigonometry ---
\subsection*{Trigonometry}
{\tiny
\begin{align*}
\sin(v+w)&{}=\sin v\cos w+\cos v\sin w\\
\cos(v+w)&{}=\cos v\cos w-\sin v\sin w\\
\tan(v+w)&{}=\dfrac{\tan v+\tan w}{1-\tan v\tan w}\\
\sin v+\sin w&{}=2\sin\dfrac{v+w}{2}\cos\dfrac{v-w}{2}\\
\cos v+\cos w&{}=2\cos\dfrac{v+w}{2}\cos\dfrac{v-w}{2}
\end{align*}
}
\[ (V+W)\tan(\frac{v-w}{2}){}=(V-W)\tan(\frac{v+w}{2}) \]
$V, W$ are sides opposite to angles $v, w$.
   $a\cos x+b\sin x=r\cos(x-\phi)$\\
   $a\sin x+b\cos x=r\sin(x+\phi)$\\
where $r=\sqrt{a^2+b^2}, \phi=\operatorname{atan2}(b,a)$.

% --- SECTION: Geometry ---
\subsection*{Geometry}

\subsubsection*{Rectangles and Squares}
\begin{itemize}
    \item Area of a rectangle: $A = l \cdot w$
    \item Perimeter of a rectangle: $P = 2l + 2w$
    \item Diagonal of a rectangle: $d = \sqrt{l^2 + w^2}$
    \item Area of a square: $A = \text{side}^2$
    \item Perimeter of a square: $P = 4 \cdot \text{side}$
    \item Diagonal of a square: $d = \sqrt{2} \cdot \text{side}$
\end{itemize}

\subsubsection*{Triangles}
Side lengths: $a,b,c$; Semiperimeter: $p=\dfrac{a+b+c}{2}$
\begin{itemize}
    \item Area: $A = \frac{1}{2} \cdot b \cdot h$
    \item Perimeter: $P = a + b + c$
    \item Heron's Area: $A=\sqrt{p(p-a)(p-b)(p-c)}$
    \item Circumradius: $R=\dfrac{abc}{4A}$
    \item Inradius: $r=\dfrac{A}{p}$
    \item Length of median: $m_a=\tfrac{1}{2}\sqrt{2b^2+2c^2-a^2}$
    \item Length of bisector: $s_a=\sqrt{bc\left[ 1 - (a / (b+c))^2 \right]}$
    \item Law of Sines: $\dfrac{\sin\alpha}{a}=\dfrac{\sin\beta}{b}=\dfrac{\sin\gamma}{c}=\dfrac{1}{2R}$
    \item Law of Cosines: $a^2=b^2+c^2-2bc\cos\alpha$
    \item Law of Tangents: $\dfrac{a+b}{a-b}=\dfrac{\tan((\alpha + \beta)/2)}{\tan((\alpha - \beta)/2)}$
\end{itemize}

\subsubsection*{Circles}
\begin{itemize}
    \item Area: $A = \pi \cdot r^2$
    \item Circumference: $C = 2 \pi \cdot r$
    \item Sector Area: $A_{\text{sector}} = \frac{\theta}{360^\circ} \cdot \pi \cdot r^2$ (in degrees)
    \item Arc Length: $l = \frac{\theta}{360^\circ} \cdot 2 \pi \cdot r$ (in degrees)
\end{itemize}

\subsubsection*{Polygons (n-sided)}
\begin{itemize}
    \item Sum of interior angles: $(n-2) \times 180^\circ$
    \item A single angle (regular): $\frac{(n-2) \times 180^\circ}{n}$
    \item Amount of diagonals: $\frac{n(n-3)}{2}$
    \item Sum of exterior angles: $360^\circ$
    \item Area (regular): $\frac{1}{4} n s^2 \cot(\frac{\pi}{n})$
    \item Area (with apothem): $\frac{1}{2} \cdot n \cdot s \cdot a$
\end{itemize}

\subsubsection*{3D Shapes}
\begin{itemize}
    \item \textbf{Cube:} Volume $V = s^3$, Surface Area $SA = 6s^2$
    \item \textbf{Sphere:} Volume $V = \frac{4}{3}\pi r^3$, Surface Area $SA = 4\pi r^2$
    \item \textbf{Cylinder:} Volume $V = \pi r^2 h$, Surface Area $SA = 2\pi r^2 + 2\pi rh$
    \item \textbf{Cone:} Volume $V = \frac{1}{3}\pi r^2 h$, Surface Area $SA = \pi r s + \pi r^2$, where $s = \sqrt{h^2+r^2}$
    \item \textbf{Cuboid:} Volume $V = lwh$, Surface Area $SA = 2(wh + lw + lh)$
\end{itemize}

\subsubsection*{Quadrilaterals}
With side lengths $a,b,c,d$, diagonals $e, f$, diagonals angle $\theta$, area $A$ and
magic flux $F=b^2+d^2-a^2-c^2$:
\[ 4A = 2ef \cdot \sin\theta = F\tan\theta = \sqrt{4e^2f^2-F^2} \]
 For cyclic quadrilaterals the sum of opposite angles is $180^\circ$,
$ef = ac + bd$, and $A = \sqrt{(p-a)(p-b)(p-c)(p-d)}$

\subsubsection*{Pick's Theorem}
For a polygon on a grid:
\[ A = I + \frac{B}{2} - 1 \]
$A$ = Area, $I$ = Interior points, $B$ = Boundary points.

\subsubsection*{Spherical coordinates}
{\tiny
\[\begin{array}{cc}
x = r\sin\theta\cos\phi & r = \sqrt{x^2+y^2+z^2}\\
y = r\sin\theta\sin\phi & \theta = \textrm{acos}(z/\sqrt{x^2+y^2+z^2})\\
z = r\cos\theta & \phi = \textrm{atan2}(y,x)
\end{array}\]
}

\subsection*{Coordinate Geometry}
{\tiny
\begin{itemize}
    \item \textbf{Distance (2 points):} $(x_1, y_1), (x_2, y_2)$
    $D = \sqrt{(x_2-x_1)^2 + (y_2-y_1)^2}$
    \item \textbf{Midpoint:} $M = \left(\frac{x_1+x_2}{2}, \frac{y_1+y_2}{2}\right)$
    \item \textbf{Slope (2 points):} $m = \frac{y_2-y_1}{x_2-x_1}$
    \item \textbf{Line (point-slope):} $y - y_1 = m(x - x_1)$
    \item \textbf{Line (slope-intercept):} $y = mx + b$
    \item \textbf{Line (two-point):} $y - y_1 = \frac{y_2-y_1}{x_2-x_1}(x - x_1)$
    \item \textbf{Line (general):} $Ax + By + C = 0$
    \item \textbf{Slope (from general):} $m = -A/B$
    \item \textbf{Parallel lines:} have the same slope ($m_1 = m_2$)
    \item \textbf{Perpendicular lines:} $m_1 = -1/m_2$
    \item \textbf{Distance (point to line):} Point $(x_0, y_0)$ to line $Ax+By+C=0$. $D = \frac{|Ax_0+By_0+C|}{\sqrt{A^2+B^2}}$
    \item \textbf{Area of Triangle (vertices):} $(x_1, y_1), (x_2, y_2), (x_3, y_3)$
    $A = \frac{1}{2} |x_1(y_2-y_3) + x_2(y_3-y_1) + x_3(y_1-y_2)|$
    \item \textbf{Circle:} Center $(h, k)$, radius $r$. $(x-h)^2 + (y-k)^2 = r^2$
    \item \textbf{Distance (2 circle centers):} $D = \sqrt{(h_2-h_1)^2 + (k_2-k_1)^2}$
    \item \textbf{Tangent slope on circle:} At point $(x_0, y_0)$ on circle $x^2+y^2=r^2$. $m = -x_0/y_0$
    \item \textbf{Area of Parallelogram (vertices):} $(x_1, y_1), \dots, (x_4, y_4)$
    $A = |x_1y_2 + x_2y_3 + x_3y_4 + x_4y_1 - x_2y_1 - x_3y_2 - x_4y_3 - x_1y_4|$
    \item \textbf{Ellipse:} $\frac{x^2}{a^2} + \frac{y^2}{b^2} = 1$
    \item \textbf{Hyperbola:} $\frac{x^2}{a^2} - \frac{y^2}{b^2} = 1$
    \item \textbf{Parabola:} Vertex $(h, k)$, focus $(h+p, k)$. $(x-h) = 4p(y-k)$
\end{itemize}
}

\subsection*{Derivatives/Integrals}
{\tiny
\begin{align*}
  \dfrac{d}{dx}\arcsin x = \dfrac{1}{\sqrt{1-x^2}} &&& \dfrac{d}{dx}\arccos x = -\dfrac{1}{\sqrt{1-x^2}} \\
  \dfrac{d}{dx}\tan x = 1+\tan^2 x &&& \dfrac{d}{dx}\arctan x = \dfrac{1}{1+x^2} \\
  \int\tan ax = -\dfrac{\ln|\cos ax|}{a} &&& \int xe^{ax}dx = \frac{e^{ax}}{a^2}(ax-1) \\
  \int e^{-x^2} = \frac{\sqrt \pi}{2} \text{erf}(x) &&& \int x\sin ax = \frac{\sin ax-ax \cos ax}{a^2}
\end{align*}}
\[\int_a^bf(x)g(x)dx = [F(x)g(x)]_a^b-\int_a^bF(x)g'(x)dx\]

\subsection*{Sums}
\subsubsection*{Basic Sums}
\begin{itemize}
    \item $\sum_{i=1}^{n} 1 = n$
    \item $\sum_{i=1}^{n} i = \frac{n(n+1)}{2}$
    \item $\sum_{i=1}^{n} i^2 = \frac{n(n+1)(2n+1)}{6}$
    \item $\sum_{i=1}^{n} i^3 = \left(\frac{n(n+1)}{2}\right)^2 = \frac{n^2(n+1)^2}{4}$
    \item $\sum_{i=1}^{n} i^4 = \frac{n(n+1)(2n+1)(3n^2+3n-1)}{30}$
    \item Sum of first $n$ odd: $\sum_{i=1}^{n} (2i-1) = n^2$
    \item Sum of first $n$ even: $\sum_{i=1}^{n} 2i = n(n+1)$
\end{itemize}

\subsubsection*{Arithmetic Progression (AP)}
$a_n = a_1 + (n-1)d$
$S_n = \frac{n}{2}(2a_1 + (n-1)d) = \frac{n}{2}(a_1 + a_n)$
$a_n = a_m + (n-m)d$

\subsubsection*{Geometric Progression (GP)}
$a_n = a_1 r^{(n-1)}$
$S_n = \frac{a_1(r^n - 1)}{r-1}$ (finite)
$S_\infty = \frac{a_1}{1-r}$ (for $|r|<1$)
$P_n = a_1^n r^{n(n-1)/2}$
$c^a + c^{a+1} + \dots + c^{b} = \frac{c^{b+1} - c^a}{c-1}, c \neq 1$

\subsubsection*{Bernoulli Numbers \& Sum of Powers}
\textbf{Pattern:} Compute $\sum_{i=1}^n i^k$ where \textbf{$n$ is large} but \textbf{$k$ is small}.
\begin{itemize}
    \item "Find $(1^5 + 2^5 + \dots + n^5) \pmod{10^9+7}$ for $n=10^{18}$."
\end{itemize}
\textbf{Sequence $B_k$ for $k=0, 1, 2, \dots$:}
\[
1, \frac{1}{2}, \frac{1}{6}, 0, -\frac{1}{30}, 0, \frac{1}{42}, 0, -\frac{1}{30}, \dots
\]
\textit{(Note: Using $B_1 = +1/2$. The $B_1 = -1/2$ convention also exists.)}

\textbf{EGF for $B_k$ (using $B_1 = -1/2$):}
\[
\frac{x}{e^x - 1} = \sum_{k=0}^\infty B_k \frac{x^k}{k!}
\]
\textbf{Faulhaber's Formula (Sum of Powers):}
\[
\sum_{i=0}^{n-1} i^m = \frac{1}{m+1} \sum_{k=0}^m \binom{m+1}{k} B_k n^{m+1-k}
\]

\section*{Combinatorics}

\subsection*{Binomial Theorem}
\textbf{Description:} Used for expanding powers of binomials $(a+b)^p$. The coefficients $\binom{p}{k}$ give the number of ways to choose $k$ items from $p$.

\textbf{Formula:}
\[ (a+b)^p = \sum_{k=0}^{p} \binom{p}{k} a^k b^{p-k} \]

\subsection*{Stars and Bars}
\textbf{Description:} Used to find the number of ways to distribute \textbf{identical (unlabeled)} objects ($n$) into \textbf{distinct} bins ($k$).

\textbf{Formulas:}
\begin{itemize}[leftmargin=*, nosep]
    \item \textbf{Empty bins NOT valid (Positive Integer Solutions):} $\binom{n-1}{k-1}$
    \item \textbf{Empty bins VALID (Non-Negative Integer Solutions):} $\binom{n+k-1}{k-1}$
\end{itemize}

\subsection*{Binomial Coefficients $\binom{n}{k}$}
\textbf{Description:} $\binom{n}{k}$ is the number of ways to choose $k$ elements from $n$ distinct elements. Essential for DP, probability, and modular arithmetic.

{\small
\begin{itemize}[leftmargin=*, nosep]
    % --- Core Definition & Computation ---
    \item \textbf{Definition:} $\binom{n}{k} = \frac{n!}{k!(n-k)!}$
    \item \textbf{Symmetry:} $\binom{n}{k} = \binom{n}{n-k}$
    \item \textbf{Multiplicative ($\mathcal{O}(k)$):} $\binom{n}{k} = \prod_{i=1}^k \frac{n-i+1}{i}$
    \item \textbf{Base Cases:} $\binom{n}{0} = 1, \quad \binom{n}{n} = 1$
    
    % --- Identities (Recurrence & Addition) ---
    \item \textbf{Pascal's Identity ($\mathcal{O}(1)$ DP):} $\binom{n}{k} = \binom{n-1}{k} + \binom{n-1}{k-1}$
    \item \textbf{Absorption Identity:} $\binom{n}{k} = \frac{n}{k} \binom{n-1}{k-1}$
    \item \textbf{Shifted Recurrence I:} $\binom{n}{k} = \frac{n-k+1}{k} \binom{n}{k-1}$
    \item \textbf{Shifted Recurrence II:} $\binom{n+1}{k} = \frac{n+1}{n-k+1} \binom{n}{k}$
    \item \textbf{Vandermonde's Identity:} $\binom{m+n}{r} = \sum_{k=0}^r \binom{m}{k} \binom{n}{r-k}$
    \item \textbf{Hockey-Stick Identity:} $\sum_{i=k}^{n} \binom{i}{k} = \binom{n+1}{k+1}$
    
    % --- Summation Formulas ---
    \item \textbf{Sum of Row (Total Subsets):} $\sum_{k=0}^{n} \binom{n}{k} = 2^n$
    \item \textbf{Sum of K (Weighted Sum):} $\sum_{k=1}^n k \binom{n}{k} = n 2^{n-1}$
    \item \textbf{Sum of $K^2$ (Weighted Sum II):} $\sum_{k=1}^n k^2 \binom{n}{k} = n(n+1) 2^{n-2}$
\end{itemize}
}

\subsection*{Stirling Numbers of the First Kind: $c(n, k)$}
\textbf{Pattern:} Count permutations in terms of their \textbf{cycle structure}.
\begin{itemize}
    \item "Arrange $n$ people around $k$ identical round tables."
    \item "Count permutations of $n$ elements with exactly $k$ cycles."
\end{itemize}
\textbf{Definition:} Number of permutations of $n$ items with $k$ cycles.
\begin{align*}
c(n, k) &= (n-1)c(n-1, k) + c(n-1, k-1) \\
c(n, 0) &= 0 \quad (n > 0), \quad c(0, 0) = 1 \\
\sum_{k=0}^n c(n, k)x^k &= x(x+1)\dots(x+n-1)
\end{align*}
\textbf{Sequence $c(n, 2)$ for $n=0, 1, 2, \dots$:}
\[
0, 0, 1, 3, 11, 50, 274, 1764, 13068, \dots
\]

\subsection*{Stirling Numbers of the Second Kind: $S(n, k) \text{ or } \genfrac{\{}{\}}{0pt}{}{n}{k}$}
\textbf{Pattern:} Partition $n$ \textbf{distinct items} into $k$ \textbf{identical, non-empty boxes}.
\begin{itemize}
    \item "How many ways to put $n$ *labeled* balls into $k$ *unlabeled* boxes?"
    \item "Count ways to partition a set of $n$ elements into $k$ non-empty subsets."
\end{itemize}
\textbf{Definition:} Number of partitions of $n$ distinct elements into exactly $k$ non-empty subsets.
\begin{align*}
S(n, k) &= S(n-1, k-1) + k \cdot S(n-1, k) \\
S(n, 1) &= 1, \quad S(n, n) = 1 \\
S(n, k) &= \frac{1}{k!} \sum_{j=0}^k (-1)^{k-j} \binom{k}{j} j^n
\end{align*}

\subsection*{Bell Numbers: $B(n)$}
\textbf{Pattern:} Total ways to partition $n$ \textbf{distinct items} (number of boxes doesn't matter).
\begin{itemize}
    \item "Find the total number of equivalence relations on a set of $n$ elements."
    \item "How many ways to put $n$ *labeled* balls into *unlabeled* boxes?"
\end{itemize}
\textbf{Definition:} Total number of partitions of $n$ distinct elements.
\[
B(n) = \sum_{k=0}^n S(n, k)
\]
\textbf{Sequence $B(n)$ for $n=0, 1, 2, \dots$:}
\[
1, 1, 2, 5, 15, 52, 203, 877, 4140, 21147, \dots
\]
\textbf{Recurrence (P-set construction):}
\[
B(n+1) = \sum_{k=0}^n \binom{n}{k} B(k)
\]

\subsection*{Catalan Numbers: $C_n$}
\textbf{Pattern:} One of the most famous sequences. Look for:
\begin{itemize}
    \item \textbf{Balanced sequences:} "Valid (balanced) parenthesis strings of length $2n$."
    \item \textbf{Recursive splitting:} "Number of full binary trees with $n$ nodes."
    \item \textbf{Non-crossing paths:} "Paths from (0, 0) to (n, n) on a grid that do not go above $y=x$."
    \item \textbf{Polygon triangulation:} "Ways to triangulate a convex polygon with $n+2$ sides."
\end{itemize}
\textbf{Sequence $C_n$ for $n=0, 1, 2, \dots$:}
\[
1, 1, 2, 5, 14, 42, 132, 429, 1430, 4862, 16796, \dots
\]
\textbf{Closed Form:}
\[
C_n = \frac{1}{n+1} \binom{2n}{n} = \binom{2n}{n} - \binom{2n}{n+1}
\]
\textbf{Recurrence Relations:}
\begin{align*}
C_0 &= 1, \quad C_{n+1} = \sum_{i=0}^n C_i C_{n-i} \\
C_0 &= 1, \quad C_{n+1} = \frac{2(2n+1)}{n+2} C_n
\end{align*}

\subsection*{Eulerian Numbers: $E(n, k)$}
\textbf{Pattern:} Count permutations based on their \textbf{"runs"} or \textbf{"ascents/descents"}.
\begin{itemize}
    \item "Count permutations of $\{1, \dots, n\}$ with exactly $k$ ascents ($p_i < p_{i+1}$)."
\end{itemize}
\textbf{Definition:} Number of $n$-permutations with exactly $k$ rises (positions $i$ with $p_i > p_{i-1}$).
\begin{align*}
E(n, k) &= (n-k)E(n-1, k-1) + (k+1)E(n-1, k) \\
E(n, 0) &= E(n, n-1) = 1 \\
E(n, k) &= \sum_{j=0}^{k} (-1)^j \binom{n+1}{j} (k-j+1)^n
\end{align*}

\subsection*{Derangements: $D(n) \text{ or } !n$}
\textbf{Pattern:} The "mixed-up hats" or "secret santa" problem.
\begin{itemize}
    \item "Count permutations of $n$ elements where \textbf{no element is in its original position}."
    \item "Find the number of permutations with \textbf{no fixed points} ($p_i \neq i$ for all $i$)."
\end{itemize}
\textbf{Definition:} Permutations of a set such that no element appears in its original position.
\textbf{Sequence $D(n)$ for $n=0, 1, 2, \dots$:}
\[
1, 0, 1, 2, 9, 44, 265, 1854, 14833, \dots
\]
\textbf{Recurrence Relations:}
\begin{align*}
D(n) &= (n-1) (D(n-1) + D(n-2)) \\
D(n) &= n \cdot D(n-1) + (-1)^n \\
D(n) &= \left\lfloor \frac{n!}{e} + \frac{1}{2} \right\rfloor = \left\lceil \frac{n!}{e} \right\rceil \quad (n \ge 1)
\end{align*}

\subsection*{Burnside's Lemma}
\textbf{Pattern:} Count "distinct" objects under \textbf{symmetry} (rotations, reflections).
\begin{itemize}
    \item "Count distinct ways to color a necklace/bracelet/cube under rotation."
    \item The key is "up to symmetry," "distinct under rotation," etc.
\end{itemize}
\textbf{Definition:} Given a group $G$ of symmetries acting on a set $X$. The number of distinct elements of $X$ up to symmetry (number of orbits) is:
\[
|X/G| = \frac{1}{|G|} \sum_{g \in G} |X^g|
\]
where $X^g = \{x \in X \mid g \cdot x = x\}$ are the elements fixed by $g$.

\textbf{Special Case (Necklaces):}
For $k$ colors and $n$ beads, with $G = \mathbb{Z}_n$ (rotations):
\[
\text{Count} = \frac{1}{n} \sum_{d | n} \phi(d) \cdot k^{n/d}
\]

\subsection*{Permutation Cycles (EGF)}
\textbf{Pattern:} Count permutations where \textbf{cycle lengths are restricted} to a set $S$.
\begin{itemize}
    \item "Count permutations of $n$ elements that consist *only* of cycles of length 2 (involutions)."
\end{itemize}
\textbf{Definition:} Let $g_S(n)$ be the number of $n$-permutations whose cycle lengths all belong to $S$. The Exponential Generating Function (EGF) is:
\[
\sum_{n \ge 0} g_S(n) \frac{x^n}{n!} = \exp \left( \sum_{n \in S} \frac{x^n}{n} \right)
\]

\subsection*{Lucas's Theorem}
\textbf{Pattern:} Compute $\binom{n}{k} \pmod p$ where \textbf{$n, k$ are large} but \textbf{$p$ is a small prime}.
\begin{itemize}
    \item "Calculate $\binom{10^{18}}{10^9} \pmod 7$."
\end{itemize}
\textbf{Definition:} Let $n, m$ be non-negative integers and $p$ a prime.
Write $n$ and $m$ in base $p$:
\begin{align*}
n &= n_k p^k + \dots + n_1 p + n_0 \\
m &= m_k p^k + \dots + m_1 p + m_0
\end{align*}
Then:
\[
\binom{n}{m} \equiv \prod_{i=0}^k \binom{n_i}{m_i} \pmod{p}
\]
(Note: $\binom{a}{b} = 0$ if $a < b$)

\subsection*{Series}
{\small
$$e^x = 1+x+\frac{x^2}{2!}+\frac{x^3}{3!}+\dots,\,(-\infty<x<\infty)$$
$$\ln(1+x) = x-\frac{x^2}{2}+\frac{x^3}{3}-\frac{x^4}{4}+\dots,\,(-1<x\leq1)$$
$$\sqrt{1+x} = 1+\frac{x}{2}-\frac{x^2}{8}+\frac{2x^3}{32}-\frac{5x^4}{128}+\dots,\,(-1\leq x\leq1)$$
$$\sin x = x-\frac{x^3}{3!}+\frac{x^5}{5!}-\frac{x^7}{7!}+\dots,\,(-\infty<x<\infty)$$
$$\cos x = 1-\frac{x^2}{2!}+\frac{x^4}{4!}-\frac{x^6}{6!}+\dots,\,(-\infty<x<\infty)$$
$$(1-x)^{-r} = \sum_{i = 0}^{\infty} \binom{r + i - 1}{i}x^i, (r \in \mathbb{R})$$
}

\subsection*{Bitwise Formulas}
{\small \begin{align*}
  a | b = a \oplus b + a \& b \\
  a \oplus(a \& b) = (a | b) \oplus b &&& a \oplus b = (a \& b) \oplus (a | b) \\
  a + b = a | b + a \& b &&& a + b = a \oplus b + 2(a \& b)
\end{align*}
  $a - b = (a \oplus (a \& b)) - ((a | b) \oplus a) = ((a | b) \oplus b) - ((a | b) \oplus a) = (a \oplus (a \& b)) - (b \oplus (a \& b)) = ((a | b) \oplus b) - (b \oplus (a \& b)) $
}

\subsection*{Algorithms}
\textbf{Rotation of a n*m matrix:}
$(i,j) \to (j, n-i-1) \to (n-i-1, m-j-1) \to (m-j-1, i)$

\subsection*{Probability theory}
Let $X$ be a discrete random variable with probability $p_X(x)$ of assuming the value $x$. It will then have an expected value (mean) $\mu=\mathbb{E}(X)=\sum_xxp_X(x)$ and variance $\sigma^2=V(X)=\mathbb{E}(X^2)-(\mathbb{E}(X))^2=\sum_x(x-\mathbb{E}(X))^2p_X(x)$ where $\sigma$ is the standard deviation. If $X$ is instead continuous it will have a probability density function $f_X(x)$ and the sums above will instead be integrals with $p_X(x)$ replaced by $f_X(x)$.

Expectation is linear:
\[\mathbb{E}(aX+bY) = a\mathbb{E}(X)+b\mathbb{E}(Y)\]
For independent $X$ and $Y$, \[V(aX+bY) = a^2V(X)+b^2V(Y).\]

\subsubsection*{Discrete distributions}
\textbf{Binomial distribution:}
The number of successes in $n$ independent yes/no experiments, each which yields success with probability $p$ is $\textrm{Bin}(n,p),\,n=1,2,\dots,\, 0\leq p\leq1$.
\[p(k)=\binom{n}{k}p^k(1-p)^{n-k}\]
\[\mu = np,\,\sigma^2=np(1-p)\]
$\textrm{Bin}(n,p)$ is approximately $\textrm{Po}(np)$ for small $p$.

\textbf{First success distribution:}
The number of trials needed to get the first success in independent yes/no experiments, each which yields success with probability $p$ is $\textrm{Fs}(p),\,0\leq p\leq1$.
\[p(k)=p(1-p)^{k-1},\,k=1,2,\dots\]
\[\mu = \frac1p,\,\sigma^2=\frac{1-p}{p^2}\]

\textbf{Poisson distribution:}
The number of events occurring in a fixed period of time $t$ if these events occur with a known average rate $\kappa$ and independently of the time since the last event is $\textrm{Po}(\lambda),\,\lambda=t\kappa$.
\[p(k)=e^{-\lambda}\frac{\lambda^k}{k!}, k=0,1,2,\dots\]
\[\mu=\lambda,\,\sigma^2=\lambda\]

\subsubsection*{Continuous distributions}
\textbf{Uniform distribution:}
If the probability density function is constant between $a$ and $b$ and 0 elsewhere it is $\textrm{U}(a,b),\,a<b$.
\[f(x) = \left\{
\begin{array}{cl}
\frac{1}{b-a} & a<x<b\\
0 & \textrm{otherwise}
\end{array}\right.\]
\[\mu=\frac{a+b}{2},\,\sigma^2=\frac{(b-a)^2}{12}\]

\textbf{Exponential distribution:}
The time between events in a Poisson process is $\textrm{Exp}(\lambda),\,\lambda>0$.
\[f(x) = \left\{
\begin{array}{cl}
\lambda e^{-\lambda x} & x\geq0\\
0 & x<0
\end{array}\right.\]
\[\mu=\frac{1}{\lambda},\,\sigma^2=\frac{1}{\lambda^2}\]

\textbf{Normal distribution:}
Most real random values with mean $\mu$ and variance $\sigma^2$ are well described by $\mathcal{N}(\mu,\sigma^2),\,\sigma>0$.
\[ f(x) = \frac{1}{\sqrt{2\pi\sigma^2}}e^{-\frac{(x-\mu)^2}{2\sigma^2}} \]
If $X_1 \sim \mathcal{N}(\mu_1,\sigma_1^2)$ and $X_2 \sim \mathcal{N}(\mu_2,\sigma_2^2)$ then
\[ aX_1 + bX_2 + c \sim \mathcal{N}(\mu_1+\mu_2+c,a^2\sigma_1^2+b^2\sigma_2^2) \]

\subsection*{Graph Theory}

\subsubsection*{Cayley's Formula}
\textbf{Pattern:} Count \textbf{spanning trees} on $n$ \textbf{labeled} vertices in a \textbf{complete graph $K_n$}.
\begin{itemize}
    \item "How many trees can be formed using $n$ labeled nodes?"
\end{itemize}
\textbf{Definition:} The number of spanning trees on $n$ labeled vertices (in $K_n$) is $n^{n-2}$.
\textbf{Sequence $n^{n-2}$ for $n=1, 2, 3, \dots$:}
\[
1, 1, 3, 16, 125, 1296, 16807, \dots
\]
\textbf{Generalizations:}
\begin{itemize}
    \item \# with degrees $d_i$: $\frac{(n-2)!}{(d_1-1)!(d_2-1)!\dots(d_n-1)!}$ (Prufer Sequence)
\end{itemize}

\subsubsection*{Kirchhoff's Matrix Tree Theorem}
\textbf{Pattern:} Count \textbf{spanning trees} in a \textbf{general graph $G$} (not complete).
\begin{itemize}
    \item "Given a grid graph, find the number of spanning trees."
\end{itemize}
\textbf{Definition:} Counts spanning trees in a graph $G$.
\begin{enumerate}
    \item Create the \textbf{Laplacian Matrix} $L = D - A$:
    \begin{itemize}
        \item $D$ = Degree Matrix (diagonal, $D_{ii} = \text{deg}(i)$)
        \item $A$ = Adjacency Matrix
    \end{itemize}
    Or, $L_{ij} =
    \begin{cases}
        \deg(i) & \text{if } i = j \\
        -1        & \text{if } i \ne j \text{ and } (i, j) \in E \\
        0         & \text{otherwise}
    \end{cases}$

    \item Remove \textbf{any} row $i$ and \textbf{any} column $j$ to get $L_{i,j}$.
    \item The number of spanning trees is $\det(L_{i,j})$.
\end{enumerate}

\subsubsection*{Erd\H{o}s–Gallai Theorem}
\textbf{Pattern:} Given a sequence of numbers, can it be the \textbf{degree sequence} of a \textbf{simple graph}?
\begin{itemize}
    \item "Is the sequence $d_1, \dots, d_n$ a valid graphic sequence?"
\end{itemize}
\textbf{Definition:} A simple graph with node degrees $d_1 \ge \dots \ge d_n$ exists iff:
\begin{enumerate}
    \item $\sum_{i=1}^n d_i$ is even.
    \item For every $k \in [1, n]$:
    \[
    \sum_{i=1}^k d_i \le k(k-1) + \sum_{i=k+1}^n \min(d_i, k)
    \]
\end{enumerate}

\subsection*{Game Theory}

\subsubsection*{Sprague–Grundy Theorem}
\textbf{Pattern:} An \textbf{impartial game} (moves depend on position, not player).
\begin{itemize}
    \item \textbf{Classic Nim:} "A game with multiple piles of stones."
    \item \textbf{Sum of games:} Game breaks into independent sub-games.
\end{itemize}
\textbf{Definition:} For impartial games.
\begin{itemize}
    \item \textbf{Grundy Value (G-value) / Nim-sum:}
    \[
    G(v) = \text{mex}(\{ G(v_i) \mid v \to v_i \text{ is a valid move} \})
    \]
    where $\text{mex}(S)$ is the Minimum Excluded value.
    
    \item \textbf{Losing Position:} $G(v) = 0$.
    \item \textbf{Winning Position:} $G(v) > 0$.
    
    \item \textbf{Sum of Games:} If a game is a sum of independent games $g_1, \dots, g_k$:
    \[
    G_{\text{total}} = G(g_1) \oplus G(g_2) \oplus \dots \oplus G(g_k)
    \]
    where $\oplus$ is the bitwise XOR operator.
\end{itemize}

\subsection*{Trivia}
\textbf{Pythagorean triples}: The Pythagorean triples are uniquely generated by $a=k\cdot (m^{2}-n^{2})$, $b=k\cdot (2mn)$, $c=k\cdot (m^{2}+n^{2})$ with $m > n > 0$, $k > 0$, $\gcd(m,n) = 1$, both $m, n$ not odd.

\textbf{Primes}: $p=962592769$ is such that $2^{21} \mid p-1$, which may be useful. For hashing use 970592641 (31-bit number), 31443539979727 (45-bit), 3006703054056749 (52-bit). There are 78498 primes less than 1\,000\,000.

\textbf{Estimates}: $\sum_{d|n} d = O(n \log \log n)$. % <-- Typo fixed here

\textbf{Prime Gaps}: For primes $> 10^{12}$, the max gap is not definitively known, but a gap of 1600 is a safe upper bound for practical purposes. (The largest known gap is 1550).

\textbf{Prime count}: 5133 upto 5e4. 9592 upto 1e5. 17984 upto 2e5. 78498 upto 1e6. 5761455 upto 1e8.

\textbf{max NOD} $\leq n$: 100 for $n = 5e4$. 500 for $n = 1e7$. 2000 for $n = 1e10$. 200\,000 for $n = 1e19$.

\textbf{max Unique Prime Factors}: 6 upto 5e5. 7 upto 9e6. 8 upto 2e8. 9 upto 6e9. 11 upto 7e12. 15 upto 3e19.

\textbf{Quadratic Residue}: $(\frac{a}{p})$ is 0 if $p | a$, 1 if $a$ is a quadratic residue, -1 otherwise. Euler: $(\frac{a}{p}) = a^{(p - 1) / 2} (\mod p)$ (prime). Jacobi: if $n = p_1^{e_1}\cdots p_k^{e_k}$ then $(\frac{a}{n}) = \prod (\frac{a}{p_i})^{e_i}$.

\textbf{Chicken McNugget:} If $a, b$ coprime, there are $\frac{1}{2}(a - 1)(b - 1)$ numbers not of form $ax + by$ $(x, y \geq 0)$, the largest being $ab - a - b$.

\section*{Template \& Utils}

\subsection*{PBDS (Ordered Set \& Hash Map)}
\textbf{Description:} 
1. \texttt{orderS}: RB-Tree. Supports \texttt{find\_by\_order(k)} ($k$-th element) and \texttt{order\_of\_key(x)} (count strictly less than $x$).
2. \texttt{hash\_map}: Faster than \texttt{std::unordered\_map}. Uses \texttt{custom\_hash} to prevent anti-hash tests.
\begin{lstlisting}
#include <ext/pb_ds/assoc_container.hpp>
#include <ext/pb_ds/tree_policy.hpp>
using namespace __gnu_pbds;
using orderS = tree<ll,null_type,less<ll>,rb_tree_tag,tree_order_statistics_node_update>;

struct custom_hash {
    static uint64_t splitmix64(uint64_t x) {
        x += 0x9e3779b97f4a7c15;
        x = (x ^ (x >> 30)) * 0xbf58476d1ce4e5b9;
        x = (x ^ (x >> 27)) * 0x94d049bb133111eb;
        return x ^ (x >> 31);
    }
    size_t operator()(uint64_t x) const {
        static const uint64_t FIXED_RANDOM =
            chrono::steady_clock::now().time_since_epoch().count();
        return splitmix64(x + FIXED_RANDOM);
    }
};
template <typename K, typename V>
using hash_map = gp_hash_table<K, V, custom_hash>;
\end{lstlisting}

\subsection*{Pragmas \& Optimization}
\textbf{Description:} Aggressive GCC optimizations. \texttt{Ofast} ignores strict IEEE floating point standards (be careful with geometry precision).
\begin{lstlisting}
#pragma GCC optimize("O3")
#pragma GCC optimize("Ofast,unroll-loops")
#pragma GCC optimize("tree-vectorize")
#pragma GCC target("avx2,sse4.2,popcnt")
\end{lstlisting}

\subsection*{Random Number Generator}
\textbf{Description:} Mersenne Twister (\texttt{mt19937}) seeded with high-resolution clock. Much better than \texttt{rand()}.
\begin{lstlisting}
mt19937 rng(chrono::high_resolution_clock::now().time_since_epoch().count());
inline ll getrandom(ll a,ll b) { return uniform_int_distribution<ll>(a,b)(rng); }
\end{lstlisting}

\subsection*{Basic Math Utils}
\textbf{Description:} 
1. \texttt{bigmod}: Modular Exponentiation $\mathcal{O}(\log P)$.
2. \texttt{inversemod}: Modular Inverse using Fermat's Little Theorem (Requires Prime Mod).
3. \texttt{sqrtt}: Integer Square Root (avoids precision errors of \texttt{sqrt}).
\begin{lstlisting}
ll bigmod(ll base, ll power) {
    ll res = 1; ll p = base % mod;
    while (power > 0) {
        if (power % 2 == 1) res = ((res % mod) * (p % mod)) % mod;
        power /= 2;
        p = ((p % mod) * (p % mod)) % mod;
    }
    return res;
}
ll inversemod(ll base) { return bigmod(base, mod - 2); }

int gcd(ll a, ll b) {
    while (b) { a %= b; swap(a, b); }
    return a;
}
ll sqrtt(ll a) {
    long long x = sqrt(a) + 2;
    while (x * x > a) x--;
    return x;
}
\end{lstlisting}

\subsection*{Grid Moves (2D)}
\textbf{Description:} Direction arrays for implicit graphs (grids).
\begin{lstlisting}
// 4 Directions: Up, Down, Left, Right
// 8 Directions: Adds Diagonals
int dx[]={-1, 1 , 0 , 0 , -1 ,-1, 1, 1};
int dy[]={ 0, 0 ,-1 , 1 , -1 , 1,-1, 1};
// up = {-1,0}, down = {1,0}, right = {0,1}, left = {0,-1}
constexpr ld PI = 3.14159265358979323846264338327950288L;
\end{lstlisting}

\subsection*{Fast I/O \& Debug}
\begin{lstlisting}
// Place inside main
ios::sync_with_stdio(0); cin.tie(0);
cout.setf(ios::fixed); cout.precision(10);

// File I/O helper
inline void file() {
#ifndef ONLINE_JUDGE  
    freopen("input.txt", "r", stdin);
    freopen("output.txt", "w", stdout);
#endif
}
// Timer
clock_t start= clock();
cerr << "Time: " <<((double)(clock() - start) / CLOCKS_PER_SEC)<<el;
\end{lstlisting}

\section*{CP Environment Setup}

\subsection*{Setup Procedure}
{ \footnotesize
\textbf{Description:} Guide to establishing the fast-testing environment in a Linux terminal. \\
\begin{enumerate}[leftmargin=*, nosep]
    \item \textbf{Transfer Files:} Create the necessary files (\texttt{stdc.h}, \texttt{template.cpp}, \texttt{cf}, \texttt{rte}, \textbf{\texttt{stress}}, \textbf{\texttt{gen.cpp}}, \texttt{right.cpp}, \texttt{wrong.cpp}) and paste the content below.
    \item \textbf{Permissions:} In the terminal, run: \texttt{chmod +x cf rte stress gen.cpp}.
    \item \textbf{Execution Alias:} Run the following alias commands \textbf{once per session} to enable short commands like \texttt{cf A.cpp}:
    \texttt{\vspace{-0.2em}}
    \begin{center}
    \texttt{alias cf='./cf ' \\ alias rte='./rte ' \\ alias stress='./stress '}
    \end{center}
    \item \textbf{Workflow:} To test \texttt{A.cpp}, type \texttt{cf A.cpp}. To debug, type \texttt{rte A.cpp}. To stress test, type \texttt{stress}.
\end{enumerate}
}

\subsection*{C++ Library Header (\texttt{stdc.h})}
\textbf{Description:} Contains all necessary includes, complex debugging macros, and types. Run \texttt{g++ stdc.h -o stdc.h.gch} once to enable fast precompilation.
\begin{lstlisting}[language=C++]
#include <bits/stdc++.h>
using namespace std;
#define TT template <typename T>

// --- Complex Variadic Template Debugger Logic (Preserved) ---

TT, typename=void> struct cerr_ok : false_type {};
TT> struct cerr_ok<T, void_t<decltype(cerr << declval<T>())>> : true_type {};

TT> constexpr void p1(const T &x);

TT, typename V> void p1(const pair<T, V> &x)
{
    cerr << "{";
    p1(x.first);
    cerr << ", ";
    p1(x.second);
    cerr << "}";
}
TT> constexpr void p1(const T &x)
{
    if constexpr (cerr_ok<T>::value) cerr << x;
    else
    {
        int f = 0;
        cerr << "{";
        for (auto &i : x)
            cerr << (f++ ? "," : ""), p1(i);
        cerr << "}";
    }
}
void p2() { cerr << "]\n"; }
TT, typename... V> void p2(T t, V... v)
{
    p1(t);
    if (sizeof...(v))
        cerr << ", ";
    p2(v...);
}
#ifdef DeBuG
#define dbg(x...) { cerr << "\t\e[93m" << \
    __func__ << ":" << __LINE__ << "[" << #x << "] " << \
    "= ["; p2(x); cerr << "\e[0m"; }
#endif 
\end{lstlisting}

\subsection*{Base Solution File (\texttt{template.cpp})}
\textbf{Description:} The starting file for every problem. Includes the necessary macros and I/O setup.
\begin{lstlisting}[language=C++]
// IIUC_MARK_US
#include "bits/stdc++.h"
using namespace std;

#ifndef DeBuG
#define dbg(...)
#endif

#define sz(x) (int) (x).size()
#define all(x) begin(x), end(x)
#define rep(i, a, b) for (int i = a; i < (b); ++i)
using ll = long long; using pii = pair<int, int>;
using pll = pair<ll, ll>; using vi = vector<int>;
template<class T> using V = vector<T>;

int main() {
    ios_base::sync_with_stdio(0);
    cin.tie(0);
    cout.tie(0);

}
\end{lstlisting}

\subsection*{Fast Compile \& Run (\texttt{cf})}
\textbf{Description:} Compiles the specified file (e.g., \texttt{A.cpp}) with full optimization (\texttt{-O2}) and runs it.
\begin{lstlisting}[language=Bash]
#!/bin/bash
TARGET_FILE=$1

if [ -z "$TARGET_FILE" ]; then
    echo "Usage: cf <source_file.cpp>"
    exit 1
fi

g++ -o sol -Wall -Wextra -std=c++17 -O2 "$TARGET_FILE"

if [ $? -eq 0 ]; then
    echo "--- Running $TARGET_FILE ---"
    time ./sol < input.txt
fi
\end{lstlisting}

\subsection*{Debug Check (\texttt{rte})}
\textbf{Description:} Compiles with Sanitizers to catch memory/integer overflow errors (\texttt{ASAN}, \texttt{UBSAN}).
\begin{lstlisting}[language=Bash]
#!/bin/bash
TARGET_FILE=$1

if [ -z "$TARGET_FILE" ]; then
    echo "Usage: rte <source_file.cpp>"
    exit 1
fi
# Compile with Sanitizers (Address and Undefined Behavior)
g++ -o sol -std=c++17 -O2 -Wall -Wextra \
-fsanitize=address,undefined "$TARGET_FILE"

if [ $? -eq 0 ]; then
    echo "--- Running sanitizer check on $TARGET_FILE ---"
    ./sol < input.txt
fi
\end{lstlisting}

\subsection*{Stress Test Script (\texttt{stress})}
\textbf{Description:} Continuous verification tool. Compiles \texttt{right.cpp} (verified solution) and \texttt{wrong.cpp} (optimized solution) and tests them against random inputs from \texttt{./gen}.
\begin{lstlisting}[language=Bash]
#!/bin/bash
# TARGET FILES: right.cpp (verified) and wrong.cpp (tested/optimized)

# 0. Compile the test case generator
g++ -o gen gen.cpp -std=c++17 -O2

# 1. Compile the verified solution (RIGHT ANSWER)
g++ -o right right.cpp -std=c++17 -Wall -O0 -D_GLIBCXX_DEBUG

# 2. Compile the optimized solution (POTENTIALLY WRONG ANSWER)
g++ -o wrong wrong.cpp -std=c++17 -Wall -O2

for ((i = 1; ; ++i)); do
    echo "Testing case $i..."
    
    # Generate input using the C++ executable, passing the case number $i$ as the seed
    ./gen $i > input.txt 
    
    # Run solutions and capture output
    ./right < input.txt > right.out
    ./wrong < input.txt > wrong.out
    
    # Check if outputs differ
    if ! diff -w right.out wrong.out; then
        echo "--- Found Difference! Failing Case Saved to input.txt ---"
        break
    fi
done
\end{lstlisting}
\end{multicols*}
% --- DOCUMENT END ---
\end{document}
